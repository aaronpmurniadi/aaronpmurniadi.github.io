% arara: lualatex
% arara: lualatex
\documentclass[11pt,twoside,a5paper,openany]{memoir}
%===========================================%
% USEPACKAGES
%===========================================%
%\usepackage[fontsize=11pt,baseline=14pt]{grid}
%\usepackage[bahasa]{babel}
\usepackage{polyglossia}
\setdefaultlanguage[variant=indonesian]{malay}
\usepackage[pagewise,switch*]{lineno}
\usepackage[noindentafter]{titlesec}
\usepackage{titletoc}
\usepackage{titleps}
\usepackage{textcase}
\usepackage{graphicx}
%\usepackage{geometry}
%\newgeometry{
% inner=2cm,
% outer=2cm,
% bottom=2cm,
% top=2cm,
% headsep=1.2em,
%}

%\baselineskip=1.618\fontdimen6\font
%\hsize=319.148446488500742pt
%\vsize=1.618\hsize


\setmainfont{Old Standard}
\usepackage{canoniclayout}
%\usepackage{lmodern} % monospace font
%\usepackage[scale=0.89]{tgheros} % Helvetica is too big
%\usepackage[baskervaldx,cmintegrals,bigdelims,vvarbb]{newtxmath} % math italic letters from Baskervaldx
%\usepackage[cal=boondoxo]{mathalfa} % mathcal from STIX, unslanted a bit
%\usepackage[lining,tabular]{ebgaramond}
%\usepackage[]{NewCaledonia}
%\usepackage{kpfonts}
%\usepackage{coelacanth}
%\usepackage{mathptmx}
%\usepackage{venturis}

%===========================================%
% HEADINGS
%===========================================%
\settitlemarks{chapter,section,subsection}
\renewcommand{\thesection}{\arabic{section}}
\newpagestyle{main}{
  \setheadrule{0pt}
  \sethead[\thepage][\itshape\chaptertitle][]
    {}{\itshape\sectiontitle}{\thepage}
  \setfootrule{0pt}
  \setfoot[][][]
    {}{}{}
}

%===========================================%
% TYPOGRAPHY
%===========================================%
%\usepackage[activate={true,nocompatibility},
%kerning=true,spacing=true,tracking=true,final,
%letterspace=80]{microtype}
%\microtypecontext{spacing=nonfrench}
\usepackage{microtype}
\lefthyphenmin=3
\righthyphenmin=3
\def\tightlist{}
\widowpenalty9000
\clubpenalty9000
\emergencystretch2em
%\usepackage{enumitem} 
%\setlist[enumerate]{itemsep=1pt}
\setlength{\parindent}{0.95em}

%===========================================%
% SECTION TITLE
%===========================================%
\usepackage{remreset}
\makeatletter
  \@removefromreset{section}{chapter}
\makeatother

%\titleformat{\chapter}[block]
%{\center\scshape\huge}
%{}{0em}{}
%\titlespacing{\chapter}{0pt}{15ex plus .1ex minus .2ex}{12ex}

\makeatletter
\def\thickhrulefill{\leavevmode \leaders \hrule height 1ex \hfill \kern \z@}
\def\@makechapterhead#1{%
  %\vspace*{50\p@}%
  \vspace*{0\p@}%
  {\parindent \z@ \centering \reset@font
        \thickhrulefill\quad
        \scshape\large \@chapapp{} \thechapter
        \quad \thickhrulefill
        \par\nobreak
        \vspace*{10\p@}%
        \interlinepenalty\@M
        \hrule
        \vspace*{10\p@}%
        \huge \scshape \MakeUppercase{#1}\par\nobreak
        \par
        \vspace*{10\p@}%
        \hrule
    %\vskip 40\p@
    \vskip 50\p@
  }}
%\def\@makeschapterhead#1{%
%  %\vspace*{50\p@}%
%  \vspace*{10\p@}%
%  {\parindent \z@ \centering \reset@font
%        \thickhrulefill
%        \par\nobreak
%        \vspace*{10\p@}%
%        \interlinepenalty\@M
%        \hrule
%        \vspace*{10\p@}%
%        \Huge \bfseries #1\par\nobreak
%        \par
%        \vspace*{10\p@}%
%        \hrule
%    %\vskip 40\p@
%    \vskip 100\p@
%  }}


\titleformat{\section}[block]
{\center\large\scshape}
{Tesis \thesection:\space}{0em}{}
\titlespacing{\section}{0pt}{5ex plus .1ex minus .2ex}{1.5ex}

\titleformat{\subsection}[block]
{\center\scshape}
{}{0em}{}
%\titlespacing{\subsection}{0pt}{4ex plus .1ex minus .2ex}{1.5ex}

\titleformat{\subsubsection}[block]
{\itshape}
{}{0em}{}
%\titlespacing{\subsubsection}{0pt}{2ex plus .1ex minus .2ex}{.5ex}

\usepackage{etoolbox}
\patchcmd{\quote}{\rightmargin}{\small\leftmargin 1.85em \rightmargin}{}{}

\preto{\section}{\clearpageafterfirst}
\newcommand{\clearpageafterfirst}{%
  \gdef\clearpageafterfirst{\clearpage}%
}
%===========================================%
% TOC STYLES
%===========================================%
\titlecontents{chapter}% <section-type>
  [0em]% <left>
  {\scshape\vspace{6pt}}% <above-code>
  {}% <numbered-entry-format>
  {}% <numberless-entry-format>
  {\it\contentspage\vspace{2pt}}% <filler-page-format>

\titlecontents{section}% <section-type>
  [\quad]% <left>
  {}% <above-code>
  {Tesis\space\thecontentslabel:\space}% <numbered-entry-format>
  {}% <numberless-entry-format>
  {\it\contentspage\vspace{2pt}}% <filler-page-format>

%===========================================%
% TITLE PAGE
%===========================================%
\makeatletter
\def\cover{
\thispagestyle{empty}
\begin{center}
  {\thickhrulefill\par}
  {\hrule\par}
  {\Huge\scshape\uppercase Pembahasan Tesis-tesis\par}
  {\Huge\scshape\uppercase Komprehensif\par}
  {\hrule\par}
  {\vspace*{\fill}\par}
  {\Large Prodi Filsafat\par}
  {\vspace*{\fill}\par}
  \includegraphics[width=4cm,height=4cm]{logo-stfd-lustrum.png}
  {\vspace*{\fill}\par}
  {\Large Sekolah Tinggi Filsafat\par}
  {\Large Driyarkara\par}
  {\vspace*{\fill}\par}
  {\large Jakarta\par}
  {\large 2020\par}
\end{center}
\clearpage
}

\def\catatan{
\vspace*{\fill}
\section*{Catatan Ujian Komprehensif}
\thispagestyle{empty}
\begin{enumerate}
\item Ujian Komprehensif akan diadakan pada bulan Desember dan Mei bagi mereka yang akan menyelesaikan studi Program Sarjana dalam Semester ybs.
\item Lulus dalam ujian Komprehensif adalah syarat untuk memperoleh gelar Sarjana Strata 1.
\item Mataujian yang tidak lulus, dapat diulang satu kali.
\item Dalam Ujian Komprehensif mahasiswa akan diuji dalam tiga tesis, masing-masing dari salah satu dari mata ujian tersebut di atas, masing-masing selama 15 menit. Dari mata ujian manakah tesis yang diuji akan diambil tidak diberitahukan sebelumnya.
\item Mahasiswa diharapkan dapat menguraikan sendiri, secara teratur dan dengan menekankan pokok dan inti permasalahannya, masalah yang termuat dalam tesis yang diujikan.
\item Mahasiswa diuji menurut bahan yang diterimanya sewaktu kuliah. Apabila ada mahasiswa Ujian Komprehensif yang mendapat kuliah dalam mata ujian ybs. Dari pengajar lain daripada yang disebut di atas, ia harus mengajukannya kepada Ketua Panitia Ujian Komprehensif secara tertulis paling lambat empat minggu sebelum Ujian Komprehensif akan berlangsung.
\end{enumerate}
\vspace*{\fill}
\clearpage
}
\usepackage[hidelinks]{hyperref}
\begin{document}
%\cover
\catatan
\setcounter{secnumdepth}{1}
\setcounter{tocdepth}{1}
%===========================================%
% TABLE OF CONTENTS
%===========================================%
\thispagestyle{empty}
\tableofcontents
\clearpage
\begingroup
  \pagestyle{empty}%
  \cleardoublepage
\endgroup
\pagestyle{main}
%\linenumbers
\hypertarget{filsafat-manusia}{%
\chapter{Filsafat Manusia}\label{filsafat-manusia}}

Filsafat Manusia merupakan bagian filsafat yang mencoba mengungkapkan
sebaik mungkin apakah sebenarnya pengada (\emph{being}) yang disebut
manusia itu (LL, 2001:15). Ilmu-ilmu tentang manusia lainnya seperti
antropologi dan psikologi memang berupaya menemukan hukum-hukum tindakan
manusia, tetapi hanya dari segi empiris saja. Setiap bidang juga
mempelajari satu segi saja dari sifat dan tindakan manusia, entah
biologisnya, fisiknya, atau psikologisnya saja. Kesimpulan-kesimpulan
yang didapat akhirnya diuraikan dalam bahasa matematika (statistika)
karena hanya mencakup apa yang dapat diukur dan dihitung. Di lain pihak,
Filsafat Manusia bertanya mengenai apa yang menjadi ciri khas dan
mendasar manusia, apa yang memberinya sifat kesatuan, apa yang menjadi
tindakan khasnya, dan apa sebenarnya yang disebut manusia itu. Maka,
sifatnya lebih fundamental, lebih luas, dan lebih mempersatukan (LL,
2001:23). Objek formal atau bahan penyelidikannya adalah unsur-unsur
pembentuk manusia yang diakui secara mutlak, material maupun immaterial.
Hasil kesimpulan dari kajian filsafat manusia tidak berasal dari
persepsi inderawi, melainkan dari hasil penangkapan intelektual atas
hal-hal yang membuat manusia memiliki sifat keberadaan yang khas.
Filsafat manusia mengandaikan adanya suatu watak-sifat manusia, yaitu
suatu kumpulan struktur dan suatu rangkaian bentuk dinamis yang khas
baginya (LL, 2001:20). Dari banyak aspek yang dapat kita telusuri
mengenai realitas manusia, relasi antarmanusia dan kematian merupakan
dua tema besar yang penting dalam filsafat manusia.

\hypertarget{relasi-antarmanusia}{%
\section{Relasi Antarmanusia}\label{relasi-antarmanusia}}

\begin{quote}
Para filsuf mengajukan pandangan yang berbeda-beda mengenai hakikat
relasi antarmanusia. Dalam konteks dan cara masing-masing, Martin Buber
dan Gabriel Marcel menekankan relasi yang setara dan dialogis. Martin
Heidegger memperlihatkan struktur eksistensial ``Ada-dengan'' pada
manusia yang sering justru membuatnya tenggelam dalam kerumunan.
Sementara itu, Jean-Paul Sartre menyoroti relasi konflik berdasarkan
hakikat kesadaran manusia. Sebaliknya, Emmanuel Levinas menunjukkan
relasi asimetris dengan orang lain yang terungkap dalam tanggung jawab
padanya.
\end{quote}

Manusia tidak hidup sendirian, ia hidup bersama orang lain. Oleh karena
itu, sosialitas merupakan dimensi fundamental manusia. Para filsuf
mengajukan pandangan yang berbeda-beda mengenai hakikat relasi
antarmanusia tersebut. Dalam bab ini, akan dibahas pandangan dari
filsuf-filsuf besar seperti Martin Buber, Gabriel Marcel, Martin
Heidegger, Jean-Paul Sartre, dan Emmanuel Levinas. Masing-masing
mengembangkan pikiran mereka dalam konteks dan caranya yang berbeda.

\hypertarget{pandangan-martin-buber}{%
\subsection{Pandangan Martin Buber}\label{pandangan-martin-buber}}

Menurut Buber, ada dua macam relasi antarmanusia, yaitu Aku-Itu dan
Aku-Engkau. Dalam relasi Aku-Itu, manusia menempatkan orang lain sebagai
objek, (hubungan tuan-budak). Relasi macam ini bersifat fungsional, dan
orang lain dianggap hanya sebagai alat atau sarana untuk tujuan
tertentu. Tetapi dalam relasi Aku-Engkau, manusia menempatkan orang lain
secara setara, sama-sama sebagai subjek yang bermartabat. Relasi macam
ini bersifat dialogis, saya berbicara sekaligus mendengar pihak lain,
dan ada timbal balik. Menurut Buber, relasi Aku-Engkau inilah yang ideal
dalam relasi antarmanusia.

Hubungan Aku-Engkau bukanlah hubungan di antara berbagai hubungan biasa,
tetapi merupakan hubungan yang utama, yang menjadi fakta primer setiap
antropologi dan filsafat. Berbeda dengan hubungan \emph{I-It} yang
merupakan relasi tuan-budak, hubungan \emph{I-Thou} tidak pernah
merupakan hubungan penguasaan Aku atas Engkau atau, sebaliknya, Engkau
atas Aku. Hubungan Aku-Engkau merupakan hubungan dua kutub yang setara,
suatu hubungan timbal balik yang sempurna (\emph{Gegenseitigkeit}).
Dalam perjumpaan dan dialog, manusia secara otentik menjadi Aku dan yang
lain secara otentik menjadi Engkau.

Hubungan Aku-itu adalah hubungan antarmanusia di mana yang satu
menganggap yang lain sebagai objek atau benda, atau sebaliknya. Hubungan
Aku-itu muncul dalam hubungan tuan-budak. Hubungan Aku-itu didominasi
oleh kehendak untuk menguasai orang lain dan dunia. Hubungan ini
mereduksi kemanusiaan dan jauh dari hubungan antarmanusia yang ideal.

Walaupun kurang ideal, tetapi bukan berarti relasi Aku-itu harus dilawan
dan dihindari. Relasi ini mungkin muncul dalam situasi sehari-hari yang
memang bersifat fungsional. Saat kita naik ojek online, misalnya,
pertama-tama relasi antara pengemudi dengan penumpang merupakan relasi
Aku-Itu. Saat pengemudi atau penumpang mulai menyapa, bertukar cerita,
dll., barulah relasi Aku-Engkau terjadi.

Sementara itu, hubungan Aku-Engkau adalah hubungan fundamental manusia.
Hubungan Aku-Engkau mendahului semua hubungan. Tidak ada pengantara
dalam pertemuan Aku-Engkau. Dalam hubungan ini, Engkau berbeda dari
benda, kemanusiaan Engkau tidak pernah aku kebawahkan, dan kemanusiaan
Engkau juga tidak tergantung dariku. Dengan demikian, hubungan
Aku-Engkau ini tidak berada dalam kerangka hubungan tuan-budak. Tidak
pernah ada hubungan penguasaan Aku atas Engkau ataupun sebaliknya. Maka,
implikasi dari hubungan Aku-Engkau adalah hubungan yang setara.

Engkau adalah misteri tak terkatakan yang tak pernah merupakan
pengalaman ilmiah. Engkau tidak pernah merupakan ``objek.'' Engkau
sebagai yang lain tak pernah secara penuh ``kuketahui,'' tetapi hadir
sebagai misteri yang tak bisa ditangkap, dan menjadi sumber refleksi
atas hubunganku dengan Yang Ilahi. Hubungan antarmanusia (Aku-Engkau)
tidak terpisahkan dari hubungan dengan Allah: perjumpaan dengan Engkau
juga merupakan jalan menuju Allah. Hubungan antarmanusia mengaitkan kita
dengan hubungan dengan Engkau yang Mutlak. Relasi Aku-Engkau bersifat:

\begin{enumerate}
\def\labelenumi{\arabic{enumi}.}
\tightlist
\item
  \emph{Langsung hadir}: kedua subjek sepenuhnya hadir dalam suatu
  relasi. Pikirannya tidak melayang ke mana-mana.
\item
  \emph{Tanpa prasangka}: subjek yang satu tidak berprasangka dari awal
  mengenai subjek yang lainnya, tetapi menerima sebagaimana adanya ia
  sekarang.
\item
  \emph{Setara}: kedua subjek berada pada tingkat yang sama, bukan
  relasi tuan-hamba, atau bos-karyawan.
\item
  \emph{Merupakan jalan menuju Allah}: menurut Buber, relasi Aku-Engkau
  dapat menjadi jalan menuju Allah.
\end{enumerate}

\hypertarget{pandangan-gabriel-marcel}{%
\subsection{Pandangan Gabriel Marcel}\label{pandangan-gabriel-marcel}}

Gabriel Marcel berbicara tentang relasi antarmanusia seperti yang
digambarkan Martin Buber, yaitu dialogis, namun sifatnya lebih intim.
Marcel melihat filsafat tradisional cenderung melihat dunia, orang lain,
bahkan diri kita sendiri sebagai objek yang perlu dikenali, yang perlu
diselidiki atau diketahui. Tetapi menurut Marcel, seharusnya relasi
dengan para `Ada' (\emph{beings}, yang termasuk di dalamnya manusia)
seharusnya terungkap dalam partisipasi (menikmati kehadiran), dan bukan
sekadar observasi atau pengamatan.

Marcel melihat manusia sebagi ada-yang-menjelma (\emph{l'etre incare}).
Pengalaman utama manusia yang mendalam adalah pengalaman antarmanusia
yang termanifestasi dalam cinta. Jadi, sebagaimana dalam pemikiran
Martin Buber, hubungan antarmanusia yang ideal merupakan hubungan
Aku-Engkau, yang menurut Gabriel Marcel memiliki ciri-ciri (1)
Ketersediaan (\emph{Disponibilité}), yaitu siap hadir untuk orang lain;
(2) Penerimaan (\emph{Réceptivité}), yaitu secara aktif menerima atau
memberikan perhatian dan bersedia mendengarkan; (3) Keterlibatan
(\emph{Engagement}), yaitu tidak hanya mengamati, tetapi ambil bagian,
bukan sikap acuh tak acuh; (4) Kesetiaan (\emph{Fidélité}), yaitu
menganggap penting relasi dengan orang lain, ada komitmen, ikut
bertanggung jawab atas rencana dan proyek bersama; (5) Kreativitas
(\emph{Créativité}), yaitu mengembangkan relasi yang sudah ada, dalam
arti membuat ruang untuk terwujudnya kebebasan yang lain dan realisasi
diri, membiarkan orang lain berkembang.

Persoalan yang dilihat Gabriel Marcel adalah hubungan antarmanusia dalam
masyarakat modern mengalami kemerosotan karena telah diperbudak oleh
mesin dan teknologi. Itu membawa manusia pada semangat abstraksi yang
dominatif. Hal ini memerosotkan relasi Aku-Engkau---yang sebenarnya
merupakan fakta primer dan kecenderungan utama manusia yang diperlukan
bagi eksistensinya---menjadi sekadar hubungan subjek-objek atau
\emph{I-It} (Aku-itu).

Filsafat harus bertolak dari situasi konkret, dan pengalaman langsung
sangat kaya untuk digali. Karena itu, Marcel juga berpendapat bahwa
perilaku manusia tidak dapat ditangkap sepenuhnya dengan suatu kejelasan
sistematik dan abstraksi saja. Bagi Marcel, manusia dan hubungan
antarmanusia harus didekati sebagai misteri. Dalam pemikirannya, ia
membedakan dua pendekatan terhadap realitas antarmanusia: (1) memandang
yang lain sebagai ``problem'' atau ``objek,'' dan (2) sebagai
``misteri.''

Kata ``problem'' berasal dari kata dalam bahasa Latin ``\emph{pro
ballo},'' artinya `melempar objek di hadapanku.' Maka, \emph{problem}
berarti objek yang dilemparkan di hadapanku. Memandang manusia sebagai
\emph{problem} berarti mengobjekkan manusia yang ada di hadapanku.
Mengapa? Karena: (a) aku hanyalah pengamat, (b) ada jarak yang
memisahkan aku dengan yang lain, dan (c) manusia dipandang sebagai
``benda'' atau ``\emph{It}'' (dalam bahasa Buber).

Memandang manusia sebagai misteri berarti mengakui adanya suatu
kedalaman yang tidak bisa ditimba sampai habis dalam kerangka sistematik
ilmiah. Misteri berarti sesuatu yang mengejawantahkan diri padaku untuk
terlibat di dalamnya. Aku tidak bisa mengenal atau mengetahui dia lebih
dalam karena selalu ada sesuatu yang tidak aku ketahui (masih
tersembunyi). Dalam perjalanan hidup, yang tersembunyi itu akan
termanifestasi ketika aku terlibat untuk mengenalnya (menekankan adanya
keterlibatan). Saat itulah manusia lain dipandang sebagai
``\emph{Thou}'' (dalam bahasa Buber).

Marcel melihat manusia sebagai ``ada-yang-menjelma'' (\emph{l'être
incaré}). Artinya, manusia harus dilihat dalam keutuhan dirinya. Dia
juga adalah ``peziarah'', ``\emph{the becoming},'' yang tidak pernah
selesai berproses, terbentuk, atau menjadi (dalam bahasa Driyarkara,
manusia disebut dengan makhluk `membelum'). Bagi Marcel, pengalaman
utama manusia yang paling mendalam adalah pengalaman antarmanusia, yang
termanifestasi dalam cinta. Dalam cinta itulah terjadi realisasi diri
tertinggi.

Dalam arti tertentu, Marcel juga dapat dilihat sebagai pemikir
`\emph{anti-sistem}.' Yang dimaksud dengan `sistem' adalah pengelompokan
atau penggolongan manusia (generalisasi atau \emph{stereotyping}).
Ketika manusia dimasukkan ke dalam sistem, keunikan diri manusia tidak
lagi diakui.

\hypertarget{pandangan-martin-heidegger}{%
\subsection{Pandangan Martin
Heidegger}\label{pandangan-martin-heidegger}}

Dalam membicarakan relasi antarmanusia, Heidegger tidak mencari
bagaimana relasi `yang ideal', melainkan ia melihat dari sudut struktur
yang ada pada manusia itu sendiri. Menurutnya, salah satu struktur
eksistensial manusia sebagai \emph{Da-Sein} (Ada-di-Sana) adalah
\emph{Mit-Sein} (Ada-Dengan: berada dengan benda-benda dan orang lain
dalam dunia), dan secara keseluruhannya manusia adalah
\emph{Mit-Dasein}, yaitu ada-dengan-orang-lain.

Dengan demikian, keprihatinan (\emph{sorge}) setiap Dasein bukan hanya
keprihatinan terhadap benda-di-dunia, tetapi juga \emph{Fursorge}, yaitu
keprihatinan terhadap manusia lain (kepedulian). Entah ia berusaha
menyanjung yang lain ataupun menjatuhkan yang lain, satu kenyataan yang
tidak terbantahkan adalah bahwa manusia tidak pernah terlepas dari yang
lain. Dari struktur ini, tampak bahwa manusia memang tidak bisa hidup
terpisah dari orang lain. Namun, struktur itu membuat manusia sering
hilang dalam kerumunan masyarakat (\emph{crowd/they}). Dalam kerumunan,
yang ada adalah hubungan impersonal (tanpa pribadi, tanpa individu,
non-individual, anonim). Hubungan impersonal ini adalah bentuk tirani
terhadap manusia, karena menempelkan anonimitas pada identitas individu.
Keunikan individu menjadi hilang. Di situ, manusia bukan lagi subjek
tertentu, bukan aku, bukan kamu, bahkan bukan kita.

Lebih lanjut, yang impersonal tersebut mengatur individu untuk
menghilangkan tanggung jawab pribadi. Semua tindakannya hanya
ikut-ikutan. Dengan itu, kehidupan sosial sehari-hari menjadi kawasan
yang impersonal, yang ditandai dengan \emph{heteronomi} dan
ketidakotentikan, manusia tidak lagi memiliki dirinya, tidak bisa
menentukan dirinya sendiri. Maka, manusia sebagai
ada-di-dalam-dunia-dengan-yang-lain sering kali justru membuatnya
tenggelam dalam kerumunan masyarakat.

\hypertarget{pandangan-jean-paul-sartre}{%
\subsection{Pandangan Jean-Paul
Sartre}\label{pandangan-jean-paul-sartre}}

Menurut Sartre, ada-bersama (\emph{Mit-Dasein}) Heidegger tidak
menunjukkan adanya hubungan pengakuan dan perjuangan timbal balik.
Padahal, eksistensi yang lain tidak dapat dipisahkan dari aku atau
\emph{Ego}. Manusia bagi Sartre pada hakikatnya adalah sadar dan bebas
(berada pada tingkat \emph{being-for-itself}). Kesadaran ini selalu
berusaha menjadikan orang lain sebagai objeknya. Dalam relasi dengan
orang lain, manusia berusaha menghindar dari situasi yang menjadikan
saya objek dari kesadaran orang lain, karena itu akan mengurangi
kebebasan saya. Itulah kenapa bagi Sartre, `orang lain adalah neraka'.
Jadi, relasi antarmanusia menurut Sartre adalah relasi yang saling
mengobjekkan.

Realitas manusia adalah pelaku yang sama sekali bebas, yaitu pengada
yang membuat dirinya sebagaimana ia kehendaki. Kebebasan menjadi
satu-satunya sumber nilai bagi martabat manusia. Masalahnya, kebebasanku
selalu terancam oleh kehadiran yang lain. Apabila aku dilihat atau
dipandang orang lain, aku menjadi objek. Sebaliknya, jika aku memandang
atu melihat orang lain, ia menjadi objek. Begitulah relasi antarmanusia.
Ketika yang lain memandang aku sebagai objek, ia membendakan aku menjadi
entitas-di-antara-entitas-di-dunia, membuatku kehilangan martabat, dan
juga mengurangi kebebasanku karena sekarang tindakanku dibatasi oleh
pandangan orang lain. Oleh karena itu, aku lalu berjuang melepaskan diri
dari cengkraman yang lain. Akan tetapi, pada saat yang sama yang lain
pun berusaha membebaskan diri dariku. Di situlah hubungan timbal balik
sangatlah penting yaitu berupa dinamika perjuangan timbal balik. Bagi
Sartre, relasi antarmanusia selalu berbentuk konflik.

Pandangan Sartre atas relasi antarmanusia sama dengan pandangan Thomas
Hobbes tentang situasi masyarakat sebagai ``perang semua melawan
semua.'' Setiap kesadaran selalu dalam potensi konflik dengan kesadaran
yang lain. Kesadaran manusia akan kebebasan eksistensial---yang harus
direalisasikan itu---akhirnya bertabrakan dengan kebebasan eksistensial
yang lain.

Kehadiran Yang Lain tak dapat dipisahkan dengan ``aku.'' Yang lain
adalah yang kulihat dan yang melihat aku. Bagi Sartre, hubungan timbal
balik ini sangat penting: bila aku dilihat orang lain, aku menjadi
objek. Maka, antarsubjektivitas itu mustahil; yang ada adalah
antarobjek, dan yang mungkin adalah konflik. Kebebasan Yang Lain
mengancam kebebasanku. Dalam pemikirannya, Sartre membedakan dua
kategori realitas: \emph{Being-for-itself} dan \emph{Being-in-itself}

\emph{Being-for-itself} adalah manusia (subjek) yang memiliki kesadaran.
Manusia menyadari bahwa ``aku bukanlah Yang Lain.'' Akan tetapi, karena
dalam perjalanan hidupnya manusia selalu berbeda atau berubah, ia juga
sadar bahwa ``aku bukanlah diriku,'' atau ``aku tidak selalu diriku yang
sekarang.'' Ia tidak identik dengan dirinya (aku sekarang tidak sama
dengan aku masih bayi), dan identitas `aku yang sekarang' juga
sebenarnya belum dapat tercapai atau ditemukan. Dengan begitu, dalam
arti tertentu, ``aku bukanlah diriku'' juga dapat berarti ``aku tidak
memiliki identitas,'' atau ``aku bisa menyangkal diriku.''
Ketidakmampuan aku untuk mendefinisikan diri ini menimbulkan kecemasan
dan frustrasi.

\emph{Being-in-itself} adalah benda (objek) yang tidak memiliki
kesadaran. Benda-benda identik dengan dirinya. Ia juga tidak mengalami
frustrasi. Yang lain adalah yang kulihat dan yang melihat aku. Hanya
bila aku berada di hadapan yang lain, aku merasa menjadi objek. Aku
menjadi \emph{malu} (menjadi objek) karena perbuatan yang tak seharusnya
dilihat orang ternyata dilihat orang lain.

Tetapi di satu sisi, malu adalah pengakuan. Aku mengakui siapa diriku di
hadapan yang lain. Di saat itu, ia (yang lain) menjadi objek dan aku
menjadi subjek yang mengobjekkan yang lain. Contohnya, bila seseorang
mengintip aku di pintu, aku menjadi objek. Tetapi ia akan menjadi malu,
dan juga akan menjadi objek bila ia kuketahui sedang mengintip. Maka,
hubungan antarsubjektivitas itu mustahil; yang mungkin adalah konflik.

Realitas manusia adalah pelaku yang sama sekali bebas, yang membuat
dirinya sekehendaknya. Sumber nilai (martabat) manusia hanyalah
kebebasannya. Tetapi kebebasanku selalu terancam oleh kehadiran yang
lain, karena dengan memandang aku sebagai objek, yang lain ``membendakan
aku, menjadikan diriku suatu benda di antara benda.'' Yang lain
menjadikan kesadaran subjektif (\emph{pour-soi}) menjadi benda
(\emph{en-soi}) oleh pandangan (\emph{le regard}) yang lain. Begitu aku
diobjekkan, aku kehilangan kebebasanku. Untuk mendapatkan kebebasanku
kembali, aku harus melepaskan diri dari cengkeraman yang lain, begitu
pula sebaliknya.

Dalam kerangka berpikir seperti itu, Sartre dapat menjelasakan berbagai
bentuk hubungan kita dengan orang lain. Mencintai, misalnya, berarti
menempatkan yang lain sebagai pusat perhatian, sekaligus bersedia
menjadikan diri sebagai objek bagi yang lain. Saling mencintai berarti
saling mengidentifikasi, saling menyingkirkan keunikan atau
ke-\emph{lain}-an yang satu dengan yang lain. Saya juga dapat mengatur
kehendakku ketika orang lain menolak diriku, aku sadar dan mengakui
bahwa dia punya kehendak sendiri, aku punya kehendak sendiri. Sementara
itu, hubungan yang `seksual saja' berarti aku menjadikan yang lain
sebagai objek. Perhatian kepada orang lain itu semu, yang dipentingkan
adalah individualisme dan kenikmatan sendiri.

Masokisme berarti aku membiarkan diri dikendalikan yang lain, memberi
kebebasan bagi yang lain untuk memanipulasi diriku. Dengan mengambil
posisi sebagai yang dimanipulasi, orang lain juga menjadi objek dari
keinginanku. Sementara sadisme berarti aku secara aktif memanipulasi
yang lain, yang lain tunduk dan mengikuti kehendakku tanpa syarat.
Tetapi, kesadaran yang lain sebagai \emph{being-for-itself} tidak mati,
sesungguhnya ia menolak kumanipulasi.

\hypertarget{pandangan-emmanuel-levinas}{%
\subsection{Pandangan Emmanuel
Levinas}\label{pandangan-emmanuel-levinas}}

Menurut Levinas, pada dasarnya manusia memiliki kecenderungan untuk
mementingkan egonya saat berelasi dengan orang lain. Kecenderungan itu
disebut `Kecenderungan \emph{The Same}'. Karena kecenderungan itu harus
dilawan, dalam berelasi manusia harus berusaha untuk sungguh-sungguh
mengutamakan orang lain (transendensi total---\emph{trans-asendensi}).
Dengan begitu, orang lain ditempatkan sebagai yang lebih tinggi. Hal ini
terungkap dalam bentuk suatu tanggung jawab kepada orang lain. Jadi,
relasi antarmanusia yang ideal menurut Levinas bersifat asimetris: Orang
lain lebih penting daripada saya. Menurut Levinas, \emph{Wajah} orang
lain itu selalu menuntut pertanggungjawaban dari aku. Wajah orang lain
juga merupakan jejak dari Yang Tak Terbatas (\emph{the infinite}).

Gagasan dasar Levinas itu bertumpu pada kritiknya terhadap filsafat
Barat (lihat bagian ``Kritik Levinas terhadap filsafat'' di Bab
Metafisika), khususnya pada konsep \emph{egologi} yang ditujukan pada
pemikiran Descartes, dan menegaskan Yang Lain sebagai paling utama
sebagai kebenaran fundamental manusia. Egologia filsafat barat
menempatkan totalitas sebagai pusat, yang menandakan keinginan akan
kekuasaan. Inilah hubungan antara Aku dan benda-benda (objek).

Gagasan Levinas tentang relasi antarmanusia dapat dimengerti dengan
menjelaskan pokok-pokok pikirannya, yaitu: Apa yang dimaksud Emmanuel
Levinas dengan ``Wajah Orang Lain''? Bagaimana hal itu dikaitkan dengan
``Yang Etis''? Bagaiamana ``Tiga Momen Epifani Wajah'' menghubungkan
kedua pokok pikiran itu?

Wajah orang lain dalam pemikiran Levinas tidak sekadar merujuk pada rupa
fisik atau kehadiran inderawi orang lain. Wajah bagi Levinas bukan
semata-mata soal fisik, melainkan cara `Yang Lain' memperlihatkan
dirinya, melampaui gagasan yang saya miliki mengenai dirinya. Melalui
Wajah, orang lain menampakkan diri seutuh-utuhnya dan melampaui apa yang
bisa saya pikirkan mengenai orang itu. Wajah juga bukan hanya persepsi,
konsep, ide atau pikiran yang dapat ditanggkap dari orang lain. Wajah
orang lain sama sekali lain dari pikiran kita tentang dia. Lebih dari
itu, wajah orang lain merupakan keseluruhan cara yang lain itu berada
dan mengungkapkan dirinya secara penuh. Ia sama sekali tidak terikat
dengan konteks dimana ia ditemukan melainkan sungguh-sungguh
mencerminkan totalitas dirinya, yang melampaui konsepsi tentangnya atau
penginderaan tentangnya.

Dihadapan Wajah yang seperti itu, kita tidak bisa `mengetahui' apa-apa.
Semua usaha kita untuk menguasai Wajah akan selalu gagal, kita tidak
bisa mendapatkan pengetahuan, dan---karena `mengetahui' bagi Levinas
adalah juga `menguasai'---kita tidak dapat menguasai Wajah Yang Lain
itu. Di hadapan Wajah, yang kita `dapatkan' hanyalah \emph{Epifani},
yaitu suatu ``manifestasi tiba-tiba atas esensi atau makna realitas
tertentu'' (THT 82). Epifani itu pun bukan kita dapatkan karena usaha
kita untuk mendapatkannya, tetapi karena keterbukaan kita untuk
menerimanya. Epifani tersebut oleh Levinas dibagi menjadi \textbf{Tiga
Momen Epifani:}

\begin{itemize}
\tightlist
\item
  Momen pertama adalah keberadaan yang lain dalam \emph{wajah yang
  tegak}. Yang lain menampakkan diri dalam wajah yang tidak terlindungi,
  rentan, telanjang dan menderita.
\item
  Momen kedua adalah momen \emph{face-to-face}. Dalam momen ini aku
  berjumpa secara langsung dan berhadapan dengan yang lain.
\item
  Momen ketiga adalah momen yang lain \emph{menuntut aku untuk
  bertanggung jawab}. Dalam perjumpaan dengan wajah yang lain, yang lain
  ini tidak hanya diam. Ia mengusik dan mengganggu
  kebebasan-kenyamananku. Ia meminta pertanggungjawaban atas keberadaan
  dirinya.
\end{itemize}

Dalam tiga momen Epifani di atas, `Yang Etis' menurut Levinas terjadi
dalam momen ketiga. Dalam perjumpaan dengan wajah yang lain, yang lain
ini tidak hanya diam. Ia mengusik dan mengganggu kebebasan-kenyamananku.
Ia meminta pertanggungjawaban atas keberadaan dirinya. Dengan demikian,
relasi yang terbentuk dari perjumpaan ini adalah relasi asimetris. Yang
Lain berada di posisi yang lebih tinggi dari aku.

Dalam relasi asimetris demikian, makna dari \emph{yang etis} ditemukan.
Yang etis adalah keadaan di mana yang lain mengusikku dan meminta
pertanggung jawabanku atas keberadaaanya. Lebih dari itu, aku diminta
untuk menjawab iya dan bertanggung jawab atasnya. Levinas memilih untuk
menggunakan kata `Yang Etis' dari pada `Etika' supaya kita tidak
terjebak hanya pada teori moral yang abstrak (yang melupakan bahwa dalam
relasi etis kita selalu sedang berhadapan dengan orang lain, dan bukan
hanya masalah bagaimana aku bertindak).

Wajah secara langsung menghancurkan kekuatan totalitasi yang bersifat
anonim dan impersonal. Kalau kita ingin `menuntut' orang maka kita harus
hadir di depannya. Kata pertama dari wajah adalah ``Jangan Bunuh Saya.''
Maka di sini bisa dipahami bahwa `Yang Etis' itu pada pertemuan dengan
orang lain. Etika hanya masuk akal ketika ada pertemuan dengan orang
lain. `Yang Etis' itu bersifat cair dan harus menjadi landasan dalam
pertemuan dengan orang lain. Oleh karena itu, etika harus dimulai dengan
sensibilitas terhadap penyingkapan wajah pada kita yang menuntut kita
untuk bertanggung jawab.

Kecenderungan \emph{The Same} (Yang Sama) selalu mereduksi segala
sesuatu dengan konsep-konsep yang menyeragamkan semua (semua
disamaratakan). Filsafat yang bertolak dari Ego mereduksi segalanya
sebagai ``objek'' yang dapat disamaratakan atau diseragamkan. Melihat
segalanya dari kacamata Ego merupakan mengungkapkan rasa ingin menguasai
segalanya. Sementara itu, \emph{The Other} (Yang Lain) itu tidak dapat
direduksi, diseragamkan, atau disamaratakan. Yang Lain tidak sama dengan
diriku. Memandang manusia sebagai Yang Sama berarti mereduksi dan
menundukkannya. Sebaliknya, Memandang manusia sebagai Yang Lain berarti
menganggap Yang Lain sebagai yang khas, mengakui perbedaan Yang Lain
dari diriku.

``Yang baik'' (\emph{the Good}) adalah yang dicari dalam perjumpaan
dengan Yang Lain. Di dalam pencarian ini memang terdapat risiko, yaitu
ketidakpastian, salah penafsiran dan pengenalan akan terus ada sampai
kita benar-benar berjumpa dengan Yang Lain. Perjumpaan dengan ``Yang
Lain'' membawa manusia pada pengalaman metafisika (perjumpaan
memunculkan kesadaran akan Ada Yang Lain) dan pengalaman religius. Dalam
hubungan Aku dengan Yang Lain, ada seberkas cahaya kemuliaan Ilahi yang
termanifestasi. Mengapa hanya `seberkas'? Karena memang hubungan ini
tidak menyebabkan Allah benar-benar hadir. Allah itu \emph{Totally
Other} (Yang Sama Sekali Lain).

Di dalam perjumpaan itu, Yang Lain menampakkan diri (epifani). Yang Lain
itu tampil di depan saya, tampak dengan wajahnya. Wajah Yang Lain
mengundang kebaikan saya terhadapnya, menarikku, menyandera, dan
memberikan kewajiban kepadaku, terutama wajah-wajah mereka yang
menderita. Dengan demikian, pengakuan akan ``Yang Lain'' menjadi
tuntutan etis. Oleh karena itu menurut Levinas, Etika adalah filsafat
yang pertama dan utama. Perbuatanku atau perjumpaanku dengan Yang Lain
tidak dinilai secara ontologis (semata-mata yang penting adalah
kehadiranku), melainkan melibatkan segala tindakan dan pengakuanku akan
keistimewaan Yang Lain.

Menurut Levinas, filsafat pertama adalah Etika, yaitu berhadapan dengan
Yang Lain. Etika tidak sekadar berhadapan dengan perintah moral seperti
dalam pemahaman Kant, tetapi lebih lagi berhadapan dengan Yang Lain.
Etika merupakan panggilan manusia untuk merespons Yang Lain, yaitu
menghargai keunikan dan ketidaksamaan (\emph{alteritas}) dari Yang Lain.

\hypertarget{pokok-pokok-lain}{%
\subsection{Pokok-pokok Lain}\label{pokok-pokok-lain}}

\hypertarget{apa-yang-dimaksud-dengan-relasi-antarmanusia}{%
\subsubsection{Apa yang dimaksud dengan ``Relasi
Antarmanusia''?}\label{apa-yang-dimaksud-dengan-relasi-antarmanusia}}

Relasi Antarmanusia adalah struktur mendasar dari kenyataan bahwa
manusia hidup bersama orang lain dalam dunia. Struktur ini juga sering
disebut sebagai `sosialitas manusia.' Relasi antarmanusia lebih luas
daripada relasi antarsubjek. Manusia (\emph{Ego}) menjadi nyata dan utuh
ketika berelasi dengan manusia lain. Ia menyadari bahwa ``saya sedang
berelasi dengan orang lain.'' Pandangan `hubungan intersubjektivitas'
dalam filsafat abad XX ini merupakan reaksi terhadap hilangnya pemahaman
atas Ego (manusia utuh) dalam filsafat modern.

\hypertarget{mengapa-ego-dikatakan-hilang-dalam-rasionalisme-idealisme-dan-empirisme}{%
\subsubsection{Mengapa `Ego' dikatakan hilang dalam Rasionalisme,
Idealisme, dan
Empirisme?}\label{mengapa-ego-dikatakan-hilang-dalam-rasionalisme-idealisme-dan-empirisme}}

Sebabnya adalah dalam kerangka berpikir rasionalisme dan idealisme, Ego
dipandang tidak lebih daripada sekadar syarat untuk terbentuknya
pengetahuan. Penjelasannya demikian: antropologi modern Descartes
dicirikan oleh cara memandang Ego sebagai ``\emph{yang terarah pada
dunia materiil}.'' Manusia ditafsirkan sebagai individu menyendiri,
tertutup dalam dirinya sendiri, dan terisolasi dari manusia lain.
Hubungan dengan manusia lain tidak disangkal, tetapi juga tidak
dipandang penting dalam memahami misteri manusia.

Terlebih lagi, keberadaan manusia lain hanya diakui secara tidak
langsung. Keberadaan manusia lain diafirmasi melalui proses penalaran
sebagai berikut: (1) Aku mengetahui diriku (bahwa aku sadar diriku
adalah manusia); (2) Lalu, aku mengetahui bahwa aku dapat
mengekspresikan diriku melalui tubuh, kata-kata, senyuman, gerak-gerik
dan lain-lain; (3) Ditemukan bahwa di antara benda-benda materiil di
sekitarku, ada beberapa ekspresi yang analog (mirip) dengan ekspresi
interioritasku.; (4) Aku menarik kesimpulan bahwa ekspresi itu
disebabkan oleh adanya subjek yang sama dengan egoku; (5) Jadi,
\emph{Ego} (identitas manusia utuh) itu merupakan hasil proyeksi dari
olah rasio yang kulakukan.

Dalam Idealisme pasca-Kantian, hilangnya Ego menjadi semakin terlihat.
Kant mereduksi Ego sebagai prasyarat \emph{a priori} yang mengatur
pengalaman manusia supaya pengetahuan dapat diperoleh. Dengan kata lain,
Ego terutama bukan dipandang sebagai manusia utuh, melainkan sebagai
prasyarat yang memang harus ada di dalam struktur untuk mencapai
pengetahuan. Sementara itu, dalam Rasionalisme, Ego diekstremkan menjadi
Ego yang melampaui individu; individu konkret merupakan perwujudan diri
dari suatu Ego Absolut, (atau oleh Hegel disebut juga Roh Absolut) yang
dapat berpikir secara rasional dan objektif. Permasalahan manusia
konkret ketika ia berada bersama yang lain dalam dunia sama sekali
diabaikan.

Dalam kerangka yang sama, di mana Ego menyendiri dan terarah terutama
pada dunia materiil, empirisisme menambah hilangnya Ego dengan
menghilangkan konsep Ego yang bertubuh dan terkait dengan dunia
material. Bagi David Hume, misalnya, Ego pada dasarnya adalah hasil
berbagai kesan, impresi, dan ide. Ego itu dibentuk oleh kesan-kesan atay
opini dari luar (kata orang, aku ini A, B, C, dst.) serta dikaitkan
dengan benda lain (di sekitarku, di sekitar Ego-ku, terdapat piring,
sendok, garpu, dan makanan, yang ketika semuanya terhubung dalam
kegiatan makan barulah dapat dikatakan bahwa aku makan). Jadi, aku
sendiri tidak pernah memutuskan ``siapakah diriku.''

\hypertarget{realitas-kematian}{%
\section{Realitas Kematian}\label{realitas-kematian}}

\begin{quote}
Sementara Martin Heidegger memandang kematian sebagai bagian dari
struktur ontologis eksistensi manusia dan tolok ukur autentisitasnya,
Jean-Paul Sartre dan Albert Camus, dengan alasan yang berbeda,
melihatnya sebagai faktor yang menciptakan absurditas dalam kehidupan
manusia. Meskipun demikian, Sartre dan Camus menolak tindakan bunuh diri
sebagai solusi atas absurditas ini. Terlepas dari berbagai pandangan
ini, haruslah diakui bahwa kematian memiliki sejumlah nilai edukatif.
\end{quote}

Para filsuf memiliki pandangan yang berbeda-beda mengenai kematian,
suatu realitas yang tidak terhindarkan dalam kehidupan manusia. Berikut
ini akan disarikan pandangan Martin Heidegger, Jean-Paul Sartre dan
Albert Camus. Perbedaan-perbedaan refleksi filosofis para pemikir ini
tidak menghalangi kita untuk dapat mengambil sejumlah nilai edukasi dari
realitas kematian.

\hypertarget{pandangan-martin-heidegger-1}{%
\subsection{Pandangan Martin
Heidegger}\label{pandangan-martin-heidegger-1}}

Heidegger memandang kematian sebagai bagian dari struktur ontologis dari
eksistensi manusia. Selain itu, kematian menjadi tolok ukur autentisitas
manusia. Pemikirannya berangkat dari kenyataan bahwa kematian itu tidak
terhindarkan. Ia tertanam sejak awal dalam struktur ontologis dari
eksistensi manusia. Maka, eksistensi manusia dapat didefinisikan sebagai
\emph{Sein-zum-Tode}, ada-menuju-kematian. Akan tetapi, dengan itu pula,
kematian memungkinkan kehidupan memiliki makna.

Bayangkan bila manusia tidak dapat mati. Hidup tidak akan memiliki pola
atau keutuhan, hanya merupakan rangkaian peristiwa tanpa makna seperti
halnya kalimat tanpa titik. Sebaliknya, kehidupan akan memiliki makna
dan kesatuan apabila ada akhir hidup, sebagai suatu batas yang
memberikan perspektif (horizon). Kematian memungkinkan kita melihat
kehidupan sebagai keseluruhan (totalitas) yang terbatas dan kita dapat
menghayatinya dengan suatu tujuan dan daya kekuatan, dalam bayangan
kematian. Di situlah autentisitas manusia sebagai eksistensi terwujud,
yaitu ketika manusia dengan dingin dan realistis menghadapi kehidupan
yang tak terelakkan dari kematian. Dalam menghidupi kematian, tak ada
orang lain yang dapat berpartisipasi, tak ada orang lain yang
menggantikan. Manusia menghadapinya dengan tanggung jawabnya sendiri,
dalam kesepian yang sempurna dan penuh. Demikianlah, kematian menjadi
jalan dan tolak ukur autentisitas manusia.

Dalam pemikiran Heidegger, kematian terkait dengan keotentikan manusia.
Manusia dan hewan berbeda, termasuk kematian-kematiannya. Manusia
menyadari bahwa kematian manusia adalah kematian yang disadari (bermakna
atau tidak). Manusia itu sudah tahu bahwa dia harus mati dan harus
menerima kenyataan itu. Oleh karena itu, Heidegger menyatakan bahwa
manusia itu mati, sedangkan hewan itu musnah karena hewan tidak dapat
menyadari kematiannya. Kematian adalah akhir dari posibilitas,
ketidakmungkinan dari kemungkinan-kemungkinan yang kita miliki selama
hidup. Namun, kematian itu juga membuat hidup manusia bermakna.

Manusia menjadi otentik dengan menerima dan mengakui kematiannya.
Manusia itu sudah tahu bahwa dia harus mati dan harus menerima kenyataan
itu. la menjadi sadar bahwa kematian tetap akan ia jalani sendiri tanpa
partisipasi orang lain. Saat-saat kematian tidak diketahui sehingga
manusia hanya menantikan kematian. Dengan tahu bahwa manusia itu mati,
manusia dapat merencanakan hidup agar bermakna (bertujuan baik). Bila
tidak ada mati, manusia akan terus maju tanpa makna. Hidup akan bermakna
bila ada mati. Dengan kematian, manusia dapat melihat keseluruhan
hidupnya, merangkum keseluruhan hidupnya sehingga bermakna. Akan tetapi,
harus diakui bahwa bagi manusia yang mati, dunia menjadi tidak enak lagi
dihuni karena manusia (mati) tidak bisa lagi mengolah dunia.

\hypertarget{pandangan-jean-paul-sartre-1}{%
\subsection{Pandangan Jean-Paul
Sartre}\label{pandangan-jean-paul-sartre-1}}

Sartre melihat kematian sebagai faktor yang menciptakan absurditas dalam
kehidupan manusia. Kematian yang sifatnya mendadak dan menimpa siapa pun
itu membuat orang merasa frustrasi dan tanpa arti. Kematian
menyingkirkan semua makna kehidupan, demikian ungkapan Sartre. Pandangan
ini tidak terlepas dengan pemikiran Sartre mengenai eksistensi manusia
sebagai \emph{being-for-itself}. Kematian tidak dapat diintegrasikan ke
dalam perencanaan eksistensi manusia. Ia bukan dimensi konstitutif
eksistensi manusia. Manusia, dengan kebebasan eksistensial yang
dimilikinya, bukanlah pengada yang berjalan menuju kematian, apalagi
menantikan atau mengharapkan kematian.

Kematian adalah realitas yang datang dari luar, yang secara radikal
mematahkan eksistensi manusia yang terarah kepada dan dalam kebebasan.
Walaupun demikian, kematian toh tidak terelakkan. Semua manusia pun
dalam kondisi yang sama, yaitu terkutuk untuk mati. Ia adalah faktor
eksternal yang menciptakan absurditas dalam kehidupan manusia sebagai
\emph{being-for-itself} (makhluk sadar dan bebas, yang juga berarti
tidak sempurna). Meskipun demikian, Sartre menolak bunuh diri sebagai
solusi atas absurditas ini. Bunuh diri adalah juga absurditas. Lebih
baik hidup di masa sekarang dengan membuat berbagai pengalaman, sejauh
itu dimungkinkan oleh kebebasan.

Dalam pemikiran Sartre, kematian adalah sesuatu yang datang dari luar
dan memotong eksistensi manusia. Kematian menjadikan kebebasan manusia
menjadi tak bermakna (mubazir) sehingga hidup manusia hancur dan tak ada
artinya. Kematian adalah fakta yang datang dari luar, tidak melekat
secara inheren dalam diri manusia. Kematian adalah sesuatu yang
menjadikan hidup manusia di dunia tidak mungkin lagi. Kematian membuat
hidup manusia sia-sia. Kematian membuat manusia tidak otentik, memotong
manusia sebagai subjek yang bebas. Kebebasan manusia adalah tak
terbatas, namun jika mati, hidup menjadi konyol (absurd).

\hypertarget{pandangan-albert-camus}{%
\subsection{Pandangan Albert Camus}\label{pandangan-albert-camus}}

Sebagaimana Sartre, Camus melihat kematian juga sebagai faktor yang
menciptakan absurditas dalam kehidupan manusia. Akan tetapi, ia
mengemukakannya dengan alasan yang berbeda. Camus memikirkan mengenai
relevansi eksistensi tentang absurditas dan konsekuensi dari absurditas
tersebut. Kehidupan itu absurd dan manusia harus menghadapinya secara
emosional dan intelektual.

Absurditas hidup manusia paling terlihat saat kita mengkontemplasikan
berlalunya waktu. Di satu sisi kita hidup untuk masa depan (membuat
rencana-rencana, menatap masa depan di mana rencana-rencana itu akan
terwujud), di sisi lain berlalunya waktu juga berarti berlalunya
kehidupan kita, dan akhir dari masa depan. Maka, apa yang kita rindukan
persis menjadi apa yang kita tolak. Itulah juga fakta atau realitas
kematian. Ia membuat kita merasa bahwa tak satu pun dari keinginan kita
dan usaha kita mempunyai makna.

Mungkin untuk sementara waktu manusia dapat menyingkirkan kesadaran akan
kematian, yaitu dengan menenggelamkan diri dalam anonimitas kehidupan
modern. Akan tetapi, pada saatnya, kondisi sebenarnya dari eksistensi
akan muncul dengan kejamnya. Kematian akan nampak sebagai alienasi
fundamental dari eksistensi manusia. Kalau demikian, apa yang dapat
dilakukan? Jawaban Camus adalah: berusahalah sedapat mungkin hidup tanpa
pengharapan, tetapi juga tanpa terjatuh ke dalam keputusasaan radikal.
Menurutnya, ada kebahagiaan, kegembiraan dan ketenangan dalam menghayati
kehidupan dengan kesadaran akan absurditas.

Konsekuensi dari penghayatan akan absurditas adalah sikap berontak
(\emph{revolt}) atau konfrontasi terus menerus antara diri manusia dan
kegelapannya. Itulah hal yang memberikan kebebasan dalam hidup, dan
akhirnya juga kebahagiaan. Albert Camus menolak bunuh diri sebagai
reaksi terhadap absurditas kehidupan. Menyelesaikan masalah
ketidakberartian hidup dengan mengarahkan hidup ke masa depan juga
merupakan solusi semu. Orang memang tidak dapat mengatasi
ketidakberartian, tetapi manusia dapat memperoleh kebahagiaan dalam
ketidakberartian, yaitu dengan mengkontemplasikan ketidakberartian
tersebut. Ketika kita sadar akan absurditas hidup, kita mencapai suatu
kemenangan atas absurditas.

Dalam pemikiran Camus, kematian membuat hidup menjadi absurd. Manusia
masuk ke dalam kehidupan yang rutin. Hidup juga menjadi absurd bila
manusia mengkontemplasikan berlalunya waktu: di satu sisi, kita
memikirkan diri kita sebagai ``yang akan menjadi sesuatu'' (dengan kata
lain hidup untuk masa depan), namun di sisi lain berlalunya waktu juga
berarti semakin dekat akhir kehidupan kita. Segala nilai menjadi absurd
karena akan berlalu. Kematian merupakan suatu fakta kehidupan, membuat
kita merasa bahwa keinginan dan usaha kita tidak memiliki makna.

Tetapi, Camus menolak tindakan bunuh diri---tindakan yang dilakukan
ketika seseorang kehilangan makna hidup. Baginya, tindakan tersebut
justru mencerminkan kebertundukan manusia pada absurditas kematian.
Alih-alih bunuh diri, manusia seharusnya memberontak (\emph{revolt})
dengan cara menjalani hidup dengan sehabis-habisnya. Sama seperti
Sisipus yang menjalani hukumannya, Sartre mengajak kita membayangkan
bahwa Sisipus menjalaninya dengan gembira (``\emph{one must imagine
Sisyphhus happy}''). Dengan begitu, kegembiraan Sisipus dapat dikatakan
bentuk pemberontakan terhadap hukuman yang dijatuhkan kepadanya, yaitu
kehidupan tanpa makna---sama seperti kita harus memberontak terhadap
hidup kita yang absurd karena kematian.

\hypertarget{nilai-nilai-edukatif-kematian}{%
\subsection{Nilai-nilai Edukatif
Kematian}\label{nilai-nilai-edukatif-kematian}}

Para filsuf di atas memiliki pemaparan berbeda tentang kematian, tetapi
refleksi mereka sama-sama mengantarkan pada ``hidup dalam kesadaran akan
kematian.'' Kita dapat belajar dari kematian bukan pada kematian itu
sendiri. Tidak ada yang dapat `dipelajari' atau `diketahui' dari
kematian karena kita tidak pernah dapat tahu apa-apa tentang kematian.
Pelajaran dapat kita ambil dengan memperhatikan bagaimana manusia
menghadapi kematian. Di situlah kita dapat mengambil nilai-nilai
edukatif dari kematian. Berikut beberapa pelajaran yang kita dapat dari
kematian.

Kesadaran akan kematian mendorong manusia melakukan sesuatu untuk
menunda kematian yang tak terelakkan. Kesadaran akan kematian membuat
manusia terdorong untuk memperbaiki kondisi hidup, dan membuat kebaikan
lebih banyak. Dengan itu, manusia berkembang menciptakan kebudayaan dan
peradaban dunia yang lebih manusiawi, kebutuhan pokok manusia tercukupi,
serta ditegakkannya keadilan. Manusia berjuang melawan penyakit,
ketidakadilan, dan berbagai bentuk alienasi. Manusia membuat
struktur-struktur yang menunda ancaman kematian dan yang memungkinkannya
hidup lebih manusiawi di dunia ini. Terakhir, manusia terdorong untuk
memaknai hidupnya.

Kesadaran akan kematian juga mengharuskan kita untuk mempertanyakan
makna pekerjaan manusia di dunia. Barang-barang duniawi nilainya
terbatas. Makna fundamental eksistensi manusia tidak dapat merupakan
akumulasi dari kekayaan pribadi yang dipergunakan untuk kepentingan
pribadi. Pada saat kematian segala sesuatu akan ditinggalkan. Kematian
mengajarkan bahwa harta yang dipergunakan untuk kepentingan pribadi
semata-mata merupakan kesia-siaan. Sekaligus, memberikan terang bahwa
semua barang kebudayaan hanya bermakna jika dipergunakan untuk
meningkatkan martabat sesama. Seperti dikatakan Levinas, barang-barang
tidak memanifestasikan diri sebagai sesuatu yang harus ditumpuk, tetapi
sebagai sesuatu yang harus diberikan.

Kematian mendorong manusia meneruskan hidupnya dengan mempertahankan
kesinambungan keturunan, demi diwarisinya nilai-nilai kebudayaan yang
dianutnya. Anak atau keturunan adalah manusia yang lebih besar dari
karya material kebudayaan apapun. Kematian dengan begitu juga mengajak
manusia untuk meneruskan kehidupannya sendiri dan memberikan cinta kasih
kepada sang anak. Pada anak harus dinyalakan kepribadian dan cinta
kasih, melalui kata-kata yang diucapkan dan melalui cinta kasih yang
diberikan.

Kematian mengajarkan tentang keterbatasan manusia. Manusia sadar bahwa
kehidupan adalah pemberian dari ``Yang Lain.'' Manusia dapat
merealisasikan eksistensinya, namun juga harus sadar bahwa eksistensinya
bersandar pada pemberian ``Yang Lain'' tersebut. Kematian membuat Anda
sadar bahwa manusia terbatas dan bukan dasar dari keseluruhan hidupnya,
karena tidak menguasai seluruh kehidupannya. Pengakuan keterbatasan
manusia adalah tanda bahwa ada sesuatu yang lain yang menjadi dasar
fundamental. Maka, ada implikasi psikologis untuk menerima
keterbatasannya, manusia harus memilih. Ada \emph{optio fundamentalis}
(pilihan dasar) saat menghadapi kematian, yaitu antara mengakui
kemandirianku di atas yang lain atau mengakui keterbatasan (tergantung
pada yang lain yang memberi hidup). Saat hidup, orang bisa mengatakan
bahwa ia mandiri, namun saat mendekati kematian, orang mau tak mau
mengakui bahwa ia tergantung pada yang lain.

Kematian menisbikan segala peran dan status sosial manusia. Kematian
mengajarkan kesamaan absolut semua manusia, karena semua akan mengalami
pengalaman maut yang sama. Semua akan kembali menjadi debu. Di hadapan
kematian, semua manusia sama-sama miskin. Di sini kematian mengajarkan
bahwa peran sosial adalah pelayanan untuk meningkatkan martabat yang
lain dan untuk mengembangkan kebersamaan.

Kematian mengalahkan egoisme dan kesombongan. Kehendak untuk berkuasa
dan keharusan akan dominasi. Kematian mengundang kita untuk bersikap
toleran dalam berhadapan dengan yang lain, memberi tempat pada yang
lain, karena tidak ada yang mutlak dalam komunitas manusia.

Kematian memberikan kepada manusia suatu makna totalitas. Ini bukan
berarti bahwa kematian merupakan bab terakhir sebuah buku yang telah
selesai. Kematian mematahkan dan mengancam, maka pada dirinya ia bukan
pemenuhan atau totalitas. Akan tetapi, kematian memberikan makna
totalitas karena: (1) sebagai horizon kesadaran manusia, ia memungkinkan
manusia melihat seluruh hidupnya secara keseluruhan, dan (2) sebagai
akhir dari segala kemungkinan, kematian menghambat kita untuk mengubah
makna hidup dan perjalanan hidup kita. Dengan kematian, segala
kemungkinan habis (tak lagi ada, kebebasan menjadi tidak berdaya untuk
mengubah orientasi atau realisasi eksistensi). Akhirnya, dengan
kematian, manusia bisa melihat hidupnya sebagai suatu totalitas; suatu
kalimat yang punya titik, punya akhir, dan dengan demikian punya makna.

\hypertarget{bahan-bahan-rujukan}{%
\subsection{Bahan-bahan Rujukan}\label{bahan-bahan-rujukan}}

\begin{enumerate}
\def\labelenumi{\arabic{enumi}.}
\tightlist
\item
  Louis Leahy. \emph{Siapakah Manusia?: Sintesis Filosofis tentang
  Manusia}. Yogyakarta: Kanisius, 2001.
\item
  M. Sastrapratedja, SJ. \emph{Filsafat Manusia}. Jakarta: Pusat Kajian
  Filsafat dan Pancasila STF Driyarkara, 2010.
\item
  Tjaya, Thomas Hidya. \emph{Emmanuel Levinas: Enigma Wajah Orang Lain},
  Jakarta: KPG, 2018.
\end{enumerate}

\hypertarget{epistemologi}{%
\chapter{Epistemologi}\label{epistemologi}}

Epistemologi merupakan cabang ilmu filsafat yang secara khusus
menggeluti pertanyaan-pertanyaan yang bersifat menyeluruh dan mendasar
tentang pengetahuan. `Epistemologi' berasal dari kata \emph{episteme}
(pengetahuan) dan \emph{logos} (perkataan, pikiran, ilmu). Gejala
pengetahuan merupakan salah satu objek kajian yang terus menyibukkan
filsafat. Berikut uraian tema-tema penting dalam epistemologi, yaitu (1)
bagaimana mempertanggung jawabkan klaim kebenaran pengetahuan, atau yang
sering disebut sebagai masalah pembenaran (\emph{the problem of
justification}), dan (2) mengenai perbedaan antara Sosiologi Pengetahuan
dan Epistemologi Sosial.

\hypertarget{masalah-pembenaran-justifikasi}{%
\section{Masalah Pembenaran
(Justifikasi)}\label{masalah-pembenaran-justifikasi}}

\begin{quote}
Salah satu pokok bahasan penting dalam Epistemologi adalah tentang
bagaimana mempertanggungjawabkan klaim kebenaran pengetahuan atau
disebut juga masalah pembenaran (\emph{problem of justification}).
Secara umum, dapat dibedakan adanya empat teori pembenaran
(\emph{theories of justification}), yaitu: Fondasionalisme,
Koherentisme, Internalisme dan Eksternalisme.
\end{quote}

Pembenaran (\emph{justification}) yang dimaksudkan di sini bukan
``rasionalisasi'' (yang bersifat peyoratif), tetapi merupakan upaya
pertanggung-jawaban rasional atas klaim kebenaran dari kepercayaan atau
pendapat yang dipegang. Upaya ini mengantarkan kita pada empat teori
pembenaran (\emph{theories of justification}).

\hypertarget{fondasionalisme}{%
\subsection{Fondasionalisme}\label{fondasionalisme}}

Fondasionalisme adalah teori pembenaran yang menyatakan bahwa suatu
klaim kebenaran pengetahuan---agar dapat dipertanggung-jawabkan secara
rasional---perlu didasarkan atas suatu fondasi atau basis yang kokoh,
yang jelas dengan sendirinya, yang tak dapat diragukan kebenarannya, dan
yang tidak memerlukan koreksi lebih lanjut. Para penganutnya meyakini
bahwa manusia dapat memperoleh pengetahuan langsung dan tak diragukan
tentang prinsip-prinsip pertama atau proposisi dasar yang jelas pada
dirinya. Prinsip-prinsip pertama itu langsung dapat dimengerti dan
mencukupi untuk dijadikan dasar bagi bangunan pengetahuan. Menurut
mereka, pembenaran pengetahuan itu bersifat hierarkis, dan digambarkan
sebagai bangunan gedung bertingkat yang selalu menuntut adanya suatu
fondasi yang kokoh untuknya.

Ada dua versi Fondasionalisme, yaitu versi ketat dan versi longgar atau
moderat. Versi ketat menuntut suatu fondasi pembenaran pengetahuan yang
tidak dapat keliru, tidak dapat diragukan dan tidak dapat dikoreksi, di
mana hubungan antara kepercayaan dasar dan kepercayaan lain berdasarkan
suatu implikasi logis atau induksi dari kepercayaan dasar tersebut.
Mereka yang termasuk penganut fondasionalisme versi ketat ini adalah:
(a) Epistemolog Rasionalis: Descartes, Leibniz dan Spinoza. (b)
Epistemolog Empiris: Locke, Berkeley, dan Hume. (c) Epistemolog Logis:
Russell, Ayer dan Carnap.

Persoalannya, tuntutan versi ketat ini dalam prakteknya sulit atau
bahkan mustahil dapat terpenuhi. Oleh karena itu, muncul fondasionalisme
versi longgar yang mengatakan bahwa suatu kepercayaan dapat disebut
kepercayaan dasar dan menjadi fondasi pembenaran pengetahuan, kalau
secara intrinsik probabilitas atau kementakan kebenarannya tinggi. Jadi,
tidak menuntut bahwa suatu kepercayaan dasar haruslah tak dapat keliru,
tak dapat diragukan, dan tak memerlukan koreksi kembali, juga bahwa
hubungan antara kepercayaan dasar dan kepercayaan lain tidak perlu dalam
bentuk implikasi logis atau induksi penuh.

Pokok yang hendak dikemukakan oleh penganut teori ini adalah harus ada
suatu kepercayaan dasar yang dapat dijadikan sebagai fondasi pembenaran,
karena tanpa itu akan terjadi penarikan argumen mundur terus-menerus
(\emph{Infinite regression}, \emph{ad infinitum}). Harus ada titik
berhenti pada kepercayaan dasar, yaitu kepercayaan yang kebenarannya
sudah jelas dengan sendirinya. Teori pembenaran ini erat kaitannya
dengan teori kebenaran korespondensi dan pandangan tentang kenyataan
yang disebut realisme.

Ada beberapa persoalan dalam teori pembenaran fondasionalisme:
\emph{Pertama}, untuk fondasionalisme ketat, tuntutan tak dapat keliru,
jelas dengan sendirinya, tak dapat diragukan dengan hubungan dalam
bentuk implikasi atau induksi penuh hampir mustahil. Kalau tuntutannya
demikian, pengetahuan yang benar menjadi hampir tidak ada. Orang menjadi
skeptis dengan kenyataan dunia luar, persepsi, kepercayaan berdasarkan
ingatan, induksi, dan usaha memperoleh kebenaran lainnya. Sementara itu,
fondasionalisme versi moderat kurang memberi dasar yang kuat bagi klaim
kebenaran pengetahuan yang kita perlukan. Terlebih lagi, karena cukup
banyak memberikan konsensi terhadap kebenaran koherentisme, versi ini
menjadi tidak berbeda dengan koherentisme moderat.

\emph{Kedua}, pembedaan bahkan pemisahan secara tegas antara kepercayaan
dasar dan kepercayaan simpulan bukanlah tanpa kesulitan. Apa yang
diklaim sebagai kepercayaan dasar kebenarannya tidak selalu sudah jelas
dengan sendirinya pada setiap orang, baik itu berkaitan dengan tingkat
probabilitasnya maupun bahwa masyarakat umum telah menganggapnya sebagai
kebenaran. Dalam kenyataan, apa yang diklaim sebagai kepercayaan dasar
ternyata merupakan penyimpulan dari kepercayaan-kepercayaan yang lain.
Seperti kata Wufrid Sellars, apa yang diklaim sebagai sesuatu yang sudah
terberikan dalam pengalaman inderawi sebagai tak mungkin keliru adalah
mitos belaka.

\emph{Ketiga}, argumen penarikan mundur terus menerus yang dijadikan
alasan harus adanya fondasi bagi kepercayaan-kepercayaan yang lain
bukanlah argumen konklusif. Suatu kepercayaan tidak melulu bersifat
hierarkis sehingga harus ada fondasi supaya kepercayaan dapat diterima
kebenarannya. Dapat saja suatu kepercayaan terbentuk karena adanya
alasan yang saling mendukung antara kepercayaan yang satu dengan
kepercayaan yang lain. Inilah yang yang dikemukakan oleh Koherentisme.

\hypertarget{koherentisme}{%
\subsection{Koherentisme}\label{koherentisme}}

Menurut Koherentisme, semua kepercayaan mempunyai kedudukan epistemik
yang sama, sehingga tidak memerlukan pembedaan antara kepercayaan dasar
dan kepercayaan simpulan. Suatu kepercayaan sudah dapat
dipertanggungjawabkan kalau kepercayaan itu koheren atau konsisten
dengan seluruh sistem kepercayaan yang selama ini diterima kebenarannya.
Gambaran dasar dari teori pembenaran ini adalah sebuah sistem jaringan
yang terbuat dan memperoleh kekuatannya dari pelbagai kepercayaan yang
saling mendukung. Suatu sistem jaringan kepercayaan dapat dibenarkan,
jika komponen kepercayaan yang membentuknya koheren dan konsisten satu
sama lain sebagai suatu pengertian yang holistik. Teori pembenaran ini
erat kaitannya dengan teori kebenaran koheren atau teori keteguhan dan
pandangan tentang kenyataan yang disebut idealisme. Para filsuf yang
memegang teori pembenaran ini adalah Hegel, Bradley (modern); Wilfrid
Sellars, W. V. Quine dan Laurence Bonjour (kontemporer).

Koherentisme juga dapat dibedakan menjadi koherentisme ketat dan
moderat. Sebagai syarat agar suatu sistem jaringan kepercayaan dianggap
koheren, koherentisme ketat menuntut komponen kepercayan yang membentuk
sistem jaringan kepercayaan harus konsisten satu sama lain, dan secara
logis saling mengimplikasikan. Sedangkan koherentisme moderat menuntut
konsistensi dari komponen-komponen yang membentuk sistem jaringan
kepercayaan, tetapi tidak perlu secara logis harus mengimplikasikan.

Koherentisme juga dibedakan antara yang bersifat linear (yang membentuk
suatu lingkaran pembenaran dan yang tampaknya tidak dapat menghindari
kesulitan penarikan mundur terus menerus), dan yang bersifat holistik
(di mana kepercayaan---yang dipersoalkan dasar
pertanggungjawabannya---ditempatkan dalam keseluruhan sistem kepercayaan
yang berlaku). Dalam koherentisme holistik, pembenaran penarikan
kesimpulan yang menjadi kepercayaan tidak dibenarkan berdasarkan suatu
kepercayaan dasar yang sudah pasti tidak dapat keliru, tetapi dibenarkan
berdasarkan koherensinya dengan kesimpulan-kesimpulan lain yang ditarik
dari pelbagai kejadian yang ada.

Setidaknya ada empat keberatan terhadap teori pembenaran koherentisme:
\emph{Pertama}, argumen isolasi atau alasan untuk berkeberatan terhadap
koherentisme karena teori pembenaran ini seperti mengisolasi diri dari
kenyataan dunia luar. Pembenaran dalam teori ini pada akhirnya
menyangkut perkara relasi antar-proposisi-proposisi dalam sistem yang
dipercayai dan tidak ada kaitannya dengan bagaimana proposisi-proposisi
tersebut sesungguhnya me.ujuk kenyataan dunia luar atau tidak. Argumen
isolasi kadang juga disebut kebneratan berdasarkan kemungkinan adanya
sistem tandingan yang sama-sama kolieren secara internal

\emph{Kedua}, keberatan karena tidak adanya masukan dari dunia luar.
Koherentisme melulu terdiri dari serangkaian kepercayaan yang secara
internal saling berhubungan dan saling mendukung tetapi tidak mempunyai
tolok ukur untuk nienilai seberapa jauh kepercayaan itu merujuk kepada
kenyataan sesungguhnya di dunia luar.

\emph{Ketiga}, keberatan berdasarkan alasan penarikan mundur terus
menerus tanpa batas. Menurut koherentisme, setiap kepercayaan tertentu
mesti memperoleh pembenaran empirisnya dari koherensinya dengan
keseluruhan sistem tersebut. Ini membuatnya tetap jatuh ke dalam
kelemahan penarikan mundur terus menerus tanpa batas, mengandalkan apa
yang masih perlu dibuktikan kebenarannya.

\emph{Keempat}, keberatan berkenaan dengan kenyataan bahwa bukan hanya
koherensi tidak mencukupi. tetapi bahkan dalam situasi tertentu, suatu
kepercayaan dapat dibenarkan tanpa harus koheren dengan
kepercaaan-kepercayaan lain sebelumnya yang sudah dianggap benar,
Misalnya, setiap revolusi dalam perkembangan pengetahuan selalu memuat
unsur penerimaan sebagai benar apa yang sesungguhnya tidak koheren
dengan sistema kepercayaan salama ini yang diterima sebagai benar.
Keberatan keempat ini mempertanyakan pula keniscayaan perlunya
konsistensi sebagai dasar pembenaran. Keberatan ini sering disebut
paradoks lotere: ``orang pertama yang membeli loetre kemungkinan akan
kalah,'' tetapi ``orang pertama yang membeli lotre kemungkinan juga akan
menang.'' Bukanlah kedua pernyataan tersebut tidak konsisten, tetapi
sama-sama dibenarkan?

\hypertarget{internalisme}{%
\subsection{Internalisme}\label{internalisme}}

Internalisme adalah pandangan bahwa orang selalu dapat menentukan apakah
kepercayaan atau pendapat yang ia miliki dapat dipertanggungjawabkan
kebenarannya atau tidak dengan melakukan introspeksi diri. Para penganut
internalisme percaya bahwa manusia, sebagai makhluk rasional, secara
\emph{prima facie} mempunyai kewajiban untuk mempertanggungjawabkan
secara rasional apa yang ia percayai. Selanjutnya, karena keharusan
untuk memberi pertanggungjawaban rasional diterima kemungkinannya, maka
manusia pun memiliki akses ke persyaratan yang menentukan apakah suatu
kepercayaan yang la pegang dapat dibenarkan atau tidak.

Internalisme garis keras, seperti pemikiran Platon, yakini bahwa pikiran
manusia yang terlatih baik dapat memiliki akses kognitif introspektif
yang tidak dapat keliru, misalnya dalam hal intuisi akal budi akan
\emph{Idea}. Demikian juga Descartes meyakini bahwa manusia mempunyai
gagasan yang begitu jelas dan terpilah-pilah sehingga tidak mungkin
dapat diragukan lagi kebenarannya. Sementara itu, internalisme garis
lunak tidak memandang bahwa manusia secara introspektif pasti memiliki
akses kognitif yang tidak dapat keliru. Pandangan ini hanya menuntut
alasan terbaik yang tersedia untuk suatu pembenaran. Jadi, tidak harus
didasari oleh kepercayaan yang harus pasti dan tidak dapat diragukan
lagi, tetapi cukuplah bahwa suatu alasan dapat dipertanggungjawabkan
secara rasional.

Ada beberapa persoalan yang dapat diajukan sebagai keberatan terhadap
teori Internalisme. Yang sering menjadi persoalan adalah klaimnya bahwa
manusia memiliki akses introspektif langsung terhadap sesuatu yang
menjamin kebenaran dari suatu kepercayaan atau pendapatnya, apalagi
kalau itu dipercaya sebagai tidak dapat keliru. Pertanyaannya,
sungguhkah apa yang kita yakini secara pribadi sebagai benar---meskipun
kita melihat dengan mata kepala sendiri atau dengan intuisi akal budi
yang begitu jelas dan terpilah-pilah---secara objektif selalu benar?
Kasus halusinasi, misalnya, dapat membut orang seperti betul-betul
melihat sesuatu atau seseorang yang sebenarnya tidak ada di hadapannya.

\hypertarget{eksternalisme}{%
\subsection{Eksternalisme}\label{eksternalisme}}

Berlawanan dengan internalisme yang menekankan syarat-syarat psikologis
internal dalam subjek sebagai syarat pembenaran pengetahuan, kaum
eksternalis lebih menekankan faktor-faktor eksternal, yaitu: (1) dapat
diandalkan atau tidaknya proses pemerolehan pengetahuan yang terjadi,
(2) berfungsi tidaknya secara normal dan semestinya sarana-sarana wajar
kita untuk mengetahui, dan juga (3) bagaimana lingkungan, sejarah, dan
konteks sosial mempengaruhi proses pemerolehan pengetahuan. Itu semua
menjadi faktor penentu dibenarkan atau tidaknya suatu kepercayaan atau
pendapat.

Salah satu bentuk Eksternalisme adalah Reliabilisme, sebagaimana dianut
Alvin Goldman. Menurutnya, kepercayaan seseorang bahwa sesuatu adalah P
dibenarkan bila kepercayaan itu dihasilkan oleh suatu proses pengetahui
yang dapat diandalkan (\emph{Reliable Cognitive Process}), misalnya oleh
penglihatan sendiri dengan daya penglihatan yang normal. Namun,
realibilisme ini dikritik oleh Alfin Platinga menganggap bahwa paham ini
tidak memadai karena dapat saja suatu proses kognitif sama-sama wajar
dan dapat diandalkan, tetapi tetap membawa pada hasil pengetahuan yang
berbeda. Sehingga, kewajaran proses kognitif saja tidak cukup menjadi
pembenaran pengetahuan. Lagipula, bagaimana kita dapat menentukan proses
mana yang wajar, dan apa saja ukurannya?

Karena masalah-masalah itu, Platinga memberikan rumusan yang lebih
rinci, yaitu bahwa pembenaran pengetahuan harus berdasarkan: (i)
daya-daya kognitif yang berfungsi semestinya sesuai dengan (ii) desain
atau rancang bangun daya kognitif tersebut dalam lingkungan yang sesuai,
di mana (iii) desain tersebut adalah desain yang baik yang mengarahkan
pada kebenaran. Konsep desain Platinga yang mendasari berfungsi dengan
semestinya daya-daya kognitif dalam menciptakan organ-organ tubuh
manusia dapat diberi tafsiran teistik (rencana ilahi dalam penciptaan),
dapat pula dimengerti secara alami. Akan tetapi, Platinga sendiri
berpendapat bahwa naturalisme metafisis yang menjelaskan perkembangan
kesadaran dan pemikiran manusia hanya secara alami berdasarkan proses
evolusi yang aksidental akan tetap gagal menjelaskan soal fungsi
semestinya dari daya-daya kognitif manusia. Baginya, epistemologi
naturalis seperti itu mengandaikan antropologi supernaturalistik.

Eksternalisme yang dikembangkan oleh Platinga di atas pun tidak lepas
dari kritikan: \emph{Pertama}, adanya desain tidak menjamin bahwa
daya-daya kognitif kita akan selalu berfungsi semestinya. Misalnya,
penglihatan kita sering mengecoh, penciuman kita tidak setajam penciuman
anjing, pendengaran kita tidak mampu menangkap frekuensi suara yang
terlalu tinggi maupun terlalu rendah. Implikasi dari kenyataan bahwa
proses kognitif kita tidak sempurna dapat ditafsirkan sebagai argumen
untuk melawan kesempurnaan atau kemahakuasaan Dia yang mendesain
keberadaan kita. Lagi pula, teori evolusi tampaknya lebih mudah
diterima, yaitu bahwa daya-daya kognitif ini adalah soal penyesuaian dan
penyelaran diri non-teleologis demi tetap dapat bertahan hidup, daripada
langsung menyimpulkan tentang desain Sang Pencipta.

\emph{Kedua}, rancang bangun atau desain bukan merupakan syarat mutlak
untuk jaminan fungsi yang semestinya. Kalau A dan B sama-sama melihat C,
akan sangat aneh dan kontraintuitif kalau lalu dinyatakan bahwa hanya A
yang sungguh atau dibenarkan melihat C, hanya karena A memang didesain
untuk berfungsi demikian sementara B tidak.

\emph{Ketiga}, Platinga menolak pandangan internalisme yang menganut
deontologisme epistemologis, yaitu paham bahwa kitu memiliki kewajiban
epistemologis untuk memperjanggungjawabkan secara rasional klaim
kebenaran pengetahuan kita untuk dapat dibenarkan. Padahal, unsur
kewajiban ini---yang pertama-tama berarti suatu pembenaran
subjektif---sesungguhnya amat relevan bagi diperolehnya pembenaran
objektif atas pengetahuan.

\hypertarget{pokok-pokok-lain-1}{%
\subsection{Pokok-pokok Lain}\label{pokok-pokok-lain-1}}

\hypertarget{apa-yang-dimaksud-dengan-pembenaran}{%
\subsubsection{Apa yang dimaksud dengan
``pembenaran?''}\label{apa-yang-dimaksud-dengan-pembenaran}}

Istilah `pembenaran' dalam bahasa Indonesia dapat mempunyai dua arti.
Arti pertama, yang bersifat peyoratif, adalah menganggap benar apa yang
sebenarnya salah. Dalam hal ini, `pembenaran' sama dengan rasionalisasi.
Arti kedua adalah melakukan pertanggungjawaban rasional atas klaim
kebenaran kepercayaan atau pendapat yang dipegang. Yang digunakan dalam
diskursus ini adalah pembenaran dalam arti yang kedua.

\hypertarget{apa-yang-dimaksud-dengan-fondasionalisme}{%
\subsubsection{Apa yang dimaksud dengan
Fondasionalisme?}\label{apa-yang-dimaksud-dengan-fondasionalisme}}

Fondasionalisme pada dasarnya menyatakan bahwa pertanggungjawaban
rasional atas suatu klaim kebenaran harus dilakukan di atas suatu
fondasi yang: (1) kokoh yang jelas dengan sendirinya, (2) tidak dapat
diragukan kebenarannya, dan (3) tidak memerlukan koreksi lebih lanjut.

Menurut para fondasionalis, pembenaran pengetahuan memiliki struktur
hierarkis: beberapa kepercayaan dipahami sebagai sudah jelas dengan
sendirinya sehingga tidak memerlukan pembenaran lain---itu disebut
kepercayaan dasar; sedangkan kepercayaan-kepercayaan lain hanya bisa
dibenarkan berdasarkan kepercayaan dasar tersebut sebagai fondasinya.
Hubungan antara kepercayaan dasar dan kepercayaan simpulan bersifat
asimetris, karena kepercayaan dasar merupakan dasar pembenaran bagi
kepercayaan simpulan. Selain itu, kaum fondasionalisme---terutama
fondasionalisme empiris---juga meyakini kepastian kebenaran dari apa
yang secara perseptual terberi dalam pengalaman langsung. Alasan pokok
bagi para fondasionalis mengapa harus ada suatu kepercayaan dasar adalah
untuk menghindari \emph{regressio ad absurdum} (penarikan kesimpulan tak
terhingga). Seandainya itu terjadi, kepercayaan dasar tidak dapat
dipertanggungjawabkan kebenarannya, yang lalu berakibat juga pada
kepercayaan-kepercayaan simpulan lain yang diturunkan darinya.

Dari sifatnya Fondasionalisme dapat dibagi dua:

\begin{enumerate}
\def\labelenumi{\arabic{enumi}.}
\item
  \textbf{Fondasionalisme Ketat}: Kepercayaan dasar haruslah (1) tak
  dapat keliru, yaitu apabila ketika dipegang, dan diikuti, kepercayaan
  tersebut secara niscaya tidak akan membawa pada kekeliruan; (2) tak
  dapat diragukan, yaitu apabila dipandang mustahil untuk meragukan
  kebenarannya; serta (3) tak dapat atau tidak perlu dikoreksi,
  kepercayaan dasar secara niscaya tidak dapat ditemukan alasan yang
  masuk akal untuk menganggapnya salah ataupun ada pembetulannya.
  Contoh: Descartes, Leibniz, Hume, Locke, dan Carnap.
\item
  \textbf{Fondasionelisme Moderat}: Menganggap tuntutan fondastonalisme
  ketat pada kenyataannya mustahil dipenuhi. Apa yang dalam sejarah
  filsafat pernah diklaim sebagai kepercayaan yang kebenarannya begitu
  jelas dengan sendirinya, ternyata dalam perjalanan waktu menjadi
  terbukti sebaliknya. Oleh karena itu, syarat bagi kepercayaan dasar
  menurut fondasionalisme moderat cukuplah bahwa kepercayaan tersebut
  secara intrinsik memiliki probabilitas kebenaran yang tinggi. Relasi
  antara kepercayaan dasar dan kepercayaan turunan juga tidak perlu
  dalam bentuk implikasi logis atau induksi penuh. Cukuplah kalau
  penyimpulan dalam bentuk penjelasan terbaik dapat diberikan
  berdasarkan kepercayaan dasar.
\end{enumerate}

Kekuatan Fondasionalisme adalah menekankan penjangkaran empiris pada
pengalaman sebagai sumber pengetahuan kita tentang dunia ini dan
pembenarannya. Ini secara intuitif terasa paling dekat dengan pengertian
umum mengenai pertanggung-jawaban kalim kebenaran. Kalau disuruh
mempertanggungjawabkan, orang pasti langsung mencari alasan yang masuk
akal untuk mendasari klaimnya.

Kelemahan Fondasionalisme adalah hampir mustahil untuk dapat terpenuhi
secara ketat. Dalam kenyataan, sulit memisahkan secara tegas antara
kepercayaan dasar dan kepercayaan simpulan. \emph{Regressio ad absurdum}
tidak secara niscaya harus dihindari. Bisa saja ada alasan yang saling
mendukung antara kepercayaan yang satu dengan yang lain tanpa harus
jatuh ke lingkaran setan.

\hypertarget{apa-yang-dimaksud-dengan-koherentisme}{%
\subsubsection{Apa yang dimaksud dengan
Koherentisme?}\label{apa-yang-dimaksud-dengan-koherentisme}}

Berbeda dengan Fondasionalisme, Koherentisme memandang semua kepercayaan
itu setara, sehingga tidak perlu ada pembedaan antara kepercayaan dasar
dengan kepercayaan simpulan. Suatu kepercayaan itu dapat
dipertanggungjawabkan kebenarannya kalau ia koheren atau konsisten
dengan keseluruhan sistem kepercayaan yang selama ini diterima
kebenarannya. Kalau fondasionalisme dapat dianalogikan dengan sebuah
rumah bertingkat, koherentisme analog dengan sebuah jaring di mana
setiap kepercayaan saling mendukung satu sama lain. Yang menjadi sasaran
pengujian atau pertanggungjawaban rasional bukanlah kepercayaan
masing-masing secara individual, tetapi seluruh sistem jaringan
kepercayaan. Contoh: Hegel, Bradley, Wilfird Sellars,dan W.V. Quine.

Koherentisme juga dapat dibagi dua menurut sifatnya:

\begin{enumerate}
\def\labelenumi{\arabic{enumi}.}
\tightlist
\item
  Koherentisme Ketat (Hegel, Bradley) menuntut kebenaran itu konsisten
  dan secara logis saling mengimplikasikan.
\item
  Koherentisme Lunak (Bonjour) menuntut kebenaran konsisten, namun tidak
  perlu secara logis saling mengimplikasikan.
\end{enumerate}

Koherentisme secara umum dapat diilustrasikan dengan usaha seorang
detektif untuk memecahkan suatu kasus. Dari berbagai kejadian dan
petunjuk kunci yang berbeda-beda, sang detektif berusaha untuk memberi
penjelasan yang koheren tentang kasus yang terjadi. Berikut beberapa
kritik terhadap koherentisme:

\begin{enumerate}
\def\labelenumi{\arabic{enumi}.}
\tightlist
\item
  Yang pertama adalah \emph{adanya pengisolasian diri dari kenyataan
  dunia}. Teori pembenaran ini hanya berkutat pada relasi antar
  proposisi dalam sistem yang dipercayai tanpa terlalu mempedulikan
  apakah proposisi-prosisi tersebut sungguh-sungguh merujuk pada
  kenyataan dunia luar atau tidak. Selain itu, teori pembenaran ini
  tidak dapat menjelaskan fenomena adanya beberapa sistem tandingan yang
  masing-masing sama-sama koheren secara internal, tetapi tidak
  kompatibel satu sama lain. Contoh: dukun vs dokter.
\item
  Kedua, \emph{tidak adanya masukan dari dunia luar}. Teori pembenaran
  ini tidak mempunyai tolok ukur untuk menilai seberapa jauh kepercayaan
  itu merujuk ke kenyataan sesungguhnya di dunia luar.
\item
  Ketiga, \emph{risiko terjebak penarikan mundur terus-menerus tanpa
  batas}. Prosedur pertanggungjawaban rasional yang diajukan
  koherentisme cenderung untuk selalu mengandalkan apa yang masih perlu
  dibuktikan kebenarannya.
\item
  Keempat, \emph{yang benar ternyata tidak harus selalu koheren}.
  Contoh: revolusi sains.
\end{enumerate}

\hypertarget{apa-yang-dimaksud-dengan-internalisme}{%
\subsubsection{Apa yang dimaksud dengan
Internalisme?}\label{apa-yang-dimaksud-dengan-internalisme}}

Kedua teori pembenaran yang telah disebutkan sebelumnya pada dasarnya
menganut internalisme. Internalisme mengandaikan orang selalu dapat
menentukan dengan introspeksi, apakah kepercayaan atau pendapatnya dapat
dipertanggungjawabkan kebenarannya secara rasional atau tidak. Pandangan
ini memandang manusia sebagai makhluk rasional yang secara prima facie
mempunyai kewajiban untuk mempertanggungjawabkan secara rasional apa
yang ia percayai atau apa yang menjadi pendapatnya. Manusia melalui
rasionya memiliki akses kognitif ke bukti atau evidensi bahwa
kepercayaan atau pendapatnya memang benar. Dengan introspeksi, manusia
bukan hanya dapat mengetahui apa yang ia percayai, melainkan juga
mengapa ia mempercayainya.

\begin{enumerate}
\def\labelenumi{\arabic{enumi}.}
\tightlist
\item
  \textbf{Internalisme ketat:} Pikiran manusia yang terlatih dengan baik
  dapat memiliki akses kognitif introspektif yang tak dapat keliru.
  Contoh: Plato, Descartes.
\item
  \textbf{Internalisme moderat:} Tidak mungkin manusia memiliki akses
  kognitif yang dapat salah. Oleh karena itu, supaya dapat
  dipertanggungjawabkan secara rasional, suatu kepercayaan cukuplah
  memiliki alasan yang masuk akal bagi orang yang memiliki kondisi
  psikologis yang sehat.
\end{enumerate}

Kaum internalis dewasa ini hanya menuntut supaya orang dapat memberi
alasan terbaik yang tersedia baginya, entah dari perspektif
fondasionalisme maupun koherentisme. Internalisme menuntut adanya
tanggung jawab dari subjek penahu untuk sungguh-sungguh berupaya mencari
kebenaran. Tanggung jawab epistemik ini setara dengan tanggung jawab
moral: sebagaimana dalam moral manusia secara prima facie kewajiban
untuk memaksimalkan perbuatan baik dan meminimalkan tindakan jahat,
demikian pula secara epistemis manusia wajib memaksimalkan jumlah
kepercayaan yang benar dan meminimalkan kepercayaan yang salah. Walaupun
tetap dapat keliru, secara epistemik tetap dapat dibenarkan kalau
setelah berusaha sekuat tenaga dan kemampuan ternyata pada akhirnya
kepercayaan dan pengetahuan yang kita peroleh dan kita pegang secara
objektif keliru. Tetapi, masih perlu dikritisi bahwa sungguhkah kita
mempunyai akses kognitif introspektif langsung seperti itu?

\hypertarget{apa-yang-dimaksud-dengan-eksternalisme}{%
\subsubsection{Apa yang dimaksud dengan
Eksternalisme?}\label{apa-yang-dimaksud-dengan-eksternalisme}}

Berlawanan dengan internalisme yang menkankan syarat-syarat psikologis
internal dalam subjek penahu sebagai syarat pembenaran pengetahuan,
teori pembenaran eksternalisme lebih menekankan proses penyebaban dari
faktor faktor eksternal, seperti: dapat diandalkan tidaknya proses
pemerolehan pengetahuan yang terjadi, berfungsi tidaknya secara normal
dan semestinya sarana-sarana wajar kita untuk mengetahui, bagaimana
konteks sosial-historis mempengaruhi proses pemerolehan pengetahuan

Alvin Goldman memperkenalkan reliabilisme yang mensyaratkan adanya
proses mengetahui yang dapat diandalkan (\emph{reliable cognitive
process}). Alvin Plantinga menolak reliabilisme karena paham ini ia
anggap tidak mencukupi. Sebab, menurutnya, proses kognitif yang wajar
dan dapat diandalkan dapat sekaligus membawa ke pengetahuan yang benar
maupun yang salah. Lagipula, reliabilisme tidak memberi acuan untuk
menentukan mana proses kognitif yang wajar dan yang tidak Bagi
Plantinga, epistemologi naturalistik memerlukan antropologi adikodrati.
Oleh karena itu, Plantinga memberi dua syarat: (1) adanya rancang bangun
dari daya daya kognitif dan (2) berfungsinya daya-daya tersebut sesuai
dengan rancang bangunnya. Dimasukkannya unsur `rancan bangun'
mengandalkan adanya suatu desain metafisis atau adikodrati atas manusia.

Kritikan terhadap Eksternalisme adalah:

\begin{enumerate}
\def\labelenumi{\arabic{enumi}.}
\tightlist
\item
  Pertama, sungguhkah daya-daya koggnitif kita selalu berfungsi
  sebagaimana mestinya seperti yang dipikirkan Plantinga? Kalau
  daya-daya kognitif kita memang didesain untuk berfungsi semestinya,
  mengapa begitu banyak kelemahannya?
\item
  Kedua, apakah sebuah rancang desain merupakan suatu syarat mutlak
  untuk adanya jaminan bahwa daya-daya kognitif akan berfungsi
  semestinya?
\item
  Ketiga, deontologisme epistemologis dalam internalisme yang ditolak
  oleh eksternalisme (Platinga) pada kenyataannya tidak dapat
  dihilangkan sama sekali.
\end{enumerate}

\hypertarget{sosiologi-pengetahuan-dan-epistemologi-sosial}{%
\section{Sosiologi Pengetahuan dan Epistemologi
Sosial}\label{sosiologi-pengetahuan-dan-epistemologi-sosial}}

\begin{quote}
Baik Sosiologi Pengetahuan maupun Epistemologi Sosial sama-sama mengkaji
dimensi sosial pengetahuan dan keduanya dapat saling memperkaya wawasan
kita tentang pengetahuan. Namun, keduanya perlu dibedakan satu sama
lain. Ada tiga pendekatan dalam Epistemologi Sosial: (1) peranan kondisi
sosial bagi pemerolehan pengetahuan, (2) pengaturan sosial kegiatan
memperoleh, menggunakan dan menyebarluaskan pengetahuan, dan (3) sifat
dasar pengetahuan kolektif.
\end{quote}

Epistemologi Sosial merupakan kajian \emph{konseptual dan normatif} atas
dimensi sosial pengetahuan, yaitu mengkaji dampak kondisi sosial
(hubungan, kepentingan, peran dan lembaga-lembaga sosial) terhadap
kajian konseptual dan normatif pengetahuan. Pertanyaan pokok yang
diajukan adalah (1) \emph{apakah kondisi sosial secara hakiki
mempengaruhi pemerolehan, pertanggungjawaban dan penyeberluasan
pengetahuan?}, dan (2) \emph{sejauh mana hal-hal tersebut berpengaruh?}
Hasil kajian Epistemologi Sosial memberi pemahaman bahwa ilmu
pengetahuan dalam kehidupan manusia bukan hanya produk kegiatan individu
dalam suatu kekosongan sosial. Kegiatan mengetahuai manusia selalu
berada di dalam konteks sosial, budaya dan sejarah tertentu. Kondisi
sosial pada gilirannya mempengaruhi pemerolehan, pertanggungjawaban dan
penyerbarluasan ilmu pengetahuan. Sehingga, ilmu pengetahuan juga dapat
dilihat sebagai suatu pengetahuan sosial atau pengetahuan publik yang
melibatkan upaya bersama.

Epistemologi Sosial memiliki kesamaan dengan Sosiologi Pengetahuan,
yaitu sama-sama mengkaji dimensi sosial dari pengetahuan, yaitu
bagaimana hubungan sosial, kepentingan sosial, dan lembaga sosial
berpengaruh pada penierolehan dan penyebarluasan pengetahuan. Lalu, di
mana letak perbedaannya? Martin Kusch mengatakan:

\begin{quote}
``\emph{Most social epistemologists recognise that social epistemology
is closely related to the sociology of knowledge. But different authors
conceive differently of this relation. Some suggest that the sociology
of knowledge is a purely descriptive and empirical enterprise, whereas
social epistemology is purely conceptual, and, at least in part, a
normative endeavour. Other social epistemologists see the two fields as
inseparable.}'' (MK, 1998).
\end{quote}

Dari situ, kurang lebih kita dapat melihat perbedaannya: (1) kalau
epistemologi sosial bersifat konseptual dan normatif, sosiologi
pengetahuan bersifat empiris-faktual-deskriptif; dan (2) Kalau sosiologi
pengetahuan mengkaji dimensi sosial pengetahuan dengan menggunakan
perangkat ilmu-ilmu sosial yang empiris sehingga bersifat sosiologis,
epistergologi sosial sifatnya filosofis.

\hypertarget{pokok-pokok-lain-2}{%
\subsection{Pokok-pokok Lain}\label{pokok-pokok-lain-2}}

\hypertarget{apa-yang-dimaksud-dengan-epistemologi-sosial}{%
\subsubsection{Apa yang dimaksud dengan Epistemologi
Sosial?}\label{apa-yang-dimaksud-dengan-epistemologi-sosial}}

Epistemologi Sosial adalah cabang epistemologi yang mempelajari (1)
karakter epistemis setiap individu yang muncul dari relasi mereka dengan
yang lain, dan (2) karakter pelstemis kelompok-kelaompok atau
sistem-sistem sosial (\emph{epistemic properties of individuals that
arise from their relations to others, as well as epistemic properties of
groups or social systems}). Contoh: perpindahan perngetahuan dari satu
orang ke orang lainnya.

Dimensi sosial dari Epistemologi Sosial adalah: (1) Berfokus pada
langkah-langkah atau jalan-jalan sosial menuju pengetahuan (\emph{It
focuses on social paths or routes to knowledge}); (2) Tidak memandang
subjek hanya sebagai orang per orang (\emph{It doesn't restrict itself
to believers taken singly}); (3) Memandang entitas kolektif maupun
bersama sebagai agen potensial pengetahuan (\emph{Instead of restricting
knowers to individuals, social epistemology may consider collective or
corporate entities as potential agents}).

Ada tiga cabang atau pendekatan Epistemologi Sosial: Pendekatan pertama
adalah dengan melihat \emph{peranan kondisi sosial bagi pemerolehan
pengetahuan tiap individu}. Pendekatan ini mempertanyakan dan meneliti
pengetahuan pada tataran individual, dan memeriksa apakah kondisi sosial
mempengaruhi kondisi-kondisi pengetahuan pada tataran individual.

Pendekatan kedua adalah dengan melihat \emph{pengaturan sosial dalam
tindakan mengetahui}. Cabang kedua ini mengkaji pengaruh sistem atau
organisasi sosial terhadap kerja kognitif, baik di antara individu
maupun kelompok Individu. Secara khusus, pendekatan ini menelaah
ditribusi pengetahuan yang optimal serta profil usaha dan tanggung jawab
kognitif dalam masyarakat. Pertanyaannya: ``Bagaimana seharusnya
tugas-tugas, tanggung jawab, dan previlese kognitif didistribusikan di
antara para penahu? Dengan cara apa pendistribusian ini bergantung pada
relasi sosial?''

Pendekatan ketiga adalah dengan melihat \emph{sifat dasar pengetahuan
kolektif}. Pendekatan ini mengkaji apakah sifat dasar suatu pengetahuan
selalu merupakan penjumlahan pengetahuan dari para anggota kelompok atau
lebih dari itu. Pertanyaannya: ``Apakah pengetahuan dimiliki oleh
kelompok individu, komunitas, atau kah institusi? Apakah pengethauan
kolektif tersebut hanyalah semata-mata penjumlahan pengetahuan tiap
anggotanya, ataukah sebenarnya lebih dari itu? Kalau ya, dalam hal apa
pengetahuan bergantung pada relasi sosial?''

\hypertarget{apa-yang-dimaksud-dengan-sosiologi-pengetahuan}{%
\subsubsection{Apa yang dimaksud dengan Sosiologi
Pengetahuan?}\label{apa-yang-dimaksud-dengan-sosiologi-pengetahuan}}

Sosiologi Pengetahuan berpendapat bahwa masyarakat turut mempengaruhi
struktur pengalaman manusia dalam pembentukan ide-ide, konsep-konsep,
dan sistem-sistem pemikiran. Lebih jauh lagi, mengingat bahwa kehidupan
sosial terkait erat dengan kehidupan kesadaran dan kemampuan reflektif
manusia, kita tidak dapat menunjuk manusia tanpa sekaligus
menunjuk-meminjam istilah Arthur Child, yaitu ``sifat sosial intrinsik
dari pikiran'' (\emph{the intrinsic sociality of mind}). Pengetahuan
dipandang sebagai sebuah kebudayaan dan banyak tipe pengetahuan
berkomunikasi serta melambangkan makna-makna sosial (seperti mengenai
kuasa dan kenikmatan, keindahan dan kematian, kebaikan dan bahaya).
Sebagai bentuk kebudayaan yang kuat, pengetahuan juga membentuk makna
dan menciptakan praksis dan objek sosial yang sepenuhnya baru. Sosiologi
pengetahuan memeriksa bagaimana objek perhatian publik muncul, yaitu
bagaimana masalah-masalah sosial dijelaskan dan fungsi-fungsi yang
dimainkan pengetahuan partikular dalam proses ini. Contoh Pemikir: Marx,
Durkheim, Mannheim, dan Mead.

\hypertarget{jika-dibandingkan-apa-kesamaan-dan-perbedaan-epistemologi-sosial-dan-sosiologi-pengetahuan-pada-titik-mana-mereka-saling-memperkaya}{%
\subsubsection{Jika dibandingkan, apa kesamaan dan perbedaan
Epistemologi Sosial dan Sosiologi Pengetahuan? Pada titik mana mereka
saling
memperkaya?}\label{jika-dibandingkan-apa-kesamaan-dan-perbedaan-epistemologi-sosial-dan-sosiologi-pengetahuan-pada-titik-mana-mereka-saling-memperkaya}}

Kesamaan epistemologi sosial dengan sosiologi pengetahuan adalah bahwa
keduanya sama-sama mengkaji pengaruh sosial dalam dan untuk pengetahuan.
Sementara itu, perbedaan keduanya adalah: a. Sosiologi Pengetahuan
kajiannya bersifat Deskriptif-Faktual, Epistemologi Sosial kajiannya
bersifat Normatif-Konseptual; b. Sosiologi Pengetahuan mempertanyakan
``Bagaimana lingkungan sosial mempengaruhi pengetahuan manusia?'',
Epistemologi Sosial mempertanyakan ``Apakah pengetahuan merupakan sifat
subjek penahu yang terisolir dari keadaan sosial (dan dalam arti apa
terisolir)? Atau, keadaan sosial juga merupakan unsur konstitutif dari
pengetahuan manusia?''; c.~Sosiologi Pengetahuan memberikan data-data
empiris-faktual bagi epistemologi sosial untuk direfleksikan.
Epistemologi Sosial memberikan landasan konseptual atas sosiologi
pengetahuan. Mengkritisi konsep-konsep yang sering diandaikan oleh
sosiologi pengetahuan, serta memberi kerangka normatif untuk
mengevaluasi sosiologi pengetahuan. Pada titik inilah Sosiologi
Pengetahuan dan Epistemologi Sosial saling memperkaya.

\hypertarget{bahan-bahan-rujukan-1}{%
\subsection{Bahan-bahan Rujukan}\label{bahan-bahan-rujukan-1}}

\begin{enumerate}
\def\labelenumi{\arabic{enumi}.}
\tightlist
\item
  J. Sudarminta. \emph{Epistemologi Dasar: Pengantar Filsafat
  Pengetahuan.} Yogyakarta: Kanisius, 2002.
\item
  Martin Kusch. ``Social Epistemology.'' \emph{University of Cambridge}.
\end{enumerate}

\hypertarget{filsafat-politik}{%
\chapter{Filsafat Politik}\label{filsafat-politik}}

Filsafat politik memberikan wawasan mendasar tentang kehidupan bersama
manusia secara politis. Di dalamnya dibahas persoalan-persoalan
filosofis politis yang mendasar seperti apa itu negara dan kekuasaan,
mengapa manusia bernegara, bagaimana konstitusi negara yang baik,
bagaimana hubungan antara keadilan, kesamaan dan kebebasan dalam kondisi
modernitas. Berikut ini pemaparan mengenai dua tema pokok dalam filsafat
politik, yaitu mengenai Hak-hak Asasi Manusia dan konsep kebebasan.

\hypertarget{hak-hak-asasi-manusia}{%
\section{Hak-hak Asasi Manusia}\label{hak-hak-asasi-manusia}}

\begin{quote}
Hak-hak asasi manusia mendahului Negara dan sistem hukum positif.
Asal-usul konsepnya dapat ditelusuri sampai pada tradisi
Yahudi-Kristiani, Filsafat Yunani Romawi dan Teori-teori Kontrak Sosial.
Lewat institusionalisasi dalam beberapa dokumen sejarah, hak-hak asasi
manusia akhirnya menjadi norma yang efektif untuk menata hubungan di
antara bangsa-bangsa beradab. Universalitas hak-hak asasi manusia pada
intensi dasariahnya adalah untuk melindungi manusia dari peristiwa
peristiwa negatif.
\end{quote}

Hak-hak asasi manusia (HAM) mendahului negara dan sistem hukum positif.
Ini karena hak-hak asasi manusia tidak diberikan oleh negara, tetapi
melekat pada manusia sebagai manusia. Maka, hak-hak ini juga mendahului
hukum positif. HAM melekat pada manusia semata-mata sebagai manusia,
bukan karena macam-macam hal seperti suku, agama, keanggotaan dalam
masyarakat, dsb. HAM itu pra-negara, pra-positif. HAM mendasari negara,
bukan sebaliknya negara mendasari HAM.

Asal-usul konsepnya dapat ditelusuri sampai pada tradisi
Yahudi-Kristiani, filsafat Yunani-Romawi dan teori-teori kontrak sosial.
Pada tradisi Yahudi Kristiani, benih-benih pengakuan terhadap HAM
terdapat dalam konsep manusia sebagai citra Allah (\emph{imago Dei},
Kej. 1:26-27). Paham ini mendasari suatu nilai intrinsik seorang
individu. Manusia sebagai citra Allah, setiap orang memilikinya,
termasuk juga para budak. Konsep citra Allah merupakan suatu nilai pada
dirinya sendiri, bahwa setiap manusia sama di hadapan Tuhan. Agama
Kristen memandang nilai diri manusia sebagai suatu tujuan yang terletak
di luar kekuasaan negara. Manusia tidak hanya ``tujuan diri bagi dirinya
(\emph{Selbstzweck für sich})'' saja, melainkan ``tujuan diri itu
sendiri (\emph{Selbstzweck schlechthin}).'' Nantinya, oleh Kant konsep
ini disekulerkan sebagai konsep martabat manusia.

Pada tradisi Yunani-Romawi, kita mengenal konsep Aristoteles mengenai
hukum kodrat. Pada filsafat Stoa, terdapat gagasan mengenai dignitas
melalui partisipasi semua manusia pada logos ilahi, yaitu kosmos
rohaniah dan religius yang di dalamnya kebebasan dan kesetaraan berkuasa
di antara manusia. Bahkan para budak pun berpartisipasi dalam kodrat
logos ilahi tersebut. Pada Abad Pertengahan, Thomas Aquinas menyatakan
mengenai adanya tatanan yang melampaul hukum positif. Setiap hukum
positif seharusnya sesuai dengar ``kodrat manusia yang melampaui negara,
yang tidak berubah sebagai bagian dari tatanan dunia ilahi
tersebut.''Kodrat manusia ini dikenali melalui asas asasnya yang sesuai
dengan ``kodrat''.

Sementara itu, dari teori-teori kontrak sosial, kita mendapati bahwa
konsep keadaan asali (\emph{state of nature}) mendasari pemikiran bahwa
hak hidup, hak milik, dan hak kebebasan itu pra-negara. Teori-teori
kontak sosial meneruskan ide tentang ruang bebas-negara bagi individu.
Kekuasaan negara yang legitimasinya diperoleh dari individu (Locke) atau
dari rakyat (Rousseau) tidak hanya merupakan ancaman bagi ``kodrat''
manusia, melainkan kekuasan itu juga memiliki kewajiban untuk
mengamankan hak-hak kodrati yang sudah tersedia di dalam keadaan asali.
Apabila kewajiban ini diabaikan, berakhirlah juga kepatuhan para warga
negara, sehingga manusia kembali ke keadaan pra kontrak Kesewenangan
dibatasi melalui respek terhadap manusia. Pada Kant, martabat manusia
ditandai dengan kemampuan untuk bertindak rasional dan bertanggungjawab.
Dari pemaparan ini kita melihat, baik tradisi Yahudi-Kristiani, filsafat
Yunant-Romawi maupun teori-teori kontrak sosial, semua memuat tendensi
dan menerima eksistensi hak-hak pra-positif, yaitu hak-hak asasi
manusia.

Lewat Institusionalisasi dalam berbagai beberapa dokumen sejarah,
hak-hak asasi manusia akhirnya menjadi norma yang efektif untuk menata
hubungan di antara bangsa-bangsa beradab. HAM adalah konsep yang sama
sekali modern, belum masuk ke dalam kosa kata politis sampai abad ke-17.
Padahal, ide martabat manusia yang terbentuk lama melalui sejarah
pemikiran di atas tidak akan memiliki dasar realpolitik kecuali
dijangkarkan di dalam konstitusi. Proses penjangkaran ke dalam
konstitusi atau institusionalisasi merupakan proses pemberlakuan (dan
pengakuan) HAM oleh suatu lembaga. Dalam sejarah, proses ini muncul di
negara-negara di mana keamanan politis Individu terancam oleh
absolutisme negara. Proses tersebut kemudian menghasilkan
dokumen-dokumen sejarah yang membicarakan mengenai ethos kebebasan
universal dan kesahihan universal dari HAM. Berikut beberapa dokumen
tersebut beserta pokok-pokoknya yang menonjol:

\begin{enumerate}
\def\labelenumi{\arabic{enumi}.}
\tightlist
\item
  \emph{Magna Charta Libertatum} (1215). Dokumen ini menggoncang
  ``cengkraman kekuasaan Norman'' di Inggris. Para penguasa di Inggris
  dituntut untuk memperhatikan hak-hak warganya.
\item
  \emph{Habeas Corpus Act} (1679). Dokumen ini memperjuangkan jaminan
  kebebasan personal dan menganggap penahanan semena-mena sebagai
  sesuatu yang mengingkari kebebasan.
\item
  \emph{Virginia Bill of Rights} (12 Juni 1776) dan \emph{American
  Declaration of Independence} (4 Juli 1776). Kedua dokumen ini
  mengandung ajaran John Locke mengenai hukum kodrat. Di dalamnya
  dinyatakan bahwa semua manusia pada kodratnya adalah bebas, otonom,
  dan memiliki hak-hak tertentu sejak lahir, antara lain hak atas hidup
  dan hak untuk mengejar kebahagiaan. Negara dan sistem hukum tidak
  dapat memaksa manusia untuk menyerahkan kebebasan itu. Bahkan, negara
  atau pemerintahan harus menjamin agar hak-hak tersebut terpenuhi.
\item
  \emph{Déclaration des droits de l'homme et du citoyen} (1789). Dokumen
  ini mengandung Teori Kontrak Sosial Jean-Jacques Rousseau dan ajaran
  Montesquieu mengenai pembagian kekuasaan (dokumen-dokumen Prancis).
\item
  Deklarasi Umum HAM (1948). Dokumen ini menjadikan HAM diakui secara
  global. Deklarasi ini penting karena merupakan tanggapan dan
  perlawanan atas pelecehan HAM di zaman modern, misalnya pemusnahan
  massal orang Yahudi oleh Nazi Jerman, Perang Dunia I dan II, dsb.
\end{enumerate}

Melalui dokumen-dokumen di atas, ide pra-HAM mengenai martabat manusia
dirumuskan ke dalam bahasa hukum. Misalnya, dalam \emph{Virginia Bill of
Rights} dituliskan: ``Semua manusia menurut kodratnya sama bebas dan
otonom dan memiliki hak-hak tertentu sejak kelahirannya. Mereka tidak
dapat memaksa anak cucu mereka lewat persetujuan sekalipun saat
berdirinya negara untuk menyerahkan hak-hak itu, yaitu hak atas
kehidupan, kebebasan, dan kemungkinan memperoleh dan memelihara harta
milik, dan atas kebahagiaan serta hak untuk meraih dan memperoleh
keamanan'' (pasal 1). Lahirnya dokumen-dokumen di atas menunjukkan bahwa
orang dapat sepakat bahwa HAM harus diberlakukan melalui instrumen hukum
positif agar ada jaminan terhadap perlindungannya, dan dengan demikian
bangsa-bangsa menjadi beradab.

Walaupun institusionalisasi HAM sudah dilakukan, pemenuhan janji-janji
HAM belum terjadi. Bahkan abad kita ini, di mana-mana terjadi persoalan
penindasan, persoalan perbudakan, fasisme, rasisme, juga ekses-ekses
dari modernitas. Padahal, manusia perlu dilindungi dari berbagai macam
hal yang membahayakan kesejahteraan kehidupannya. Oleh karena itu, PBB
yang didirikan sesudah PD II merasa berkewajiban untuk memaklumkan DUHAM
(Deklarasi Universal Hak hak Asasi Manusia) ke seluruh dunia. Untuk
pertama kalinya, dirumuskanı martabat manusia, nilai person manusia dan
kesamaan hak antara laki-laki dan perempuan. Selain hak-hak pembelaan
diri klasik, muncul hak-hak atas prestasi prestasi sosio-kultural
negara, larangan diskriminasi, larangan penyiksaan, hak atas suaka dan
kebebasan untuk nengikat tali perkawinan.

Universalitas hak-hak asasi manusia tidak terletak pada gambaran manusia
yang tersirat di dalamnya, melainkan pada intensi dasariahnya untuk
melindungi manusia dari peristiwa-peristiwa negatif. Ide HAM Barat yang
tampak pada model Prancis, Ainerika dan DUHAM dari PBB tidak
mempersoalkan gambaran manusia. Maksudnya, HAM yang dimaklumkan itu
berlaku untuk setiap manusia tanpa kecuali dan sahih untuk setiap
lingkungan kultural tanpa mengindahkan gambaran manusia di dalam
kebudayaan-kebudayaan yang berbeda-beda. Jadi, intensi dasarlah bahwa
HAM itu diinstitusionalisasikan lebih ditujukan agar tidak terjadi
kembali pelanggaran-pelanggaran terhadap HAM, sebagaimana telah terjadi
dan kemudian melahirkan dokumen-dokumen tentang HAM di atas. Namun
demikian, apabila dicermati lebih dalam, kita dapat menemukan gambaran
manusia yang tercermin di dalam ide HAM Barat Dalam ide HAM Barat,
tergambar manusia sebagai individu yang bebas, setara dan menentukan
dirinyasendiri. Ini adalah konsep individualisme, suatu konsep tentang
manusia yang terbentuk lewat sekularisasi dan individualisasi. Inilah
gambaran di mana individu dimuliakan, bukan kosmos yang sakral atau
kelompok yan mendominasi individu.

\hypertarget{pokok-pokok-lain-3}{%
\subsection{Pokok-pokok Lain}\label{pokok-pokok-lain-3}}

\begin{itemize}
\item
  \emph{Apa maksudnya HAM mendahului negara dan hukum positif?}

  Terlebih dulu harus dipahami apa itu HAM. HAM merupakan hak-hak yang
  dimiliki manusia karena ia adalah manusia. Hak tersebut melekat pada
  dirinya sejak ia lahir, bukan pemberian dari masyarakat atau
  komunitasnya. Oleh karena itu, hak tersebut tidak dapat dipisahkan
  dari manusia itu. Asalnya adalah tradisi liberal Barat. Pemahaman
  liberal Barat memandang HAM sebagai hak yang sudah ada pada diri
  manusia sebelum negara ada. Dengan demikian, HAM mendahului negara
  dalam arti negara tidak menganugrahi warganya dengan HAM. Tanpa negara
  pun manusia sudah memiliki hak-hak asasinya sejak lahir. Sementara
  itu, sistem hukum juga berbeda dari HAM. Karena HAM itu sudah dimiliki
  manusia sejak lahir, adanya HAM bukanlah karena diatur atau diberikan
  oleh sistem hukum. Oleh karena itu, hukum positif juga dibuat sesuai
  dengan kriteria yang mendukung HAM, bukan untuk memaksa manusia agar
  menyerahkan hak-hak asasi mereka kepada pihak lain.
\item
  \emph{Bagaimana asal-usul HAM dalam tradisi Yahudi-Kristiani, Filsafat
  Yunani-Romawi, dan teori Kontrak Sosial?}

  Di dalam tradisi Yahudi-Kristiani, setiap manusia dilihat sebagai
  citra Allah (\emph{imago Dei}). Manusia menyadari keistimewaan ini
  dengan memahaminya melalui pewahyuan di dalam Kitab Suci. Dengan dapat
  memahami pewahyuan Allah itu sendiri manusia juga istimewa. Dengan
  pandangan tersebut, tradisi Yahudi Kristiani mengakui bahwa setiap
  manusia memiliki nilai intrinsik (nilai yang dikandung dalam diri)
  yang menjadikannya berharga sebagai manusia. Tidak satu kekuasaan
  duniawi pun yang dapat memberi nilai intrinsik itu kepada. manusia.
  Dengan demikian, martabat manusia tidak dapat dijadikan sarana bagi
  makhluk lain untuk mencapai tujuannya, tetapi merupakan tujuan yang
  harus dicapai oleh diri manusia itu sendiri. Lebih lanjut lagi,
  pandangan tradisi Yahudi Kristiani ini dikembangkan oleh Thomas
  Aquinas lewat Teori Hukum Kodratnya.

  Tuntutan Teori Hukum Kodrat dari Thomas Aquinas adalah setiap orang
  seharusnya diperlakukan seturut martabatnya sebagai manusia, juga
  memiliki kewajiban moral untuk memperlakukan sesamanya sebagai
  manusia. Alasannya bertitik tolak dari pandangan Kristiani mengenai
  manusia, yaitu (1) manusia adalah citra dan gambar Allah (\emph{imago
  Dei}), (2) manusia adalah pribadi, (3) manusia memiliki suara hati
  yang bebas. Oleh karena tiga gambaran mengenai manusia itulah
  perlakuan terhadap manusia sebagai manusia dijadikan prinsip moral
  yang harus dipegang.

  Di dalam filsafat Yunani-Romawi, penghargaan yang tinggi pada manusia
  dapat ditemukan dalam gagasan filsafat Stoa mengenai martabat manusia.
  Manusia memiliki niartabat lewat partisipasi mereka pada Logos ilahi,
  yaitu kosmos rohaniah dan religius yang di dalamnya terdapat kebebasan
  dan kesetaraan. Oleh karena itu, kepada siapapun manusia harus berbuat
  baik. Perempuan berhak diperlakukan setara dengan laki-laki, hak-hak
  budak dihormati, bahkan musuh juga berhak untuk memperoleh
  pengampunan. Setiap batasan harus diatasi dan seluruh umat manusia
  harus dirangkul tanpa kecuali; umat manusia membentuk semacam
  persaudaraan universal yang merangkul seluruh umat manusia.

  Di dalam teori Kontrak Sosial, pendapat beberapa filsuf berikut dapat
  dijadikan acuan:

  \begin{enumerate}
  \def\labelenumi{\arabic{enumi}.}
  \tightlist
  \item
    Thomas Hobbes melihat ``hak alamiah manusia'' sebagai tadanya
    kendala dari luar bagi manusia (=kebebasan) sehingga ia dapat
    menggunakan kekuasaannya sendiri sekehendaknya untuk memelihara
    dirinya (hak kedaulatan).
  \item
    John Locke akan mengatakan bahwa keadaan alamiah manusia adalah
    situasi kebebasan sepenuh-penuhnya untuk mengatur tindakan-tindakan
    mereka, serta memakai milik pribadi mereka tanpa perlu minta izin
    atau tergantung pada kehendak orang lain (hak milik)
  \item
    Jean-Jacques Rousseau memandang kebebasan sebagai lepasnya manusia
    dari struktur-struktur yang membelenggu sehingga ia dapat terlibat
    di dalam komunitasnya. Setiap orang menjadi tidak bebas dan
    diperbudak bukan karena sejak awal dilahirkan demikian, melainkan
    karena masuk ke dalam struktur yang membelenggu.
  \end{enumerate}

  Dalam hal ini, memang benar bahwa hukum diperlukan agar dengan
  kebebasan yang ia miliki, setiap manusia tidak berusaha menghilangkan
  kebebasan orang lain. Akan tetapi, lebih dari itu, kekuasaan negara
  memiliki kewajiban untuk mengamankan hak-hak kodrati yang sudah
  tersedia di dalam keadaan asali manusia. Bila kewajiban ini diabaikan,
  kepatuhan manusia terhadap kekuasaan negara itu berakhir dan manusia
  kembali pada keadaan alamiahnya (keadaan sebelum ada kontrak).
\item
  \emph{Apa itu institusionalisasi HAM? Apa maksudnya HAM akhirnya
  menjadi norma yang efektif bagi bangsa beradab?}

  Harus diakui bahwa HAM merupakan konsep modern dan belum umum
  dibicarakan sebelum abad 18. Gagasan martabat manusia kiranya tidak
  memiliki dasar bila tidak dilembagakan pada suatu konstitusi atau
  pasal-pasal aturan tertentu. Dengan kata lain, melalui
  institusionalisasi HAM pada pasal-pasal suatu konstitusi atau aturan,
  gagasan HAM diimplementasikan secara praktis dan politis (memengaruhi
  kebijakan yang berlaku bagi banyak orang). Institusionalisasi HAM itu
  dapat dilihat dalam bentuk dokumen atau deklarasi yang melindungi dan
  menjunjung tinggi HAM (lihat di atas). Dengan institusionalisasi, HAM
  menjadi norma yang efektif bagi bangsa beradab. Artinya,
  institusionalisasi menjadi dasar bagi pelaksanaan HAM sehingga dapat
  ``dipaksakan'' bagi bangsa-bangsa yang ingin diakui sebagai bangsa
  beradab.
\item
  \emph{Apa maksudnya klaim universalitas HAM?}

  Pertama: HAM menjadi suatu cara untuk hidup bersama, yang di dalamnya
  martabat manusia tanpa kecuali diangkat dan diperjuangkan sehingga
  tidak dapat diganggu gugat oleh manusia lain.

  Kedua: jangan lupa bahwa HAM juga dapat disebut universal karena
  diperoleh dari kesepakatan atau persetujuan banyak orang. Artinya,
  hak-hak yang terangkum dalam gagasan HAM didasarkan pada kesepakatan,
  dari karena kesepakatan itu pula dapat diubah, baik itu ditambah
  (hak-hak manusia yang sebelumnya tidak diakui menjadi mendapat
  pengakuan sebagai HAM) maupun dikurangi (hak-hak manusia yang
  sebelumnya disepakati sebagai HAM menjadi tidak lagi diakui dalam
  kesepakatan yang baru).

  Ketiga: Universalitas HAM terutama menyangkut intensi dasarnya, yaitu
  untuk menjaga umat manusia agar tidak jatuh pada pengalaman pengalaman
  buruk yang menghina martabat kemanusiaan. Prioritas pada intensi ini
  menjadi penting karena pada kenyataannya bisa muncul banyak gambaran
  manusia, terlebih lagi dalam iklim multikultur dewasa ini. Ini
  merupakan solusi yang diajukan Pak Franky untuk menghadapi
  pertentangan antara DUHAM dan DUHAMIS yang bertitik tolak pada
  gambaran manusia yang berbeda. Oposisi di antara keduanya sebenarnya
  tidak perlu dilakukan: alih-alih berselisih paham mengenai ``bagaimana
  gambaran manusia yang seharusnya mendasari deklarasi HAM?'', adalah
  jauh lebih baik apabila kedua deklarasi tersebut sama-sama berusaha
  untuk menjawab ``apa yang dapat dilakukan untuk mencegah
  pelanggaran-pelanggaran HAM di masa mendatang?''
\end{itemize}

\hypertarget{konsep-kebebasan}{%
\section{Konsep Kebebasan}\label{konsep-kebebasan}}

\begin{quote}
Konsep kebebasan yang dimiliki oheh para filsuf, seperti Hobbes, Locke,
dan Kant, berbeda dari konsep kebebasan yang dimiliki oleh para filsuf,
seperti Rousseau, Tocqueville dan Hannah Arendt. Kedua tradisi besar itu
juga menjadi akar perdebatan antara liberalisme dan komunitarianisme,
khususnya dalam pandangan mereka tentang manusia dan identitasnya. Model
demokrasi deliberatif mencoba mendamaikan kedua arti kebebasan itu lewat
konsep diskursus praktis.
\end{quote}

\hypertarget{konsep-kebebasan-negatif-vs.-positif}{%
\subsection{Konsep Kebebasan Negatif
vs.~Positif}\label{konsep-kebebasan-negatif-vs.-positif}}

Konsep kebebasan yang pertama, yang dipahami para filsuf seperti Hobbes,
merupakan hak-hak liberal klasik. Kebebasan di sini merupakan kebebasan
negatif (\emph{freedom from}), yaitu tidak adanya hambatan-hambatan.
Beranjak dari konsep \emph{state of nature} yang tanpa kendali, Thomas
Hobbes memahami kebebasan sebagai \emph{right of nature} di mana
kebebasan dimiliki setiap orang untuk mempergunakan kekuasaannya sendiri
sekehendaknya untuk memelihara dirinya (\emph{self-preservation}).
Kebebasan akhirnya didefinisikan sebagai ``tiadanya segala kendala atau
penghalang dari luar.'' Sementara itu, John Locke (dengan \emph{state of
nature}-nya yang bukan tanpa kendali) memandang bahwa manusia mempunyai
kebebasan untuk memakai hak milik pribadinya, tetapi ia tidak memiliki
kebebasan untuk menghancurkan dirinya sendiri atau makhluk lain yang
dimilikinya. Selanjutnya, konsep kebebasan Immanuel Kant termasuk di
dalam konsep kebebasan negatif ini ketika ia mengatakan bahwa
kesewenang-wenangan seseorang dibatasi oleh kesewenang-wenangan orang
lain.

Konsep kebebasan yang kedua, yang dinyatakan para filsuf seperti
Jean-Jacques Rousseau, merupakan kebebasan positif (\emph{freedom for}).
Dalam konsep kebebasan ini, kita sebagai kelompok memposisikan diri
sebagai \emph{collective self determination}, Kebebasan positif ini
lebih tua dan kuno, berbeda dengan kebebasan negatif yang baru kemudian.
Kebebasan Rousseau digolongkan ke dalam kebebasan positif dengan
konsepnya mengenai \emph{volunte generale}. Konsep `partisipasi'
Tocqueville atau konsep `legislasi' Hannah Arendt juga memandang
kebebasan dari segi positif. Hannah Arendt mengatakan, ``Bebas berarti
tidak menaklukkan maupun tidak ditaklukkan \ldots{} Bebas berarti bebas
dari ketidaksamaan yang melekat pada segala hubungan penaklukan, yaitu
bergerak di dalam sebuah ruang yang di dalamnya tak ada penaklukkan
maupun ditaklukkan.'' Bagi Arendt, kesetaraan merupakan hakekat
sebenarnya dari kebebasan.

\hypertarget{liberalisme-vs.-komunitarianisme}{%
\subsection{Liberalisme
vs.~Komunitarianisme}\label{liberalisme-vs.-komunitarianisme}}

Kebebasan positif ataupun negatif, kedua tradisi besar tentang konsep
kebebasan itu lah yang menjadi akar perdebatan antara Liberalisme dan
Komunitarianisme, khususnya dalam pandangan mereka tentang manusia dan
identitasnya.

Konsep kebebasan negatif dari para filsuf seperti Hobbes, Locke, dan
Kant adalah dasar bagi Liberalisme. Aliran liberalisme menekankan soal
kebebasan dan kesamaan manusia sebagai individu dalam hal memiliki
kesempatan dalam bidang politik, sosial, ekonomi, dan kebudayaan. Bagi
liberalisme, hukum dan negara menjadi institusi yang mengabdi
kepentingan manusia. Dengan begitu, para penganut liberalisme memandang
manusia sebagai individu yang lepas dari kelompoknya (\emph{unencumbered
self}). \emph{Unencumbered self} mengacu pada pengertian mengenai
individu sebagai subjek yang universal dan abstrak, yaitu dilepaskan
dari konteks masyarakat dan kebudayaan konkret. Jika individu
diabstraksi dengan menyingkirkan ciri-ciri konkret yang berasal dari
komunitasnya, kita akan mendapatkan suatu ``gambaran manusia pada
umumnya.''

Sementara itu, pandangan mengenai kebebasan positif dari para filsuf
seperti Rousseau, Tocqueville, dan Arendt menjadi dasar bagi
Komunitarianisme. Menurut aliran komunitarianisme, komunitas atau
kelompok masyarakat dipandang penting dalam pembentukan individu. Oleh
karena itu, manusia memperoleh kebebasan dan hak-haknya justru ketika ia
Hidup dan bertindak di dalam dan bersama komunitas. Dalam pandangan ini,
subjek atau individualitas berasal dari keanggotaan dalam suatu
komunitas yang terbentuk melalui tradisi-tradisi dan nilai-nilai
kultural. Misalnya, bahwa saya adalah orang Jawa, orang Katolik,
mahasiswa STF, dll. Dalam konteks masyarakat liberal, individu yang
disebut ``\emph{unencumbered self}'' ini toh bagian dari suatu komunitas
juga. Mereka adalah komunitas orang-orang liberal! Jadi, tidak ada
subjek-subjek tanpa konteks. Menurut kaum komunitarian, manusia dalam
pandangan liberalisme berada dalam bahaya kehilangan rasa ``kekitaan,''
yaitu makna kebersamaan yang sangat penting bagi kehidupan sosial.

\hypertarget{masalah-identitas-dalam-pandangan-liberalisme-dan-komunitarianisme}{%
\subsection{Masalah Identitas dalam pandangan Liberalisme dan
Komunitarianisme}\label{masalah-identitas-dalam-pandangan-liberalisme-dan-komunitarianisme}}

Liberalisme dan komunitarianisme persis berbeda di dalam pandangan
mengenai identitas manusia. Bagi liberalisme, identitas manusia individu
harus ditonjolkan, bahkan negara atau hukum didirikan untuk mengabdi
kepentingan manusia. Pemikiran tersebut lahir dari latar belakang
sejarah berupa perlawanan terhadap kekuasaan absolut dalam pemerintahan
monarki pada abad 17-18. Untuk melawan kekuasaan itu, hak-hak asasi
manusia individual ditonjolkan sebagai hak yang tidak dapat dikekang
oleh negara. Individu terlepas dari ikatan terhadap otoritas dan
kelompok. Memang benar bahwa setiap individu konkret berasal dari latar
belakang konteks kebudayaan atau masyarakat tertentu. Akan tetapi, bila
diabstraksi dengan menyingkirkan ciri-ciri konkret yang berasal dari
konteks komunitasnya, individu itu akan muncul dengan gambaran sebagai
``manusia universal.'' Individu ini terutama memiliki ciri universal,
yaitu mampu memilih tujuan-tujuan menurut preferensi individualnya.

Sementara itu bagi komunitarianisme, keyakinan liberalisme bahwa aspek
individu harus ditonjolkan dan dilepaskan dari kelompoknya menjadi
keprihatinan yang mendalam. Alasannya adalah pengutamaan terhadap aspek
individu dilihat tidak saja berakibat emansipasi (persamaan hak) seperti
yang diyakini liberalisme, tetapi juga berisiko mencerabut individu itu
dari komunitas tempat ia bertumbuh. Individu yang bebas menurut
komunitarianisme adalah individu yang dapat mempertahankan identitasnya
hanya di dalam konteks komunitas (sosio kultural) yang menjadi tempat ia
tumbuh dan berkembang. Bagi komunitarianisme yang meyakini pentingnya
komunitas atau kelompok di dalam pembentukar manusia, tercerabutnya
manusia indiv!du dari kelompoknya membuatnya kehilangan identitas
kolektif, yang juga mencakup rasa dimiliki oleh suatu kelompok

\hypertarget{demokrasi-deliberatif}{%
\subsection{Demokrasi Deliberatif}\label{demokrasi-deliberatif}}

Model demokrasi deliberatif mencoba mendamaikan kedua arti kebebasan itu
lewat konsep diskursus praktis. Melalui konsep demokrasi deliberatif,
terjadi pendamaian. Penjelasannya adalah sebagai berikut. Teori
demokrasi deliberatif menunjuk pada prosedur untuk menghasilkan
aturan-aturan. Pertanyaan yang dilontarkan: bagaimana
keputusan-keputusan politis diambil, dan dalam kondisi kondisi manakah
aturan tersebut dihasilkan sedemikian rupa, sehingga para warganegara
mematuhi aturan-aturan itu?

Jadi, demokrasi deliberatif meminati persoalan kesahihan
keputusan-keputusan kolektif. Alasan-alasan yang bagus untuk sebuah
keputusan politis haruslah diuji secara publik sedemikian rupa sehingga
alasan-alasan tersebut diterima secara intersubjektif oleh semua
warganegara dan tidak menutup diri dari kritik dan revisi yang
diperlukan (Lih. FBH, 2009:128-129). Dari situ, kita dapat memahami
bahwa dalam demokrasi deliberatif meniscayakan diskursus. Teori
diskursus yang dimaksudkan merupakan teori yang memperjelas
praktik-praktik negara hukum yang sudah ada dan mendorong pembukaan
kanal-kanal komunikasi politis di dalamnya. Dalam demokrasi deliberatif,
semua tipe diskursus praktis beroperasi, yaitu di dalam formasi opini
dan aspirasi secara demokratis agar secara publik alasan-alasan bagi
peraturan politis dapat diuji.

Untuk sampai pada diskursus semacam ini, setiap manusia (warga negara)
perlu diandaikan memiliki kebebasan dalam pengertian liberal manupun
komunitarian. Tanpa kebebasan liberal, tidak mungkin terjadi diskursus.
Akan tetapi, perlu diketahui pula bahwa kebebasan yang diperlukan tidak
hanya kebebasan individual (\emph{self-determination}), tetapi juga
kebebasan kolektif (\emph{collective self-determination}), karena tidak
ada individu tanpa konteks. ini hanya efektif jika
diinstitusionalisasikan dalam suatu komunitas. Oleh karena itu, dalam
demokrasi deliberatif (hukum positif), kedua kebebasan ini sifatnya
komplementer (saling melengkapi).

\hypertarget{konsep-kebebasan-dalam-pemikir-pemikir-filsafat-politik}{%
\subsection{Konsep Kebebasan dalam pemikir-pemikir filsafat
politik}\label{konsep-kebebasan-dalam-pemikir-pemikir-filsafat-politik}}

Konsep kebebasan dipahami secara berbeda-beda oleh para tokoh pemikir
besar. Perbedaan itu terlihat dalam pemikiran Hobbes, Locke, Kant,
Rousseau, Tocqueville, dan Hannah Arendt. Keenam pemikir tersebut dapat
dibagi menjadi dua kelompok menurut sifat dasar kebebasan, yaitu
kebebasan Negatif (Hobbes, Locke, Kant) dan kebebasan Positif (Rousseau,
Tocqueville, dan Hannah Arendt). Berikut akan diuraikan pemikiran keenam
tokoh besar tersebut.

\hypertarget{konsep-kebebasan-negatif}{%
\subsubsection{Konsep Kebebasan
Negatif}\label{konsep-kebebasan-negatif}}

Hobbes memahami kebebasan sebagai hak kodrati manusia, dengan mengatakan
bahwa kebebasan merupakan tiadanya kendala dari luar diri manusia
sehingga setiap orang dapat menggunakan kekuasaannya sendiri
sekehendaknya untuk memelihara dirinya. Persoalannya adalah dengan
kebebasan itu, manusia menjadi makhluk rasional yang menjadi egois
radikal (mengejar kepentingan diri tanpa batas). Oleh karena itu, hak
kodrati itu diatur dengan tatanan hukum kodrat sehingga hak itu tetap
bisa terwujud dan tidak malah ditiadakan oleh orang lain. Fakta yang
dipegang dalam hal ini adalah orang lain juga bebas seperti diriku.

Mirip dengan Hobbers, Locke juga memahami kebebasan sebagai keadaan
alamiah manusia. Kebebasan yang dimaksud adalah kebebasan
sepenuh-penuhnya untuk mengatur tindakan tindakan mereka, serta memakai
milik pribadi mereka tanpa perlu minta izin atau tergantung pada
kehendak orang lain. Sama seperti Hobbes, kebebasan itu juga diatur
dengan tatanan hukum kodrat sehingga setiap orang mengalami situasi
kesamaan, tidak seorangpun dapat merugikan hidup, kesehatan, serta
kepunyaan orang lain.

Sementara itu, Kant memahami kebebasan dalam dua arti yaitu kebebasan
Eksternal dan kebebasan Internal. Kebebasan Eksternal berarti kebebasan
dipandang sebagai kesanggupan seseorang untuk bertindak terlepas dari
kuasa di luar dirinya. Sementara kebebasan Internal adalah kemampuan
seseorang untuk bertindak sesuai dengan hukum yang ia buat sendiri dan
pada saat yang sama dapat berlaku secara universal universal (ingat
etika Deontologis Kant).

Dengan kata lain, menurut Kant, tindakan yang bebas bukan berarti
dilakukan seseorang tanpa ikatan dengan hukum apapun (eksternal),
melainkan justru dilakukan dan dapat dipertanggungjawabkan dengan
mengikatkan diri pada hukum yang ia buat sendiri (internal, dengan
bebas). Dengan demikian, tindakannya yang bebas itu akan sesuai dengan
kebebasan setiap orang.

Kesimpulannya: kebebasan menurut Hobbes, Locke, dan Kant merupakan
kebebasan negatif, yaitu kesanggupan seseorang untuk mengatur tindakan
tindakan mereka lepas dari hambatan-hambatan di luar diri orang
tersebut. Hukum ditujukan supaya orang lain tidak meniadakan kebebasan
yang dimiliki orang tersebut dan ia sendiri tidak meniadakan kebebasan
orang lain.

\hypertarget{konsep-kebebasan-positif}{%
\subsubsection{Konsep Kebebasan
Positif}\label{konsep-kebebasan-positif}}

Berbeda dari pemikir-pemikir di atas, Rousseau memandang kebebasan
sebagai lepasnya manusia dari struktur-struktur yang membelenggu
sehingga ia dapat terlibat di dalam komunitasnya. Menurutnya, setiap
orang terlahir bebas. Orang kemudian menjadi tidak bebas karena masuk ke
dalam struktur-struktur masyarakat yang membelenggu.

Tocqueville memahami kebebasan bukan sebagai kesamaan kondisi kondisi
sosial di dalam masyarakat (orang yang satu tidak lebih istimewa
daripada yang lain). Kesamaan kondisi-kondisi sosial itu bahkan dapat
membahayakan kebebasan karena membuat seseorang tidak gampang diyakinkan
oleh orang-orang lain, serta membuatnya tidak siap untuk mengakui salah
satu dari mereka sebagai pemimpin. Kebebasan justru berarti memampukan
setiap orang untuk menciptakan masyarakat demokratis, yaitu masyarakat
yang dapat membiarkan massa atau opini umum untuk memimpin diri mereka.

Hannah Arendt memahami konsep `bebas' sebagai tidak menaklukkan maupun
tidak ditaklukkan, serta bebas dari ketidaksamaan yang muncul akibat
dari penaklukan. Kebebasan berarti setiap orang mampu melakukan tindakan
bersama dan menciptakan kekuasaan melalui tindakan bersama tersebut.
Kebebasan ini hidup di dalam ruang publik, yang menjadi tempat
orang-orang bergaul dengan bertindak dan berbicara satu sama lain.
Kesimpulannya: kebebasan menurut Rousseau, Tocqueville, dan Arendt
merupakan kebebasan positif, yaitu kesanggupan setiap orang untuk
melakukan sesuatu. Dalam hal ini, kebebasan seseorang memampukannya
untuk berpartisipasi di dalam komunitasnya, melakukan tindakan bersama,
dan menjadi bagian dari masyarakat demokratis.

\hypertarget{pokok-pokok-lain-4}{%
\subsection{Pokok-pokok Lain}\label{pokok-pokok-lain-4}}

\hypertarget{apakah-demokrasi-deliberatif-itu-dikatakan-bahwa-liberalisme-dan-komunitarianisme-didamaikan-oleh-konsep-diskursus-praktis.-apa-maksudnya-pada-aspek-apa-perlu-kebebasan-positif-pada-aspek-apa-perlu-kebebasan-negatif}{%
\subsubsection{Apakah demokrasi deliberatif itu? Dikatakan bahwa
Liberalisme dan Komunitarianisme didamaikan oleh konsep Diskursus
Praktis. Apa maksudnya? Pada aspek apa perlu kebebasan positif? Pada
aspek apa perlu kebebasan
negatif?}\label{apakah-demokrasi-deliberatif-itu-dikatakan-bahwa-liberalisme-dan-komunitarianisme-didamaikan-oleh-konsep-diskursus-praktis.-apa-maksudnya-pada-aspek-apa-perlu-kebebasan-positif-pada-aspek-apa-perlu-kebebasan-negatif}}

Demokrasi deliberatif merupakan bentuk demokrasi yang mendesak
masyarakat membuka ruang-ruang dan saluran komunikasi politis dengan
tujuan mencapai kesepahaman (konsensus) di dalam masyarakat mengenai
suatu persoalan. Tindakan komunikatif mendapat tekanan penting di dalam
model demokrasi ini, yaitu sebagai suatu tindakan rasional dalam bentuk
diskursus praktis yang berorientasi pada kesepahaman bersama. Dengan
kata lain, diskursus praktis merupakan suatu cara bagi masyarakat untuk
berkomunikasi secara rasional dalam rangka mencari kesepahaman bersama.
Di dalam diskursus praktis ini, setiap anggota masyarakat dapat saling
berbicara untuk menganalisis norma-norma yang mengatur hidup keseharian
mereka.

Persis pada pokok diskursus praktis inilah konsep demokrasi deliberatif
dari Habermas mencoba mendamaikan liberarisme dan komunitarianisme. Ada
unsur liberalisme ketika Habermas merumuskan identitas manusia di dalam
diskursus praktis. Identitas manusia yang berdiskursus bagi Habermas
tidak terlalu dilekatkan pada suatu komunitas tertentu. Identitas
individu tidak diturunkan langsung dari identitas suatu kelompok.

Akan tetapi, ada pula unsur komunitarianisme yang dimuat, yaitu bahwa
kepentingan-kepentingan individual baru dapat diabstraksi menjadi
kepentingan bersama setelah melalui proses komunikasi dengan orang lain.
Melalui proses komunikasi itu pula para individu itu membangun identitas
mereka; melalui komunikasi mereka sadar akan identitas dan kepentingan
mereka yang berbeda, namun mendapat pengakuan dari orang-orang lain.
Dengan kata lain, Habermas membentuk suatu pemahaman baru mengenai
tempat terbentuknya identitas manusia, yaitu komunikasi. Di dalam
diskursus praktis, kebebasan positif diperlukan, dalam arti setiap
individu dapat memperoleh kebebasan untuk mengernbangkan partisipasinya
di dalam diskursus itu. Kebebasan negatif juga diperlukan di dalam
diskursus praktis, dalam arti setiap individu harus dibebaskan dari
aturan aturan atau hukum yang memaksa, apalagi dari tindak kekerasan
yang dapat menghalangi mereka berdiskursus dan berkomunikasi.

\hypertarget{bahan-bahan-rujukan-2}{%
\subsection{Bahan-bahan Rujukan}\label{bahan-bahan-rujukan-2}}

\begin{enumerate}
\def\labelenumi{\arabic{enumi}.}
\tightlist
\item
  F. Budi Hardiman. \emph{Demokrasi Deliberatif: Menimbang Negara Hukum
  dan} \emph{Ruang Publik dalam Teori Diskursus Jürgen Habermas}.
  Yogyakarta: Kanistus, 2009.
\item
  F. Budi Hardiman. \emph{Filsafat Politik} (Silabus 2014). Jakarta:
  Sekolah Tinggi Filsafat Driyarkara, 2014.
\item
  F. Budi Hardiman. \emph{Hak-hak Asasi Manusia: Polemik dengan Agama
  dan} \emph{Kebudayaan}. Yogyakarta: Kanisius, 2011.
\end{enumerate}

\hypertarget{filsafat-ketuhanan}{%
\chapter{Filsafat Ketuhanan}\label{filsafat-ketuhanan}}

Filsafat ketuhanan adalah pemikiran filosofis tentang Tuhan. Sebagaimana
filsafat sebagai ilmu, di dalam Filsafat Ketuhanan manusia memastikan,
menata dan mengembangkan pengetahuannya berkaitan dengan ``Tuhan''
secara objektif dan sistematik, bukan dari sudut-sudut tertentu,
melainkan secara mendasar. Dengan menyingsingnya masa Pencerahan
(\emph{Aufklarung}), filsafat menjadi kritis terhadap agama, bahkan
kemudian setelah tahap ateisme, para filsuf diam-diam sepakat bahwa
filsafat tidak dapat berbicara tentang Tuhan. Di pihak lain, orang
beragama pun kelihatan menolak pemikiran rasional tentang Tuhan,
menganggapnya sebagai tidak bermanfaat. Namun demikian, orang yang
percaya kepada Tuhan sesungguhnya ditantang untuk mempertanggungjawabkan
keyakinannya akan Tuhan secara rasional, karena kepercayaan akan Tuhan
baginya adalah kebenaran yang menjadi dasar seluruh hidupnya. Di situlah
filsafat Ketuhanan tetap relevan, yaitu sebagai suatu bentuk
pertanggungjawaban. Berikut akan diuraikan dua tema penting terkait
dengan usaha rasional tersebut, yaitu berkenaan dengan suara hati
sebagai petunjuk akan adanya Allah dan kajian mengenai ateisme modern.

\hypertarget{suara-hati-sebagai-petunjuk-adanya-allah}{%
\section{Suara Hati Sebagai Petunjuk Adanya
Allah}\label{suara-hati-sebagai-petunjuk-adanya-allah}}

\begin{quote}
Kenyataan bahwa manusia memiliki kesadaran moral, atau dengan kata lain
memiliki suara hati, sering dikatakan merupakan petunjuk paling kuat
akan adanya Allah. Jelaskanlah anggapan itu!
\end{quote}

Kenyataan bahwa manusia memiliki kesadaran moral, dengan kata lain suara
hati, sering dikatakan sebagai petunjuk paling kuat akan adanya Allah.
Suara hati (\emph{conscience}) merupakan tempat manusia bersentuhan
dengan realitas ilahi, demikian kata John Henry Newman (1801-1890).
Menurutnya, dalam suara hati kita menyadari bahwa kita berkewajiban
mutlak untuk melakukan yang baik dan yang benar dan, dan juga
sebaliknya, yaitu menolak yang tidak baik dan tidak benar. Suara hati
bagaikan panggilan dari suatu realitas personal yang berkuasa atas diri
kita yang kalau kita mengikutinya akan membuat kita merasa bernilai,
aman dan sedia untuk menyerah. Sebelumnya, Immanuel Kant telah
menyatakan bahwa kesadaran moral manusia (kewajiban untuk melakukan yang
baik dan benar) hanya dapat dimengerti dengan pengandaian (postulat)
akan adanya Allah. Kesadaran moral merupakan tempat paling utama manusia
bertemu dengan dasar eksistensinya, yaitu Allah. Franz Magnis-Suseno
memberikan uraian singkat bagaimana kesadaran moral manusia menunjuk
pada Allah sebagai berikut:

\begin{enumerate}
\def\labelenumi{\arabic{enumi}.}
\tightlist
\item
  Manusia itu berkesadaran moral (mempunyai suara hati).
\item
  Dalam kesadaran moral, manusia sadar bahwa ia mutlak wajib untuk
  memilih yang benar.
\item
  Kesadaran itu berakar dalam hati nurani, yaitu dalam kesadaran di
  dasar hati kita bahwa kita wajib secara mutlak untuk memilih yang
  baik, jujur, adil dan menolak yang sebaliknya.
\item
  Kesadaran akan kewajiban mutlak ini tidak berasal dari dunia luar dan
  juga tidak dari diri kita sendiri.
\item
  Melainkan, kesadaran itu kita sadari langsung sebagai jawaban terhadap
  suatu tuntutan dari sebuah realitas yang kita hadapi, yang dari
  padanya kita tidak dapat lari, di mana sikap kita terhadapnya
  menentukan mutu kita sebagai manusia.
\item
  Realitas itu bersifat mutlak, personal dan suci dan itulah yang kita
  sebut Allah.
\end{enumerate}

\hypertarget{hubungan-suara-hati-dan-kesadaran-moral}{%
\subsection{Hubungan Suara Hati dan Kesadaran
Moral}\label{hubungan-suara-hati-dan-kesadaran-moral}}

Kesadaran moral merupakan kesadaran bahwa perbuatan kita bisa bernilai
baik atau buruk. Yang menjadi acuan dalam penilaian ini adalah martabat
kita sebagai manusia. Kualitas kita sebagai manusia tergantung dari
apakah kita memilih yang baik atau yang buruk itu. Suara hati adalah
kesadaran moral dalam situasi konkret. Dikatakan konkret karena
kesadaran itu mengajak kita pada situasi tertentu untuk memilih antara
melalukan yang benar atau yang tidak benar, sekaligus mengetahui bahwa
kita tidak boleh melakukan yang tidak benar. Ada tiga sifat yang biasa
dibawa oleh suara hati:

\begin{enumerate}
\def\labelenumi{\arabic{enumi}.}
\tightlist
\item
  Mutlak: ia muncul dan berlaku tanpa tergantung pada perasaan
  senang-tidak senang, untung-rugi, dsb.
\item
  Universal: ia muncul dan membawa keyakinan pada setiap orang bahwa
  "orang lain seharusnya juga berbuat yang sama seperti yang aku lakukan
  dalam situasi konkret.
\item
  Rasional: kemunculannya dapat dipertanggungjawabkan.
\end{enumerate}

Suara hati itu berkaitan dengan hati nurani. Keterkaitan itu dapat
dijelaskan sebagai berikut: suara hati sesungguhnya masih dapat keliru
ketika suatu pertimbangan moral yang baru muncul dan membuat kita
mempertanyakan atau meragukan apa yang sebelumnya kita pandang sebagai
kewajiban moral. Dalam situasi tersebut, nalar akan ikut digunakan untuk
membantu kita menimbang-nimbang, namun kita juga harus mengakui bahwa
nalar kita juga bisa keliru. Akan tetapi, pada situasi konkret di dalam
lubuk hati kita tetap ada dorongan untuk terarah pada pilihan yang baik,
jujur, dan adil daripada memilih yang jahat, tidak jujur, tidak adil,
dst. Dorongan pada keterarahan itulah yang disebut sebagai hati nurani.
Suara hati, sebagai kesadaran moral dalam situasi konkret, berakar pada
hati nurani itu.

\hypertarget{suara-hati-sebagai-petunjuk-paling-kuat-akan-adanya-allah}{%
\subsection{Suara Hati sebagai Petunjuk Paling Kuat Akan Adanya
Allah}\label{suara-hati-sebagai-petunjuk-paling-kuat-akan-adanya-allah}}

Hati nurani selalu mendorong kita untuk secara mutlak selalu memilih
yang baik dan benar, serta menolak yang berlawanan darinya. Persoalannya
adalah dari mana dorongan yang mutlak itu berasal? Dorongan itu tidak
mungkin berasal dari realitas di luar diri manusia (alam, orang lain,
masyarakat) karena tuntutan apa pun dari luar selalu dapat dipertanyakan
oleh suara hati, apakah memang sesuai atau tidak dengan hati nurani.
Sejauh tidak bertentangan dengan hati nurani, barulah tuntutan itu
kemudian dapat kita ikuti. Dengan kata lain, dorongan atau tuntutan itu
selalu sudah kalah terhadap suara hati. Dorongan itu juga tidak mungkin
berasal dari diri manusia sendiri karena jika manusia dapat mewajibkan
diri, ia juga dapat mencabut kewajiban itu. Padahal, tuntutan atau
dorongan moral itu bersifat mutlak dan pasti. Ketika tuntutan atau
dorongan moral itu muncul, kita langsung berhadapan dengan tuntutan
mutlak itu.

Kita sering ingin menghindar, tetapi langsung sadar bahwa kita salah dan
tak bisa lepas. Kita tidak dapat menghindar darinya. Suara hati kita
sadari sebagai suara yang tidak dapat kita cabut begitu saja, tetapi
langsung kita hadapi tanpa dapat kita hindari, seakan-akan tuntutan itu
tidak peduli apakah kita senang dengannya atau tidak. Oleh karena itu,
tuntutan mutlak ini membawa indikasi bahwa suara itu dari luar diri
kita, bahkan dari luar lingkungan kita, dari luar dunia. Suara itu
transenden. Suara itu mutlak karena melampaui segalanya dan tak dapat
diguncangkan oleh segala apa pun. Dalam suara hati, kita berhadapan
dengan realitas transenden.

\hypertarget{ateisme-modern}{%
\section{Ateisme Modern}\label{ateisme-modern}}

\begin{quote}
Salah satu teori tentang Ateisme modern ditemukan dalam pandangan L.A.
Feuerbach (1804-1872) tentang agama sebagai proyeksi. Bertolak dari
ajaran Feuerbach ini, Karl Marx (1818-1883) mengajarkan agama sebagai
opium masyarakat. Kedua pandangan ini pantas dievaluasi.
\end{quote}

\hypertarget{pandangan-l.a.-feuerbach}{%
\subsection{Pandangan L.A. Feuerbach}\label{pandangan-l.a.-feuerbach}}

Salah satu teori tentang Ateisme modern ditemukan dalam pandangan L.A.
Feuerbach (1804-1872) tentang agama sebagai proyeksi. Pandangan ini
disebut teori ateisme modern karena merupakan teori yang menyangkal
keberadaan Allah dalam argumentasi yang rasional, artinya memiliki
pendasaran yang ilmiah.

Bagaimana pandangan ini diuraikan akan kita mulai dengan kritik
Feuerbach terhadap Filsafat Hegel. Feuerbach mengkritik Hegel yang
memberi kesan seakan-akan yang nyata adalah Allah (yang tidak
kelihatan), sedangkan manusia (yang kelihatan) hanyalah wayangnya.
Padahal, yang nyata tak terbantahkan justru manusia yang kelihatan.
Sehingga, bukan manusia itu pikiran Allah, melainkan Allah adalah
pikiran manusia. Menurut Feuerbach, filsafat Hegel justru menunjukkan
kemenangan agama atas rasionalitas.

Kritik Feuerbach atas Hegel di atas mengandaikan bahwa pengalaman
inderawi manusia lah realitas yang tak terbantahkan, dan bukan pikiran
spekulatifnya. Realitas manusia inderawi adalah kepastian yang tidak
dapat dibantah karena langsung menyatakan diri. Dengan dasar itu,
mengenai Allah, bukanlah Allah yang menciptakan manusia, melainkan
sebaliknya Allah adalah ciptaan angan-angan manusia. Maka, agama
hanyalah sebuah proyeksi manusia: Allah, malaikat, surga, neraka, tidak
mempunyai kenyataan pada dirinya sendiri, melainkan hanya merupakan
gambar-gambar yang dibentuk manusia tentang dirinya sendiri, sebagai
angan-angan manusia tentang hakikatnya sendiri. Hanya, manusia kemudian
lupa bahwa angan-angan tersebut adalah hasil ciptaannya sendiri.

Agama adalah penyembahan manusia terhadap hasil ciptaannya sendiri,
tanpa manusia menyadarinya. Yang sebenarnya hanyalah angan-angan lalu
dianggap memiliki eksistensinya sendiri sehingga manusia takut kepadanya
dan menghormatinya sebagai Allah. Dengan demikian, manusia manusia
menyatakan keseganannya terhadap hakikatnya sendiri. Di sinilah agama
mengungkapkan keterasingan manusia dari dirinya sendiri. Menurut
Feuerbach, hakikat ilahi bukan lain adalah hakikat manusia, atau lebih
tepat, hakikat manusia yang dipisahkan dari batas-batas manusia
individual yang nyata dan jasmani, yang lalu diobjektifkan (artinya
dipandang dan dipuja sebagai makhluk lain yang berbeda dari padanya).

Bagi Feuerbach, sebenarnya agama mempunyai nilai positif karena
merupakan proyeksi hakikat manusia. Dalam agama, manusia dapat melihat
dia yang kuasa, kreatif, baik, berbelas kasihan, dapat saling
menyelamatkan, dsb. Celakanya, manusia lupa bahwa proyeksi itu adalah
dirinya sendiri, menganggapnya memiliki eksistensinya sendiri, bahkan
jauh lebih hebat (mahakuasa, mahabaik, maha-adil, mahatahu) sehingga
manusia takut dan menyembahnya. Di situlah manusia menjadi lumpuh. Bukan
merealisasikan hakikatnya, manusia menjadi pasif, mengharapkan berkah
dari padanya, berdoa kepadanya. Agama dengan demikian mengasingkan
manusia.

\hypertarget{pandangan-karl-marx}{%
\subsection{Pandangan Karl Marx}\label{pandangan-karl-marx}}

Bertolak dari ajaran Feuerbach, Karl Marx (1818-1883) mengajarkan agama
sebagai opium rakyat jelata atau ``candu masyarakat.'' Ajaran ini sering
diartikan seakan-akan Marx menuduh agama menyesatkan dan menipu rakyat,
bahkan lebih lanjut agama sebagai ciptaan kelas-kelas atas untuk
menenangkan rakyat tertindas. Akan tetapi, ucapan Marx ini sebenarnya
untuk menanggapi kritik agama Feuerbach. Marx setuju dengan kritik
Feuerbach, tetapi menurutnya Feuerbach berhenti di tengah jalan. Benar
bahwa agama adalah angan-angan manusia, tetapi Feuerbach tidak bertanya
mengapa manusia melarikan diri ke khayalan daripada mewujudkan diri
dalam kehidupan nyata. Jawaban Marx adalah karena kehidupan nyata, yaitu
struktur kekuasaan dalam masyarakat, tidak mengizinkan manusia untuk
mewujudkan kekayaan mereka. Manusia melarikan diri ke dunia khayalan
karena dunia nyata menindasnya. Jadi, agama sebenarnya merupakan protes
manusia terhadap keadaannya yang terhina dan tertindas.

\begin{quote}
The foundation of irreligious criticism is: Man makes religion, religion
does not make man. Religion is, indeed, the self-consciousness and
self-esteem of man who has either not yet won through to himself, or has
already lost himself again. But man is no abstract being squatting
outside the world. Man is the world of man---state, society. This state
and this society produce religion, which is an inverted consciousness of
the world, because they are an inverted world. Religion is the general
theory of this world, its encyclopaedic compendium, its logic in popular
form, its spiritual point d'honneur, its enthusiasm, its moral sanction,
its solemn complement, and its universal basis of consolation and
justification. It is the fantastic realization of the human essence
since the human essence has not acquired any true reality. The struggle
against religion is, therefore, indirectly the struggle against that
world whose spiritual aroma is religion.

Religious suffering is, at one and the same time, the expression of real
suffering and a protest against real suffering. Religion is the sigh of
the oppressed creature, the heart of a heartless world, and the soul of
soulless conditions. It is the opium of the people.

The abolition of religion as the illusory happiness of the people is the
demand for their real happiness. To call on them to give up their
illusions about their condition is to call on them to give up a
condition that requires illusions. The criticism of religion is,
therefore, in embryo, the criticism of that vale of tears of which
religion is the halo.

Source: Marx, K. 1976. ``Introduction to A Contribution to the Critique
of Hegel's Philosophy of Right.'' In \emph{Marx/Engels Collected Works}
(MECW), v. 3. New York.
\end{quote}

Karena agama adalah ilusi manusia tentang keadaannya, maka kritik tidak
boleh berhenti pada agama, melainkan harus diarahkan pada keadaan
sosial-politik yang mendorong manusia ke dalam agama. Kesimpulan Marx,
``Kritik surga berubah menjadi kritik dunia, kritik agama menjadi kritik
hukum, kritik teologi menjadi kritik politik.'' Yang dibutuhkan bukan
kritik agama, melainkan revolusi. Agama akan menghilang dengan
sendirinya apabila manusia dapat membangun dunia yang memungkinkan
manusia mengembangkan hakikatnya secara nyata dan positif.

\hypertarget{evaluasi}{%
\subsection{Evaluasi}\label{evaluasi}}

\hypertarget{kritikan-atas-feuerbach}{%
\subsubsection{Kritikan atas Feuerbach}\label{kritikan-atas-feuerbach}}

Apakah benar bahwa agama \emph{tidak lebih dari} proyeksi manusia?
Memang harus diakui, bahwa dalam agama-agama ditemukan unsur-unsur yang
mencerminkan cita-cita, prasangka, dan emosi manusia. Banyak hal
dipercayai dan dilakukan atas nama agama tetapi sebenarnya tidak
ditemukan dalam wahyu asli agama yang bersangkutan karena merupakan
interpretasi yang miring atau tambahan kontekstual kemudian. Juga adanya
institusionalitas dalam agama-agama yang terjadi di kemudian hari. Semua
itu merupakan unsur manusiawi yang merupakan proyeksi manusia.
Masalahnya, apakah agama melulu proyeksi manusia? Pertanyaan ini tidak
terjawab dan terbuktikan melalui argumentasi Feuerbach. \emph{Bahwa
agama-agama mengandung proyeksi manusia tidak membuktikan (atau berarti)
bahwa agama tidak lebih dari sekadar proyeksi.} Bagaimana kebenaran
mengenai agama pada dirinya sendiri, apakah Allah itu ada atau tidak,
ternyata tidak masuk atau tidak tersentuh dalam kritik agama Feuerbach.

Lebih dari hal di atas, pengakuan Allah sebagai proyeksi manusia (bahwa
Allah Mahakuasa, Mahabaik, dst., adalah proyeksi dari manusia yang
berkuasa, yang baik, dst.), menyisakan pertanyaan dari mana manusia
mendapat subjek atau pembawa sifat-sifat terasing ``ke-maha-an'' atau
``ketakberhinggaan'' Allah yang tidak ada pada diri manusia tersebut.
Dalam lingkup pengalaman kita, tidak ada yang tak berhingga, tidak ada
yang maha-baik, maha-kuasa, atau pun maha bijaksana. Jadi tidak mungkin
bahwa unsur ke-maha-an atau ketakberhinggaan tersebut merupakan proyeksi
hakikat manusia. Tindakan manusia merentangkan hati dan pikiran ke arah
Yang Tak Berhingga hanya mungkin kalau manusia mempunyai suatu
pengalaman yang sungguh-sungguh tentang Yang Tak Berhingga tersebut.
Allah dapat dipikirkan hanya jika kita secara nyata tersentuh oleh
realitas-Nya.

Kalau ternyata Allah itu ada, kritik Feuerbach bahwa agama mengasingkan
manusia justru bermasalah. Kalau Allah Pencipta segala yang ada di dunia
itu ada, maka menghormati dan menyembah Allah tentu tidak mungkin
menjauhkan manusia dari dirinya sendiri. Karena diri manusia sendiri
berdasar pada Allah, maka ia justru akan menemukan diri apabila ia
menemukan Allah. Maka, kalau ada Allah, beragama adalah sikap manusia
yang paling tepat, paling masuk akal dan paling akan membantu manusia
dalam merealisasikan hakikatnya sebagai manusia.

\hypertarget{kritikan-atas-karl-marx}{%
\subsubsection{Kritikan atas Karl Marx}\label{kritikan-atas-karl-marx}}

Kritik agama Marx merupakan tantangan bagi agama-agama, tetapi yang khas
adalah bahwa baginya agama menunjuk pada ketidakberesan keadaan dalam
masyarakat. Kritik Marxisme bahwa agama melumpuhkan perlawanan
kelas-kelas tertindas sehingga menguntungkan kelas-kelas atas juga perlu
ditanggapi dengan sungguh-sungguh. Tidak jarang agama menjadi sekutu
para penghisap dan penindas. Akan tetapi, ada dua pertanyaan terkait
dengan kritik Karl Marx sendiri:

\emph{Benarkah agama pada hakikatnya merupakan pelarian?} Dalam
kenyataan, profil orang beragama yang otentik menunjukkan bahwa
pencarian akan Allah bukan hanya tidak mengasingkan manusia, tetapi
justru mengembangkan identitas dan hakikat dirinya yang positif.

\emph{Benarkah agar dapat mengembangkan diri sebagai makhluk sosial dan
politik manusia harus berhenti tunduk terhadap Allah?} Dalam praksisnya,
agama justru sering kali mendorong para penganutnya untuk membangun
masyarakat yang mengayomi mereka yang miskin dan lemah, sehingga
tercipta masyarakat yang positif, damai, saling menghormati, melawan
ketidakadilan, dan juga penindasan terhadap mereka yang tidak berdaya.

Pada akhirnya, pola kritik agama Feuerbach dan Marx menunjukkan apa yang
disebut sebagai reduksionalisme. Mereka mereduksi atau mengembalikan
sebuah fenomena A (dalam hal ini, hal percaya pada Allah) pada suatu
realitas B (dalam hal ini, soal realisasi diri). Jadi yang dicari
kebenarannya bukan gagasan pengakuan akan Allah, tetapi apa-apa yang
disangkakan ada di balik atau di belakang gagasan mengenai pengakuan
akan Allah tersebut. Kebenaran tentang ada atau tidaknya Allah pun
sebenarnya tidak tersentuh dengan pola kritik yang demikian.

\hypertarget{pokok-pokok-lain-5}{%
\subsection{Pokok-pokok Lain}\label{pokok-pokok-lain-5}}

\hypertarget{apa-yang-dimaksud-dengan-ateisme-modern}{%
\subsubsection{Apa yang dimaksud dengan ateisme
modern?}\label{apa-yang-dimaksud-dengan-ateisme-modern}}

Ateisme adalah pandangan yang menolak adanya Tuhan. Ia bukan saja tidak
percaya bahwa Tuhan itu ada, melainkan percaya bahwa Tuhan tidak ada.
Dengan argumen-argumennya, paham ini menegasi adanya Tuhan. Ateisme
modern dibedakan dengan ateisme lainnya atas dasar argumennya: ia tidak
mendasarkan pandangannya pada pernyataan bahwa Tuhan hanyalah objek
kepercayaan takhayul, melainkan mengasalkannya pada kesadaran manusia.

Namun demikian, harus diakui pula bahwa tak jarang umat beragama juga
bertanggung jawab atas timbulnya ateisme. Sebab, ``hanya di mana Tuhan
diwartakan dan dipercaya secara begitu radikal, maka di situ Dia bisa
juga dinegasi secara radikal.'' (Gerhard Ebeling) . Berdasarkan cara
penghayatannya, ateisme modern dapat dibagi ke dalam 2 kelompok:

\begin{enumerate}
\def\labelenumi{\arabic{enumi}.}
\tightlist
\item
  Ateisme Praktis: Pandangan ini merupakan sikap hidup indiferen dan
  tidak menghiraukan hal-hal religius. Seorang ateis praktis barangkali
  memang percaya sungguh pada Tuhan dan tidak mempersoalkan
  keberadaan-Nya, namun dalam praktik hidupnya ia melakukan segala
  sesuatu seolah-olah tidak ada Tuhan.
\item
  Ateisme Teoretis: Pandangan ini secara terang-terangan dan lugas
  menyangkal adanya Tuhan dan berusaha juga mempertanggungjawabkan
  keyakinannya dengan mengajukan argumentasi-argumentasi.
\end{enumerate}

Berdasarkan argumentasinya, Ateisme Teoretis dapat dibagi lagi ke dalam
4 kelompok:

\begin{enumerate}
\def\labelenumi{\arabic{enumi}.}
\tightlist
\item
  Ateisme Humanistik: Kelompok ini memandang adanya Tuhan sebagai
  ancaman dan penindasan terhadap manusia. Oleh karena itu, demi
  kemanusiaannya, manusia harus menolak adanya Tuhan. Contoh: Nietzsche,
  Sartre.
\item
  Ateisme Sosial-Politis: Kelompok ini menganggap iman kepada Tuhan
  hanyalah dampak dari ketidakberesan dalam masyarakat. Tuhan lalu
  dipandang sebagai gejala yang menunjuk pada suatu 'penyakit dalam
  masyarakat. Oleh karena itu, agar orang mau menghadapi penyakitnya dan
  menyembuhkannya, Tuhan harus ditolak. Contoh: Marx.
\item
  Ateisme Penderitaan: Kelompok ini menolak adanya Tuhan dengan bertitik
  tolak pada fakta bahwa manusia mengalami penderitaan di dunia ini.
  Contoh: Hume.
\item
  Ateisme Saintifik: Kelompok ini berpendapat bahwa kepercayaan kepada
  Tuhan tidak dapat didamaikan dengan perkembangan ilmu pengetahuan,
  khususnya sains dan teknologi.
\end{enumerate}

\hypertarget{jelaskan-pandangan-feuerbach-tentang-agama-sebagai-proyeksi}{%
\subsubsection{Jelaskan pandangan Feuerbach tentang agama sebagai
proyeksi!}\label{jelaskan-pandangan-feuerbach-tentang-agama-sebagai-proyeksi}}

Alam material adalah kenyataan terakhir, demikian pandangan Feuerbach.
Manusia menjadi sadar diri dengan membedakan dirinya dari dasar terakhir
itu. Hal itu berarti bahwa selain mampu membedakan dirinya dari Alam,
manusia juga mampu merefleksikan hakikatnya sendiri, yaitu rasio,
kehendak, dan perasaan. Rasio, kehendak, dan perasaan ini dapat
idealisasikan sampai tak terhingga, sehingga menjadi sesuatu yang
disebut ``Allah.''

Dalam agama Kristen, idealisasi itu jelas: Allah dipahami sebagai Yang
Mahatahu (rasio sempurna), Yang Mahabaik (kehendak sempurna), dan Kasih
(hati sempurna). Apa yang disebut sebagai hakikat Allah tidak lain
daripada hakikat manusia yang sudah dibersihkan dari macam-macam
keterbatasan atau ciri individualnya, lalu dianggap sebagai sebuah
kenyataan otonom yang berdiri di luar manusia. Dengan mengasalkan
hakikat Allah pada hakikat manusia, Feuerbach memandang teologi tak lain
daripada antropologi belaka.

Dalam hal inilah ta menyatakan bahwa Tuhan adalah proyeksi, yaitu bahwa
Allah tak lain daripada hakikat manusia yang diabsolutkan dan
diobjektifkan. Proyeksi ini terjadi karena manusia membenturkan hakikat
yang diidealkannya dengan fakta bahwa diri mereka serba terbatas. Hasil
proyeksinya itu lalu dipandang sebagai sebuah kenyataan otonom yang
berdiri di luar dirinya, dan memandang entitas otonom tersebut memandang
manusia sebagai objeknya. Akhirnya, mereka meletakkan dirinya lebih hina
dari hasil proyeksinya sendiri.

Feuerbach mengkritik semua proses ini yang membuat manusia justru
terasing dari dirinya sendiri: \emph{ia tidak lagi mengenali bahwa Allah
yang diagungkannya itu tak lain dari hakikat idealnya sendiri}. Walaupun
sebagai individu terbatas, Feuerbach berpendapat bahwa manusia sebagai
keseluruhan umat manusia (\emph{Gattung}) adalah hebat dan tak terbatas.
Oleh karena itu, Tuhan harus dicoret agar manusia terlepas dari
alienasinya dan kembali berusaha merengkuh hakikat yang diidealkannya.

Akhirnya, manusia harus menolak kepercayaan pada Allah yang mahakuasa,
mahabaik, maha-adil, dan mahatahu, agar ia sendiri bisa kuat, baik,
adil, dan berpengetahuan (\emph{Homo homini Deus}).

Ambivalensi agama menurut Feuerbach:

\begin{enumerate}
\def\labelenumi{\arabic{enumi}.}
\tightlist
\item
  Aspek Positif: Manusia mengenali dirinya yang ideal melalui pengenalan
  akan Tuhan.
\item
  Aspek Negatif: Manusia teralienasi dari sifat-sifatnya yang
  di-idealkan.
\end{enumerate}

\hypertarget{bagaimana-pendapat-marx-tentang-pemikiran-feuerbach}{%
\subsubsection{Bagaimana pendapat Marx tentang pemikiran
Feuerbach?}\label{bagaimana-pendapat-marx-tentang-pemikiran-feuerbach}}

Pada dasarnya Marx setuju atas pandangan Feuerbach bahwa manusia
mengasingkan diri dalam agama. Persoalan Tuhan sudah diatasi Feuerbach
dengan semboyannya ``homo homini Deus.'' Namun demikian, Marx
berpendapat bahwa Feuerbach masih terlalu abstrak Manusia yang
dipersoalkan Feuerbach yaitu Gattung--adalah suatu entitas yang sama
sekali tidak konkret. Menurut Marx, itu hanyalah mengganti nama pada
entitas yang sama: dari yang semula bernama ``Tuhan,'' Feuerbach
menggantikannya dengan nama ``Gattung.'' Itulah sebabnya, Marx menyebut
Feuerbach sebagai seorang ``ateis saleh.'' Marx lalu menuntut supaya
manusia dipandang sebagai sebuah realitas konkret. Pertanyaan mendasar
yang diajukan Marx adalah mengapa manusia mengasingkan dirinya dalam
agama?

Untuk menjawab pertanyaan ini, perlu diingat kembali sistem filsafat
Marx yang dikenal dengan nama ``materialisme sejarah.'' Dalam sistem
tersebut, Marx membagi realitas masyarakat ke dalam dua bagian: (1)
basis atau suprastruktur, yaitu kegiatan ekonomi; serta (2) bangunan
atas atau superstruktur, yaitu hukum, politik, ideologi, filsafat, seni,
agama. Menurut Marx, basis menentukan bangunan atas. Atas dasar sistem
ini, maka Marx memandang agama sebagai sebuah gejala sosial dengan
sifat-sifat berikut: Di satu sisi, agama merupakan tanda adanya
ketidakberesan sosial di tengah masyarakat. Di sisi lain, agama
meninabobokkan masyarakat sehingga mereka lari dari masalah sosial yang
mereka miliki.

\hypertarget{evaluasi-atas-pandangan-feuerbach}{%
\subsubsection{Evaluasi atas pandangan
Feuerbach}\label{evaluasi-atas-pandangan-feuerbach}}

Pandangan Feuerbach memiliki keunggulan dari banyaknya contoh kasus.
Dalam kenyataan, memang bukan hal yang baru bahwa orang beragama,
khususnya para pemimpinnya, mengklaim berbuat ini-itu atau memerintahkan
para pengikutnya untuk melakukan sesuatu atas nama Tuhan, padahal
sebenarnya semua itu merupakan pantulan dari kehendaknya sendiri (misal:
untuk berkuasa, untuk mendominasi, \emph{hidden needs}).

Tetapi, pemikiran Feuerbach tentang Homo Homini Deus dan Gattung dapat
dipertanyakan secara radikal: Manusia manakah bisa berfungsi sebagai
``Allah'' bagi manusia lain? Manakah ukuran bagi manusia? Lagipula,
perjalanan sejarah umat manusia justru menunjukkan yang sebaliknya:
bukankah manusia masa depan idaman Feuerbach malahan terbukti menjadi
monster bagi sesamanya (contoh: Perang Dunia I dan 2, Auschwitz,
Hiroshima-Nagasaki)?

Kekeliruan antara fungsi dengan hakikat Teori Proyeksi yang diajukan
Feuerbach memang menjelaskan bagaimana fungsi agama bagi manusia. Akan
tetapi, teori tersebut justru tidak menyentuh pertanyaan mendasar
mengenai hakikat agama, yaitu Tuhan yang disembah dalam agama itu.
Justru karena yang dibicarakan adalah fungsi agama, maka ateisme yang
diajukan Feuerbach ini sama sekali tidak menyentuh pertanyaan dasariah
mengenai apakah Allah itu pada dirinya sendiri ada atau tidak.

Menurut Feuerbach, proyeksi dilakukan karena hakikat ideal manusia
terbentur pada keterbatasannya. Pertanyaannya: dari manakah manusia
memperoleh konsep kesempurnaan yang tercermin dalam kata ``maha-'' yang
dialamatkannya pada Tuhan itu? Kalau manusia sanggup mengenal Tuhan
sebagai ``Yang Maha-'', padahal ia sendiri sama sekali tidak mempunyai
pengalaman inderawi tentang-Nya, maka nyatalah bahwa argumentasi
Feuerbach yang mau mengembalikan gejala agama pada soal psikologis dan
inderawi melulu memuat kontradiksi dalam dirinya sendiri.

\hypertarget{evaluasi-atas-pandangan-marx}{%
\subsubsection{Evaluasi atas pandangan
Marx}\label{evaluasi-atas-pandangan-marx}}

Pemikiran Marx mengenai penyalahgunaan agama Marx dapat didukung. Memang
benar bahwa agama dapat disalahgunakan. Alih-alih menghadapi
persoalannya di dunia, sering kali manusia justru lari kepada agama.
Pandangan naif mengenai nasib, misalnya, dapat digolongkan ke dalam
penyalahgunaan ini.

Kritik Marx mendorong agama untuk berkontribusi secara nyata dalam
kehidupan manusia. Kritik Marx bahwa agama adalah candu pada dasarnya
merupakan reaksi atas cara keberagamaan sebagian umat yang seolah cuci
tangan atas masalah dunia ini. Dengan beramal, misalnya, seorang
koruptor lalu merasa sudah membersihkan dirinya dan akan kembali
melakukan kejahatan yang sama. Lebih dari itu, agama seharusnya juga
terlibat dalam kehidupan sosial.

Fungsi dan hakikat dari Basis atau Bangunan Atas dalam pemikiran Marx
Keliru. Pada kenyataannya bisa ada relasi yang timbal balik.

\hypertarget{bahan-rujukan}{%
\subsection{Bahan Rujukan}\label{bahan-rujukan}}

\begin{enumerate}
\def\labelenumi{\arabic{enumi}.}
\tightlist
\item
  Franz Magnis-Suseno. \emph{Menalar Tuhan}, Yogyakarta: Kanisius, 2006.
\item
  S.P. Lili Tjahjadi. ``Ateisme Modern.'' Disampaikan sebagai salah satu
  materi mata kuliah Sejarah Pemikiran Modern di STF Driyarkara.
  (Sebagai Pembanding)
\end{enumerate}

\hypertarget{etika}{%
\chapter{Etika}\label{etika}}

Etika merupakan filsafat atau pemikiran kritis dan mendasar tentang
ajaran ajaran dan pandangan-pandangan moral. Etika bukan ajaran moral,
yaitu ajaran yang mengatakan bagaimana kita harus hidup. Etika adalah
ilmu yang mau mengerti mengapa kita kita harus mengikuti ajaran
tertentu, atau bagaimana kita dapat mengambil sikap yang
bertanggungjawab berhadapan dengan ajaran moral tertentu. Etika berusaha
mengerti menapa atau atas dasar apa kita harus hidup menurut norma-norma
tertentu. Singkatnya, etika adalah pemikiran sistematis tentang
moralitas. Berikut ini akan diuraikan dua teori etika modern yang amat
berpengaruh, dan juga akan diuraikan permasalahan seputar suara hati.

\hypertarget{etika-utilitarianisme-dan-deontologi}{%
\section{Etika Utilitarianisme dan
Deontologi}\label{etika-utilitarianisme-dan-deontologi}}

\begin{quote}
Etika Utilitarianisme dan Etika Deontologis Immanuel Kant merupakan dua
teori etika normatif modern yang amat berpengaruh. Etika Deontologis
Immanuel Kant dapat mengatasi kelemahan Etika Utilitarianisme dan lebih
mampu menjamin keberlakuan mutlak keadilan serta hak asasi manusia.
Selain itu, etika Kant juga memberi dasar yang kokoh bagi rasionalitas
dan objektivitas kesadaran moral. Kendati begitu, teori itu juga tidak
lepas dari beberapa kelemahan yang pantas dikritik.
\end{quote}

\hypertarget{utilitarianisme-klasik}{%
\subsection{Utilitarianisme Klasik}\label{utilitarianisme-klasik}}

Utilitarisme klasik, sebagaimana dikemukakan Jeremy Bentham (1748-1832)
dan John Stuart Mill (1806-1873), dapat diringkaskan dalam tiga
pernyataan berikut.

\begin{enumerate}
\def\labelenumi{\arabic{enumi}.}
\tightlist
\item
  Tindakan harus dinilai benar atau salah hanya demi akibat-akibatnya
  (\emph{consequences}).
\item
  Dalam mengukur akibat-akibatnya, satu-satunya yang penting hanyalah
  jumlah kebahagiaan atau ketidakbahagiaan yang dihasilkan.
\item
  Kesejahteraan atau kebahagiaan semua orang sama pentingnya.
\end{enumerate}

Bagi utilitarisme klasik, yang baik tidak berbeda dengan yang benar.
Tindakan yang benar adalah tindakan yang baik, dan yang baik itu adalah
kebahagiaan, sebagai satu-satunya yang diinginkan, sebagai tujuan akhir.
Utilitarisme klasik ini selanjutnya disebut utilitarisme-tindakan, yang
memiliki persoalan berkaitan dengan keadilan dan hak-hak individual.
Para pembela utilitarisme kemudian mengajukan utilitarisme-peraturan. Di
sini, tindakan-tindakan individual tidak lagi diadili dengan prinsip
utilitas, tetapi dengan menipertanyakan dulu perangkat aturan mana yang
paling haik menurut sudut pandang utilitarisme (kebahagiaan terbesar).

\hypertarget{etika-deontologi-immanuel-kant}{%
\subsection{Etika Deontologi Immanuel
Kant}\label{etika-deontologi-immanuel-kant}}

Etika Deontologi Immanuel Kant tidak menilai moralitas suatu tindakan
dari akibat-akibatnya, melainkan dari apa yang diwajibkan secara
kategoris. Prinsip ini disebut imperatif kategoris. Kant mengatakan,
``Bertindaklah hanya menurut kaidah dengan mana Anda dapat sekaligus
menghendaki supaya kaidah itu berlaku sebagai hukum universal.'' Prinsip
ini nierumuskan prosedur untuk memutuskan apakah suatu tindakan itu
secara moral diijinkan atau tidak. Ketika mempertimbangkan suatu
tindakan, Anda perlu bertanya aturan mana yang akan diikuti (sebagai
kaidah tindakan), lalu bertanya apakah Anda akan menerima aturan itu
untuk diikuti oleh setiap orang sepanjang waktu (sebagai hukum
universal). Jika jawabannya ``ya,'' aturan tersebut dapat diikuti dan
tindakan tersebut boleh dilakukan secara moral. Jika jawabannya
``tidak,'' aturan tersebut tidak boleh diikuti dan tindakan tersebut
secara moral tidak diperbolehkan.

Bagi Kant, menjadi pelaku moral berarti mengarahkan perilaku dengan
hukum-hukum universal, yaitu aturan-aturan moral yang berlaku tanpa
kekecualian dalam segala waktu, misalnya aturan yang melarang orang
berbohong. Prinsip universalitas ini adalah prinsip yang pertama.
Prinsip kedua terkait dengan martabat manusia yang harus dihormati.
Formulasinya adalah sebagai berikut. ``Bertindaklah sedemikian sehingga
engkau memperlakukan kemanusiaan, entah dalam dirimu sendiri atau orang
lain, selalu sebagai tujuan dan bukan hanya sebagai sarana''.

Etika Deontologis Immanuel Kant dapat mengatasi kelemahan Etika
Utilitarianisme dan lebih mampu menjamin keberlakuan mutlak keadilan
serta hak asasi manusia. Ada sesuatu yang tetap tidak boleh dilakukan,
entah apa pun alasannya. Misalnya, bahwa kita tidak boleh membunuh orang
yang tidak bersalah, meskipun itu demi kesejahteraan orang yang lebih
banyak Etika deontologis menunjukkan adanya hal yang mutlak yang tidak
dapat dibatalkan walaupun demi tujuan-tujuan yang baik. Menurut Kant,
nilai manusia mengatasi segala harga. Inilah kesimpulan objektif
mengenai tempat manusia dalam kerangka hal-hal atau barang-barang lain.
Ada dua alasan mengapa manusia mengatasi segala harga. Pertama, hanya
manusia yang memiliki keinginan dan tujuan. Hal-hal lain memiliki nilai
bagi manusia sejauh pada hubungannya dengan rencana-rencana manusia.
Kedua, manusia mempunyai nilai intrinsik, yaitu martabat, karena manusia
adalah pelaku rasional, artinya pelaku bebas yang mampu mengambil
keputusan untuk dirinya sendiri, menempatkan tujuan-tujuannya sendiri
dan menuntun perilakunya dengan akal budi. Karena nilai manusia
mengatasi segala harga inilah bagi Kant manusia harus diperlakukan
selalu sebagai tujuan, dan tidak pernah hanya sebagai sarana. Di sinilah
teori etika ini lebih mampu menjamin keberlakuan keadilan dan hak-hak
asasi manusia.

Gagasan Kant memberi dasar yang kokoh bagi rasionalitas dan objektivitas
kesadaran moral. Jika Anda dapat mengartikan suatu hal sebagai alasan
untuk suatu kasus, Anda juga harus menerimanya sebagai alasan dalam
kasus yang lain yang serupa dengan kasus yang pertama. Misalnya, pikiran
bahwa Anda tidak boleh membakar hutan karena hal itu akan merusak harta
orang dan membuat orang-orang terbunuh. Jika ada kasus lain di mana
harta rusak dan orang-orang terbunuh, Anda harus menerima hal itu
sebagai alasan untuk tidnakan dalam kasus itu jugn. Tidaklah baik
mengatakan bahwa Anda menerima alasan-alasan itu pada suatu saaat,
tetapi tidak untuk seterusnya. Tidaklah baik mengatakan bahwa
orang-orang lain harus menghormati Anda, tetapi Anda sendiri tidak harus
menghormati orang-orang lain. Alasan-alasan moral yang sahih mengikat
semua orang pada setiap waktu. Inilah tuntutan untuk konsistensi. Kant
benar dalam anggapannya bahwa tak seorang pun yang rasional yang mampu
menyangkalnya.

Meskipun memberikan dasar yang kokoh bagi rasionalitas dan objektivitas
kesadaran moral, teori etika Kant tidak lepas dari beberapa kelemahan
yang pantas dikritik. Pertama, gagasan bahwa aturan moral ditaati tanpa
pengecualian sulit dipertahankan. Kita dapat membayangkan bahwa ada
lingkungan tertentu di mana mengikuti aturan justru akan mengakibatkan
hasil yang mengerikan. Misalnya, aturan dan kewajiban untuk tidak
berbohong, Ilustrasi yang sering dikemukakan untuk mengkritisi aturan
atau kewajiban mutlak untuk tidak berbohong adalah mengenai ``Si
Pembunuh yang Bertanya.'' Apakah kita harus memberitahukan kepadanya apa
yang benar, padahal dengan berkata jujur kita akan membuat seseorang
terbunuh? Kedua, adanya konflik antara aturan-aturan. Misalnya, jika
kedua aturan ini, yaitu ``berbohong itu salah'' dan ``mengijinkan
pembunihan atas orang-orang yang tak bersalah juga salah,'' diterima
secara mutlak, mana yang harus ditaati?

\hypertarget{pokok-pokok-lain-6}{%
\subsection{Pokok-pokok Lain}\label{pokok-pokok-lain-6}}

\hypertarget{apa-yang-dimaksud-dengan-teori-etika-normatif}{%
\subsubsection{Apa yang dimaksud dengan teori Etika
Normatif?}\label{apa-yang-dimaksud-dengan-teori-etika-normatif}}

Masalah pokok dalam teori etika normatif adalah bagaimana kita
mempertanggungjawabkan secara rasional penilaian dan putusan moral kita.
Pada dasarnya, penilaian dan putusan moral dapat dipertanggungjawabkan
secara rasional apabila didasarkan atas prinsip dan norma moral yang
sehat Maka dari itu, permasalahannya adalah, manakah prinsip dan norma
moral yang sehat yang dapat dijadikan acuan dan dasar pertanggungjawaban
tersebut? Inilah pertanyaan pokok yang dihadapi oleh etika normatif.
Hanya saja, ternyata tidak ada kesepahaman mengenai bagaimana menjawab
pertanyaan tersebut. Akibatnya, ada beberapa teori etika normatif dalam
sejarah filsafat Barat. Secara garis besar, teori-teori etika normatif
dapat dikelompokkan ke dalam dua kategori:

\begin{enumerate}
\def\labelenumi{\arabic{enumi}.}
\tightlist
\item
  Teori-teori Konsekuensialis: Kelompok teori ini menilai baik-buruk dan
  benar-salahnya tindakan manusia berdasarkan dari konsekuensi atau
  akibatnya, yaitu apakah perbuatan tersebut secara keseluruhan membawa
  akibat baik lebih banyak daripada akibat buruknya ataukah sebaliknya.
  Kelompok teori ini juga disebut sebagai teori-teori etika teleologis
  karena merujuk pada tujuan dari tindakan yang bersangkutan. Contoh:
  etika egoisme, eudaimonisme, dan utilitarisme.
\item
  Teori-teori Non-Konsekuensialis: Kelompok teori ini menilai baik-buruk
  dan benar-salahnya tindakan manusia. tanpa memperhatikan konsekuensi
  dari tindakan tersebut, melainkan berdasarkan sesuai-tidaknya suatu
  tindakan dengan hukum atau standar moral. Kelompok teori ini juga
  dikenal dengan teori-teori etika deontologis karena menekankan konsep
  kewajiban (\emph{deon}) moral yang harus ditaati oleh manusia sebagai
  makhluk rasional. Contoh: etika deontologis Kant.
\end{enumerate}

\hypertarget{apa-yang-dimaksud-dengan-teori-etika-utilitarisme}{%
\subsubsection{Apa yang dimaksud dengan teori Etika
Utilitarisme?}\label{apa-yang-dimaksud-dengan-teori-etika-utilitarisme}}

Etika utilitarisme, yang dikembangkan oleh Jeremy Bentham (1748-1832)
dan J.S. Mill (1806-1873), adalah teori etika yang menekankan prinsip
manfaat atau kegunaan sebagai prinsip moral yang paling dasariah. Suatu
tindakan dilnilai berguna kalau akibat secara keseluruhan dari tindakan
tersebut dengan memperhitungkan semua pihak yang terlibat serta tidak
membeda-bedakan membawa akibat baik berupa keunduntan atau kebahagiaan
yang semakin besar bagi semakin banyak orang. Semboyannya adalah
``\emph{The greatest good to the greatest number}.''

Ada dua macam etika utilitarisme:

\begin{enumerate}
\def\labelenumi{\arabic{enumi}.}
\tightlist
\item
  Utilitarisme Tindakan: ``Bertindaklah sedemikian rupa sehingga setiap
  tindakanmu itu menghasilkan akibat-akibat baik yang lebih besar di
  dunia daripada akibat buruknya.'' Utilitarianisme tindakan sulit
  diterima, karena mudah disalahgunakan untuk membenarkan tindakan yang
  melanggar hukum.
\item
  Utilitarisme Peraturan: ``Bertindaklah selalu sesuai dengan
  kidah-kidah yang penerapannya menghasilkan akibat baik yang lebih
  besar di dunia ini daripada akibat buruknya.'' Utilitarianisme
  peraturan lebih dapat diterima.
\end{enumerate}

Kelemahan Utilitarisme adalah sulitnya menentukan nilai kebaikan dari
suatu akibat. Bentham menanggapi kelemahan ini dengan menjabarkan 7
dimensi \emph{hedonic calculus} (penilaian kuantitatif), sementara Mill
menambahkan unsur kualitas. Selain itu, penerapan Utilitarianisme
mungkin bertentangan dengan prinsip keadilan.

\hypertarget{apa-yang-dimaksud-dengan-etika-deontologis}{%
\subsubsection{Apa yang dimaksud dengan Etika
Deontologis?}\label{apa-yang-dimaksud-dengan-etika-deontologis}}

Etika deontologis, yang dipelopori oleh Immanuel Kant (1724-1804),
ajalah teori filsafat moral yang mengajarkan bahwa sebuah tindakan itu
benar kalau tindakan tersebut selaras dengan prinsip kewajiban yang
relevan tuntuknya. Para penganut teori ini meyakini bahwa norma moral
itu mengikat secara inutlak dan tidak tergantung pada apakah akibat dari
ketaatan tersebut menguntungkan atau tidak. Dalam buku Kritik atas Rasio
Praktis, Kant berpendapat bahwa nilai moral (baik-buruknya tindakan)
tidak terletak pada hasil tindakan, tetapi pada sesuatu dalam kesadaran
subjek moral yang disebutnya ``maksim.'' Maksim berbeda dengan asas:
asas-asas bersifat objektif dalam rasio praktis makhluk rasional,
sedangkan maksim bersifat subjektif dalam kehendak.

Ajaran pokok teori etika deontologis Kant:

\begin{enumerate}
\def\labelenumi{\arabic{enumi}.}
\tightlist
\item
  Kemurnian motivasi: ``Bertindaklah bukan hanya sesuai dengan kewajiban
  moral, melainkan juga demi kewajiban moral tersebut.'' Yang
  sungguh-sungguh baik adalah kehendak baik.
\item
  Imperatif Kategoris: ``Bertindaklah seolah-olah maksim tindakan Anda
  melalui keinginan Anda sendiri dapat menjadi sebuah Hukum Alam yang
  Universal.''
\item
  Hormat terhadap pribadi: ``Bertindaklah sedemikian rupa sehingga Anda
  selalu memperlakukan umat manusia -entah di dalam pribadi Anda maupun
  di dalam pribadi setiap orang lain-sebagai tujuan, bukan sebagai
  sarana.''
\end{enumerate}

Kekuatan teori etika deontologis:

\begin{enumerate}
\def\labelenumi{\arabic{enumi}.}
\tightlist
\item
  Memberikan dasar kokoh bagi rasionalitas dan objektivitas kesadaran
  moral Kant menekankan bahwa prinsip moralitas bisa diturunkan secara
  apriori dari akalbudi murni dan tidak ditentukan oleh objek tindakan,
  akibat tindakan, maupun kepentingan-kepentingan subjek pelaku.
  Penilaian mora, menurut Kant, bukanlah perkara selera atau perasaan
  bepala, melainkan berdasarkan suatu prinsip yang masuk akal Oleh
  karena itu, pertanggungjawaban moral selalu dapat diuji secara
  intersubjektif.
\item
  Memberikan tolak ukur yang perlu dan penting untuk menilai moralitas
  suatu tindakan, yakni prinsip universalitas.
\item
  Menjamin otonomi dan keluhuran martabat manusia Etika deontologis Kant
  menekankan peran manusia sebagai tujuan setiap tindakan dan bukan
  sebagai sarana.
\end{enumerate}

Kelemahan teori etika deontologis:

\begin{enumerate}
\def\labelenumi{\arabic{enumi}.}
\tightlist
\item
  Tidak ada ruang bagi dilema moral dan tidak memberi solusi ketika hal
  itu terjadi. (Solusi W.D. Ross: saat terjadi dilema moral, norma moral
  yang berlaku adalah \emph{prima facie}.)
\item
  Kemutlakan norma tanpa sama sekali mengindahkan akibat tindakan, sulit
  diterima.
\item
  Imperatif Kategoris Kant melulu formal sehingga tidak membantu
  mengerti kewajiban mana yang secara kontkret mengikat seorang pelaku
  moral.
\end{enumerate}

\hypertarget{suara-hati}{%
\section{Suara Hati}\label{suara-hati}}

\begin{quote}
Salah satu pokok bahasan penting dalam Etika Umum/Dasar adalah tentang
suara hati. Suara hati tidak sama dengan apa yang oleh Sigmund Freud
disebut \emph{super ego}. Meski kita tidak pernah boleh bertindak
melawan suara hati, namun keputusan suara hati harus dapat kita
pertanggungjawabkan dengan mengacu pada norma-norma objektif. Suara hati
kita juga perlu terus dididik atau dikembangkan, baik secara kognitif,
afektif maupun konatif.
\end{quote}

Suara hati adalah kesadaran tentang apa yang menjadi kewajiban manusia
berhadapan dengan masalah konkret yang dihadapinya. Merujuk pada
Magnis-Suseno, ada tiga lembaga normatif yang mengajukan norma-norma
(apa yang harus dianggap baik atau tidak) kepada kita:

\begin{enumerate}
\def\labelenumi{\arabic{enumi}.}
\tightlist
\item
  Masyarakat, yaitu semua orang dan lembaga yang berpengaruh pada hidup
  kita: keluarga/orang tua, sekolah/para guru, agama, tempat kerja,
  negara.
\item
  Superego, yaitu perasaan moral spontan sebagai hasil integrasi dan
  pembatinan perintah-perintah, nilai-nilai dan larangan.
\item
  Ideologi, yaitu segala macam ajaran tentang makna kehidupan, tentang
  nilai-nilai dasar dan tentang bagaimana manusia barus hidup dan
  bertindak.
\end{enumerate}

Suara hati berbeda dengan ketiga lembaga normatif ini. Berhadapan dengan
pendapat masyarakat dan tuntutan ideologi, manusia menjadi sadar, bahwa
ia tidak boleh mengikuti pendapat moral mereka begitu saja, melainkan
harus memastikan sendiri apa yang sebenarnya merupakan kewajiban dalam
situasi yang sedang dihadapinya. Secara moral, akhirnya kita sendirilah
yang harus memutuskan apa yang akan kita lakukan. Di sinilah juga suara
hati berbeda dengan superego. Akan tetapi, ternyata dua istilah ini
sering disamakan, baik dalam masyarakat maupun oleh sementara alirani
psikologi. Bagaimana perbedaan ini dijelaskan?

\hypertarget{suara-hati-vs.-superego}{%
\subsection{Suara Hati vs.~Superego}\label{suara-hati-vs.-superego}}

Istilah superego diperkenalkan oleh Sigmund Freud (1856-1939) yang
mengidentifikasikan unsur-unsur utama dalam kesadaran manusia sebagai
Id, Ego dan Superego. Id adalah semua kecenderungan irrasional yang
muncul dari kedalaman diri kita dan menghadapkan kita dengan
tuntutan-tuntutan mereka, yaitu segala dorongan, nafsu, naluri dan
insting. Superego adalah perasaan bersalah yang kita rasakan apabila
kita melakukan hal-hal yang terlarang. Ego adalah ``aku'' yang sadar,
subjektivitas kita, pusat kesadaran dan keinginan kita. Ego ini adalah
diri kita sendiri, yang selalu bertindak berhadapan dengan dorongan Id
dengan pengawasan dari Superego.

Dari penjelasan di atas, superego lebih merupakan perasaan bersalah
(yang sifatnya otomatis, tanpa pertimbangan terlebih dahulu). Superego
mirip sensor atau pengawas bagi tindakan-tindakan dan kesadaran kita
sebagai Ego. Bentuk pengawasan atau sensor ini tergantung pada
pendidikan seseorang semasa kecilnya.

Suara hati, di pihak lain, tidak melulu soal rasa bersalah, tetapi suatu
kesadaran seseorang untuk memilih yang sesuai dengan tanggung jawabnya.
Sementara superego menghendaki kita memilih hal yang membebaskan kita
dari rasa bersalah, dorongan suara hati justru seringkali memunculkan
rasa bersalah, yaitu ketika kita melakukan hal yang kita anggap benar
tetapi merugikan orang lain. Kesadaran moral bukan semata-mata perasaan,
tetapi suatu pengertian yang sifatnya objektif dan bertanggungjawab.
Berbeda dengan suara hati, Superego tidak mempedulikan tepat tidaknya
suatu tindakan dari sudut tanggung jawab ini.

Jadi, keputusan suara hati harus dapat dipertanggungjawabkan dengan
mengacu pada norma-norma objektif. Mengambil contoh diperbolehkan atau
tidak seorang siswi SMA yang hamil di luar nikah untuk menggugurkan
kandungannya, ada kebenarar, objektif di situ. Mana yang benar dari dua
pilihan, yaitu boleh atau tidak, dapat diperiksa dan diperdebatkan
dengan argumentasi yang objektif. Masing-masing pendapat harus memiliki
pertanggungjawaban rasional. Maka, yang dituntut dari suatu keputusan
suara hati adalah rasionalitasnya (tanpa harus jatuh ke dalam
rasionalisme yang menuntut pembuktian akan segala sesuatunya), yaitu
bahwa keputusan moral tersebut harus kita buka terhadap tantangan dan
sangkalan dan harus didukung oleh argumen-argumen yang objektif. Di
sinilah diperlukan informasi, pertimbangan yang relevan, termasuk juga
pendapat pihak lain.

Dari kenyataan bahwa suara hati itu terkait dengan situasi konkret dan
melibatkan pertimbangan-pertimbangan rasional manusia, suara hati
(meskipun berlaku universal-maksudnya hanya satu pendapat yang benar dan
mesti diberlakukan, universalitas dalam pemahaman Kant) tetap dapat
keliru. Di sinilah suara hati perlu dididik atau dikembangkan dari tiga
sudut, yaitu secara kognitif, afektif, dan konatif:

\begin{enumerate}
\def\labelenumi{\arabic{enumi}.}
\tightlist
\item
  Dalam dimensi Kognitif, pendidikan suara hati akan melibatkan usaha
  untuk bersedia terus menerus belajar guna meningkatkan pengetahuan dan
  pengertian moral kita. Untuk ini diperlukan sikap terbuka terhadap
  macam-macam pertimbangan serta kemungkinan perlunya perubahan
  padangan.
\item
  Berkenaan dengan dimensi Afektif, pendidikan suara hati bermaksud
  menumbuhkan citarasa moral atau kepekaan hati terhadap apa yang memang
  baik atau secara objektif bernilai, apa yang pantas dicita-citakan
  dalam hidup dan apa yang jahat dan perlu dihindari.
\item
  Dalam dimensi Konatif, pendidikan moral bermaksud membangun kehendak
  atau tekat moral. Ada kemungkinan bahwa sikap dan kesadaran moral kita
  dalam situasi konkret menjadi bengkok (tidak lagi tepat), bukan karena
  pengetahuan dan pemahaman kita kurang atau keliru, tetapi karena
  kehendak kita kurang kuat (istilah Aristoteles: \emph{akrasia},
  artinya kelemahan kehendak). Kehendak yang kuat dalam moralitas ini
  dapat dibangun dengan melatihkannya.
\end{enumerate}

\hypertarget{pokok-pokok-lain-7}{%
\subsection{Pokok-pokok Lain}\label{pokok-pokok-lain-7}}

\hypertarget{apa-yang-dimaksud-dengan-etika-dasar-atau-etika-umum}{%
\subsubsection{Apa yang dimaksud dengan Etika Dasar atau Etika
Umum?}\label{apa-yang-dimaksud-dengan-etika-dasar-atau-etika-umum}}

Etika berasal dari bahasa Yunani ``ethos,'' yang secara harafiah berarti
``adat kebiasaan'', ``watak,'' atau ``kelakuan manusia.'' Dalam hidup
sehari-hari, setidaknya ada 3 arti kata etika:

\begin{enumerate}
\def\labelenumi{\arabic{enumi}.}
\tightlist
\item
  Etika dalam arti sistem nilai dan norma-norma moral yang menjadi
  pegangan hidup atau pedoman penilaian baik-buruknya perilaku manusia,
  baik secara individual maupun komunal. Contoh: ``Etika Jawa,''Etika
  Protestan" (Max Weber).
\item
  Etika dalam arti kode etik, yaitu sekumpulan norma dan nilai moral
  yang wajib diperhatikan oleh pemegang profesi tertentu. Contoh:
  ``etika kedokteran'', ``etika jurnalistik''.
\item
  Etika dalam arti ilmu yang melakukan refleksi kritis dan sistematis
  tentang moralitas. Arti yang ketiga inilah yang digunakan untuk
  merujuk pada etika sebagai cabang ilmu filsafat. Walaupun secara
  etimologis memiliki makna yang sama, terdapat perbedaan pemaknaan atas
  Istilah ``etika'' dan ``moral''. ``Etika'' digunakan untuk menyebut
  ilmu dan prinsip-prinsip dasar penilaian baik-buruknya perilaku
  manusia sebagai manusia; sedangkan ``moral'' digunakan untuk menyebut
  aturan dan norma yang lebih konkret bagi penilaian tersebut.
\end{enumerate}

Dengan begitu, dapat dibedakan antara objek material dan objek formal
dari Etika. Objek material etika adalah tingkah laku atau tindakan
manusia sebagai manusia (\emph{actus humanus}, bukan \emph{actus
hominis}), dan objek formal etika adalah segi baik-buruk atau
benar-salahnya tindakan tersebut berdasarkan norma moral.

Dalam Etika, ada tiga bentuk pendekatan:

\begin{enumerate}
\def\labelenumi{\arabic{enumi}.}
\tightlist
\item
  Pendekatan Deskriptif, yaitu memaparkan apa yg secara faktual terjadi.
\item
  Pendekatan Normatif-Preskriptif, yaitu membahas apa yang seharusnya
  dilakukan.
\item
  Pendekatan Analitis-Metaetis, yaitu menganalisa
  terminologi-terminologi yang dipakai dalam diskursus moral.
\end{enumerate}

Etika memiliki dua cabang besar:

\begin{enumerate}
\def\labelenumi{\arabic{enumi}.}
\tightlist
\item
  Etika Umum/Dasar: menyajikan beberapa pengertian dasar dan mengkaji
  beberapa permasalahan pokok dalam filsafat moral, seperti suara hati,
  kebebasan-tanggung jawab, teori etika normatif, serta relativisme
  moral.
\item
  Etika Khusus: membahas beberapa permasalahan moral dalam bidang bidang
  khusus, seperti etika sosial, etika biomedis, etika bisnis, etika
  lingkungan hidup.
\end{enumerate}

\hypertarget{apa-yang-dimaksud-dengan-suara-hati-apa-bedanya-dengan-superego-freudian}{%
\subsubsection{Apa yang dimaksud dengan Suara Hati? Apa bedanya dengan
Superego
Freudian?}\label{apa-yang-dimaksud-dengan-suara-hati-apa-bedanya-dengan-superego-freudian}}

Suara hati (\emph{conscientia}, Lat) secara etimologis berarti
``mengetahui bersama'' atau ``turut mengetahui.'' Dalam hidup
sehari-hari, suara hati merupakan instansi yang turut mengetahui atau
menjadi saksi sekaligus hakim bagi perbuatan-perbuatan moral manusia.
Secara ringkas, dapat dirumuskan bahwa suara hati adalah kesadaran moral
manusia dalam situasi konkret. Sebagai sebuah kesadaran, suara hati
mengandaikan adanya pertimbangan akalbudi dan bukan sekedar ungkapan
perasaan spontan belaka. Suara hati menjadi pedoman atau pegangan moral
manusia saat ia harus mengambil keputusan untuk bertindak.

Sementara itu, Superego adalah internalisasi atas perintah-perintah,
larangan-larangan, dan nilai-nilai moral yang menurut Freud berasal dari
didikan orang tua (atau orang lain di sekitar kita). Dengan demikian,
batin kita sendiri mengumandangkan tuntutan-tuntutan masyarakat kepada
kita, pertama-tama tuntutan orang tua, kemudian lembaga-lembaga
masyarakat lainnya. Dalam ilmu psikologi, superego dipahami sebagai
perasaan moral spontan di mana ia menyatakan diri dalam perasaan malu
dan bersalah yang muncul secara otomatis ketika seseorang melanggar
norma-norma yang telah dibatinkan tersebut. Yang khas bagi superego
adalah bahwa perasaan-perasaan itu juga muncul apabila tidak ada orang
lain yang menyaksikan pelanggaran kita.

Secara singkat, perbedaan Superego dan Suara Hati adalah:

\begin{enumerate}
\def\labelenumi{\alph{enumi}.}
\tightlist
\item
  Superego bersifat tidak sadar dan otomatis. Suara Hati bersifat sadar.
\item
  Superego bekerja di ranah perasaan atau emosi. Suara Hati bekerja di
  ranah rasionalitas.
\item
  Tuntutan Superego berasal dari luar (keluarga, masyarakat) dan
  menindas subjek. Suara Hati berasal dari dalam diri dan mendorong
  subjek untuk bertindak.
\end{enumerate}

\hypertarget{bagaimana-kemutlakan-suara-hati-dapat-dijelaskan}{%
\subsubsection{Bagaimana kemutlakan suara hati dapat
dijelaskan?}\label{bagaimana-kemutlakan-suara-hati-dapat-dijelaskan}}

Suara hati pertama-tama bersifat mutlak, artinya apa yang ditegaskan
olehnya tetap tidak dapat ditawar-tawar keberlakukannya. Manusia mungkin
dapat mengabaikannya, namun teguran suara hati tetap tidak dapat
dibungkam oleh pertimbangan-pertimbangan yang sepintas dirasa
menguntungkan. Selain itu, kemutlakan suara hati juga tampak dari
kenyataan bahwa teguran yang diutarakannya tidak dapat diajak berdamai
melalui rasionalisasi si subjek atas tindakannya. Namun demikian, suara
hati tidak serta-merta identik dengan suara Tuhan.

Di satu sisi, kemutlakan suara hati memang menjadi petunjuk adanya Tuhan
sebagai Yang Mutlak. Sebab, apa yang mutlak tidak dapat berasal dari
manusia yang secara ontologis bersifat kontingen. Namun di sisi lain,
suara hati diungkapkan dalam kesadaran manusia yang adalah pengada
terbatas sehingga dapat keliru. Yang mutlak dalam suara hati adalah
tuntutannya, sedangkan isi kewajibannya tidaklah secara mutlak benar.

\hypertarget{bagaimana-suara-hati-dapat-dipertanggungjawabkan-dengan-cara-apa-suara-hati-dapat-dididik}{%
\subsubsection{Bagaimana suara hati dapat dipertanggungjawabkan? Dengan
cara apa suara hati dapat
dididik?}\label{bagaimana-suara-hati-dapat-dipertanggungjawabkan-dengan-cara-apa-suara-hati-dapat-dididik}}

Karena suara hati mengandaikan akal budi, maka ia juga bersifat
rasional. Penilaian yang dilakukan oleh suara hati selalu didasarkan
pada argumen-argumen yang dapat dipertanggungjawabkan secara rasional.
Dengan pernyataan tersebut, maka di sini hendak ditolak pandangan
emotivisme moral yang menyatakan bahwa penilaian moral pada hakikatnya
merupakan masalah perasaan belaka sehingga tidak dapat ditentukan
benar-salahnya secara objektif. Oleh karena itu, suara hati juga dapat
diuji kebenarannya secara intersubjektif karena didasarkan atas prinsip
yang berlaku umum.

Pendidikan suara hati harus mencakup tiga dimensi:

\begin{enumerate}
\def\labelenumi{\arabic{enumi}.}
\tightlist
\item
  Kognitif: dalam bentuk belajar, menambah pengetatuan, memperluas
  cakrawaia, supaya berpikiran terbuka, dan menjauhkan diri dari
  dogmatisme.
\item
  Afektif: dalam bentuk melatih kepekaan hati, pemberian teladan,
  kisah-kisah mengenai tokoh-tokoh bermoral.
\item
  Konatif: dalam bentuk pembiasaan atau mati raga.
\end{enumerate}

\hypertarget{bahan-rujukan-1}{%
\subsection{Bahan Rujukan}\label{bahan-rujukan-1}}

\begin{enumerate}
\def\labelenumi{\arabic{enumi}.}
\tightlist
\item
  Franz Magnis-Suseno. \emph{Etika Dasar: Masalah-masalah Pokok
  Filsafat} \emph{Moral}. Yogyakarta: Kanisius, 1987.
\item
  James Rachels. \emph{Filsafat Moral}. Penerj. A. Sudiarja. Yogyakarta:
  Kanisius, 1994.
\item
  J. Sudarminta. \emph{Etika Umum: Kajian tentang Beberapa Masalah
  Pokok} \emph{dan Teori Etika Normatif}. Jakarta: Pusat Penelitian dan
  Pengabdian Kepada Masyarakat STF Driyarkara, 2010.
\end{enumerate}

\hypertarget{metafisika}{%
\chapter{Metafisika}\label{metafisika}}

Metafisika pada umumnya membicarakan tentang ``ada sebagaimana ada-nya
sendiri'' (\emph{being qua being}). Metafisika tidak tertarik untuk
membicarakan sudut pandang sudut pandang partikular, dan mau
mempertanyakan dan membahas ``ada'' itu sendiri. Penjelasan Anton Bakker
berikut mungkin menambah pemahaman mengenai metafisika yang dipahami
secara umum. Nama ``Metafisika'' mencakup pemikiran dan refleksi
filosofis mengenai kenyataan yang secara mutlak paling mendalam dan
paling ultima. Metafisika bermaksud menyatukan seluruh kenyataan dalam
visi menyeluruh, menurut intinya yang paling mutlak. Maka, dapat
dikatakan bahwa metafisika adalah usaha pencarian rasional terhadap
landasan terakhir dari realitas. Dalam pengertian ini pula, metafisika
kemudian mengalami krisisnya.

\hypertarget{krisis-metafisika}{%
\section{Krisis Metafisika}\label{krisis-metafisika}}

\begin{quote}
Zaman kita sering dikatakan sebagai era posmetafisis. Ada dua pemikiran
besar yang bisa ditengarai sebagai penanda krisis yang dialami
Metafisika: pertama, teori tentang tiga hukum perkembangan masyarakat
dari Auguste Comte (di abad 19), dan kedua, kritik Heidegger atas
Metafisika sebagai \emph{onto-theo-logi}.
\end{quote}

Para filsuf postmodernis seperti Jean Francois Lyotard, Jacques Derrida
dan Michel Foucoult mengumandangkan bahwa tidak ada yang disebut sebagai
kebenaran fundamental (persis yang dicari-cari oleh Metafisika).
Pemikiran yang mau mencari landasan pokok yang logis dari kenyataan tak
lebih dari fondasionalisme atau representasionalisme. Kebenaran itu
subjektif, tergantung pada subjek, dan tidak pernah mutlak. Manusialah
yang mengkonstruksi kebenaran oleh karena keyakinan bahwa dirinya adalah
subjek yang ada di luar dunia. Postmodernisme menolak narasi-narasi
besar yang termuat dalam filsafat metafisis, filsafat sejarah, atau
segala usaha yang berpretensi untuk sampai pada pengetahuan yang
universal. Ujung dari penolakan terhadap kebenaran mutlak yang menjadi
obsesi modernisme adalah Relativisme. Karena itu, zaman kita juga sering
dikatakan sebagai era \emph{post-metafisis}.

Di pihak lain, meskipun tetap percaya pada karakter rasional manusia,
Eric Weil menolak metafisika dari segi sikap dogmatisnya yang
melanggengkan kekerasan. Ini karena metafisika membicarakan tentang
kebenaran tertinggi---sesuatu yang mustahil dicapai---yang jika diyakini
akan secara otomatis mengeliminasi hal-hal lain bertentangan dengan
kebenaran terakhir yang mereka temukan. Bagaimana bisa kebenaran
tertinggi dapat dicapai jika subjek filsafat adalah manusia yang
historis dan dalam ruang-waktu yang terbatas? Menurut Well, ``\emph{the
ultimate truth}'' yang menjadi obsesi metafisika tidak lagi penting atau
relevan untuk dicari. Yang perlu ditangani adalah situasi ``absennya
kebenaran atau penolakan terhadap kebenaran.'' Sebagaimana kebenaran
yang ultima melahirkan kekerasan, absennya kebenaran pun melahirkan hal
yang serupa, yaitu kekerasan juga.

Dengan pandangan-pandangan demikian ini, zaman ini menjadi genderang
kematian-atau setidak-tidaknya, krisis bagi metafisika yang memiliki
hasrat untuk menemukan landasan terakhir dari realitas. Ketika
metafisika mati, ada dua aliran yang lahir.

Pertama, pandangan bahwa rasio sepenuhnya hilang. Tidak ada konsep,
tidak ada rasio, yang ada hanyalah kebutuhan. Percaya Tuhan berarti
``butuh'' percaya pada Tuhan. Inilah jalur yang dibangun Nietzsche
(\emph{Will to Power}), Husserl, dan Feidegger, yang diteruskan dalam
pandangan-pandangan postmodernisme. Mereka berpendapat bahwa kekayaan
objek tidak dapat direduksi oleh konsep-konsep.

Kedua, pandangan bahwa rasio yang hilang hanyalah rasio instrumental
saja. Inilah pandangan mereka yang mengembangkan mazhab Frankfurt
(Adorno, Marcuse, Horkheimer), dan juga Habermas dan Eric Weil. Habermas
yang masih percaya dengan rasio, misalnya dengan melihat zaman ini
sebagai zaman postmetafisis. la menyadari bahwa ilmu-ilmu alam dan
sosial sifatnya mendominasi. Ilmu-ilmu alam, misalnya, mendominasi
bidang masing-masing, entah itu fisika, biologi molekuler, komputer,
atau berbagai teknologi. Di sinilah peran ilmu-ilmu kritis (filsafat)
yang menawarkan emansipasi, yaitu menyadarkan manusia akan
dominasi-doniinasi yang tengah terjadi. Dimensi rasio yang memahami
dominasi-dominasi ini tidak mati. Inilah yang disebut Habermas sebagai
ratio komunikatif.

Selain kedua pandangan di atas, ada dua pemikiran besar yang ditengarai
sebagai penanda krisis yang dialami metafisika: Pertama, teori tentang
tiga hukum perkembangan masyarakat dari Auguste Comte, dan kedua kritik
\emph{onto-teo-logis} Heidegger terhadap Metafisika.

\hypertarget{auguste-comte-tiga-hukum-perkembangan-masyarakat}{%
\subsection{Auguste Comte: Tiga Hukum Perkembangan
Masyarakat}\label{auguste-comte-tiga-hukum-perkembangan-masyarakat}}

Menurut Comte, ada tiga tahap perkembangan masyarakat, yaitu (i) Tahap
kanak-kanak, di mana masyarakat berada dalam kondisi \emph{teologis}
(fiktif), (ii) tahap usia muda, masyarakat dalam kondisi
\emph{metafisis} (abstrak), dan (iii) tahap kematangan masyarakat, yaitu
ada dalam kondisi \emph{positif} (saintifik ilmiah). Ketiga tahap itu
merujuk pada proses perkembangan `roh manusia' (cara-cara manusia
``berfilsafat'') yang menurut Compte selalu melewati tiga tahapan:
metode teologis, metode metafisis dan akhirnya metode positif. Hukum
tiga tahap ini menunjukkan perkembangan bukan hanya pemikiran manusia
individu, tetapi juga sejarah umat manusia pada umumnya.

\begin{enumerate}
\def\labelenumi{\arabic{enumi}.}
\tightlist
\item
  \emph{Kondisi Teologis} merujuk pada situasi masyarakat yang ditemukan
  sejak awal sampai dengan Abad ke-12. Dalam masyarakat teologis,
  kehidupan dilandaskan pada kepercayaan-kepercayaan religius yang
  berkembang dari tahap fetisisme, politeisme sampai ke monoteisme. Kaum
  elit pemimpin masyarakat adalah \emph{`Man of God'}, manusia-manusia
  pilihan, atau wakil Tuhan. Dalam kondisi ini, orang percaya bahwa
  fenomen di dunia ini adalah hasil tindakan suatu entitas yang
  supranatural. Kalau ada bencana, orang lalu beranggapan bahwa hal itu
  karena manusia ``dihukum Allah akibat dosa-dosanya.'' Kondisi teologis
  tidak merujuk pada sistem teologi Abad Pertengahan, melainkan merujuk
  pada sebuah cara menjelaskan fenomen-fenomen.
\item
  \emph{Kondisi Metafisis} merujuk pada periode Abad ke-14 sampai Abad
  ke-18. Ini merupakan tahap peralihan. Dalam masyarakat periode ini,
  tumbuh kritisisme yang akan menghancurkan hierarki. Pencarian
  ``\emph{causa}'' yang pada kondisi teologis berakhir pada dewa-dewi
  atau Tuhan, dalam kondisi metafisis diperingkas menjadi beberapa
  prinsip abstrak saja. Kejadian atau fakta sosial diterangkan secara
  abstrak dan dicarikan hukum dasarnya, misalnya politik diterangkan
  sebagai hasil Kontrak Sosial. Kondisi ini merujuk pada cara
  menjelaskan fenomen di mana orang menggunakan abstraksi-abstraksi ide
  yang seolah-olah riil (misalnya \emph{aether} untuk merujuk pada ruang
  kosong (\emph{vacuum}) dan jiwa untuk menjelaskan manusia).
\item
  \emph{Kondisi Positif} merujuk pada masyarakat Abad ke-19, yaitu pada
  zaman Auguste Comte hidup. Masyarakat `normal' ini berlandaskan pada
  hasil-hasil temuan ilmiah (sains). Pada kondisi positif, masyarakat
  menemukan hukum-hukum konstan berdasarkan fenomen-fenomen yang telah
  diamati. Pada kondisi positif ini, orang meninggalkan cara berpikir
  yang mencari-cari sebab atau causa dan memuaskan diri sekadar pada
  relasi-relasi atau hukum-hukum yang dapat diamati dari fakta fakta
  yang diselidiki. Manusia tidak lagi berambisi mencari mengapa atau
  sebab terdasar sebuah fenomen muncul, cukup dengan deskripsi
  ``bagaimana'' relasi antar-fenomen yang ada dan dapat diamati ini
  terjadi. Kita dapat melihat di sini, bagaimana metafisika kehilangan
  tempatnya dalam cara berpikir dan pengetahuan manusia.
\end{enumerate}

\hypertarget{martin-heidegger-metafisika-sebagai-onto-teo-logi}{%
\subsection{Martin Heidegger: Metafisika sebagai
Onto-teo-logi}\label{martin-heidegger-metafisika-sebagai-onto-teo-logi}}

Pemikiran kedua yang menjadi penanda krisis metafisika adalah kritik
Heidegger atas metafisika sebagai \emph{onto-teo-logi}. Melalui
pembedaan ontologisnya, Heidegger menyatakan hahwa Metafisika Barat
memiliki berstrukturkan onto-teo-logis. Pembedaan ontologis menjadi
distingsi induk pemikiran Heidegger, yaitu antara Sein (\emph{Be}, Ada)
dan Seindes (\emph{beings}, adaan-adaan). Intuisi dasar Heidegger
mengenai `Onto-theo-logi' dan `Pembedaan Ontologis' ini dijelaskan dalam
\emph{Introduction to `What is Metaphysics'}, Pathmarks, hlm. 287.
Heidegger menuliskan sebagai berikut:

\begin{quote}
``Metaphysics states what beings are as beings. It offers a
\textbf{logos} (statement) about the \textbf{on} (beings). The latter
title `ontology' characterizes its essence, provided, of course that we
understand it in accordance with its proper significance and not through
its narrow Scholastic meaning. In this manner, metaphysics always
represents beings as such in their totality; it represents the
\emph{beingness of beings} (the \emph{ousia} of the \emph{on}). But
metaphysics represents the beingness of beings in a \emph{twofold
manner}; in the first place, the totality of beings as such with an eye
to their \emph{most universal} traits (\emph{on katholou},
\emph{kainon}); but at the same time also the totality of beings as such
in the sence of \emph{the highest and therefore divine} being (\emph{on
katholau}, \emph{akrotaton}, \textbf{thelon}). In the metaphysics of
Aristotle, the unconcealedness of beings as such has specifically
developed in this twofold manner.''
\end{quote}

\begin{quote}
``Because it represents beings as beings, metaphysics is, in a twofold
and yet unitary manner, the truth of beings in their universality and in
the highest being. \emph{According to its essence}, \textbf{metaphysics
at the same time both ontology in the narrower sense, and theology}.
This \emph{onto-theo-logical} essence of philosophy proper
{[}\emph{prote philosophia}, \emph{first philosophy}, \emph{filsafat
pertama}{]} must indeed be grounded in the way in which the \emph{on}
opens up in it, namely, as \emph{on}.''
\end{quote}

Dalam ``The Onto-Theo-Logical Constitution of Metaphysics,''
\emph{Identity and Difference}, New York: Harper Torchbooks, 1974, hlm.
54, Heidegger juga menulis:

\begin{quote}
``Western metaphyiscs, however, since its beginning with the Greeks has
eminently been both ontology and theology, still without being tied to
these rubrics. For this reason my inaugural lecture \emph{What is
Metaphysics} (1929) defines metaphysics as the question about beings as
such \emph{and} as a whole. The wholeness of this whole is the unity of
all beings that unifies as the generative ground. To those who can read,
\emph{this means: metaphysics is onto-theo-logy}.''
\end{quote}

Metafisika sebelum Heidegger dikenal sebagai pembicaraan tentang ``ada
sebagaimana ada-nya sendiri'' (\emph{being qua being}). Kata
\emph{being} merujuk pada apa saja, entah itu yang inteligibel maupun
sensibel. Melampaui apakah malaikat itu bisa dilihat atau tidak, apakah
meja itu bisa dilihat atau tidak, yang jelas malaikat dan meja itu
``ada.'' Jadi, metafisika tidak tertarik dengan aspek-aspek partikular
atas ``ada'' (aspek berat, panjang, dll). Metafisika mau mempertanyakan
dan membahas ``ada''-nya sendiri dari binatang atau malaikat itu
``sejauh ia ada.'' Binatang ``ada'' dan malaikat ``ada,'' dan terlepas
dari segala sudut pandang partikular atasnya, Metafisika mencari ``ada
sejauh ada'' itu sendiri.

Dalam kutipan di atas, Heidegger menunjukkan bahwa metafisika yang
seperti itu ternyata memiliki struktur \emph{onto-teo-logis}. Setelah
melampaui pembahasan binatang atau malaikat dari sudut ilmu partikular,
dan mencoba membahas ada-nya dalam dirinya sendiri, orang biasanya akan
tiba pada sebuah \emph{`ada' yang paling umum} atau \emph{universal},
atau pada sebuah \emph{`ada' yang paling ilahi}. `Ada' tersebut lalu
dijadikan sebagai `sebab akhir' atau sebagai `pengasal,' atau
`pencipta.' Ada sejauh `ada' (\emph{to on}) yang dicari-cari dan
ditemukan akhirnya ditempatkan sebagai ``yang paling umum'' dalam arti
yang paling tinggi atau yang paling ilahi (\emph{theos}). Heidegger
melihat metafisika semacam itu bersifat \emph{ontologis} sekaligus
\emph{teologis}. Semuanya itu kemudian dibungkus atau dibicarakan dalam
sebuah wacana atau uraian yang benar dan terstuktur (\emph{logos}).

Masalah lain dalam metafisika yang dilihat Heidegger adalah bahwa
metafisika selalu berpretensi atau berambisi menemukan ``kebenaran''
lewat representasi yang mereka temukan disekitarnya. Benda, terlepas
dari kesadaran kita, sudah mem-\emph{presentasi}-kan dirinya. Ketika
kita menyadari benda tersebut, maka yang terjadi adalah kita
me-\emph{representasi}-kan benda tersebut dalam kesadaran kita. Yang
namanya \emph{representasi} itu tidak pernah mengacu kepada keseluruhan
realitas benda. Representasi selalu \emph{mencerabut kekayaan realitas}
dan \emph{mereduksinya}. Di sini masalahnya: metafisika onto-teo-logis
selalu berambisi menemukan keseluruhan dan totalitas, padahal dalam
kesadaran atas realitas selalu ada reduksi.

Heidegger kemudian menyimpulkan bahwa metafisika yang seperti itu
sebenarnya tidak pernah sampai pada pembahasan `ada' dalam dirinya
sendiri. Kenapa begitu? Kenapa sulit sekali untuk membicarakan
\emph{Being qua being}? Menurut Heidegger, itu karena `ada' memang
bersifat \emph{mewujudkan sekaligus menyembunyikan dirinya}.

`Ada,' bagi Heidegger tidak pernah dapat habis diuraikan karena setiap
ia muncul, ia muncul pada kesadaran kita dalam bentuk sebuah
representasi. Karena selalu tereduksi, `Ada' dapat dikatakan mewahyukan
sekaligus menyembunyikan dirinya. Kebenaran dibalik ``Ada'' bagi
Heidegger adalah \emph{a-letheia}, yang artinya `tidak lupa,' `sejauh
tersingkap dari kelupaan,' atau lebih tepatnya:
\emph{ketaktersembunyian}. Bila kebenaran adalah \emph{a-letheia},
konsep ini dengan sendirinya mengandaikan adanya \emph{lethe} (sesuatu
yang masih menunggu untuk disingkapkan dari kelupaannya), yang Heidegger
sebut \emph{the unthought}.

Itulah mengapa supaya ``ada'' dalam dirinya sendiri dapat dibicarakan,
Heidegger mengatakan bahwa kita membutuhkan sebuah ruang pembicaraan
atau wacana yang baru, yang disebut \emph{Ontologi Fundamental}. Tujuan
ontologi fundamental bukanlah mencari kebenaran dari ``ada'' sebagaimana
termanifestasi dalam representasi-representasi atas ``ada'' tersebut,
melainkan masuk lebih dalam lagi dan membicarakan kembali ``ada'' dalam
dirinya sendiri. Heidegger mengajak kita beralih dari ``berpikir tentang
`ada' \emph{via} representasi'' ke ``berpikir tentang ada itu sendiri.''
Dalam ontologi fundamental, ada \emph{pembedaan ontologis}, yaitu
pembedaan antara \emph{Sein} (\emph{Be}, Ada) dan \emph{Seindes}
(\emph{beings}, adaan-adaan). Pembedaan inilah pokok utama dari
pemikiran Heidegger. Pembedaan Ontologis dapat dijelaskan dengan anekdot
berikut:

\begin{quote}
Bayangkan seseorang yang mau membeli buah-buahan. Ia pergi ke pasar,
lalu berkata pada pedagang buah: ``Saya mau membeli buah, Pak.''
Pedagang buah lalu menawarkan buah apel, buah pisang, buah mangga, buah
jeruk, dan buah delima, dan segala buah-buahan yang ada di warungnya.
Tetapi orang itu selalu menjawab ``Tidak, Pak. Saya mau membeli buah.''
Meskipun benda-benda yang ditawarkan pedagang toh juga buah-buahan,
``buah'' yang dicari oleh si pembeli tidak juga ketemu. Ternyata, orang
tersebut adalah seorang metafisikus, dan yang dicarinya adalah
\emph{kebuahan}. Itu adalah semacam karakter paling umum dan paling
mulia atau tinggi dari segala buah-buahan yang ada. Sedangkan ``buah''
hanyalah adalah nama generik dan umum yang merangkumi segala macam-macam
buah-buahan yang ada.
\end{quote}

Apakah seperti itu yang dimaksudkan Heidegger dengan ``Pembedaan
ontologis'' antara \emph{Be} dan \emph{being}? Apakah \emph{kebuahan}
merupakan \emph{Be}, dan \emph{buah-buah} partikular itu adalah
\emph{beings}? Bukan! Apakah kalau \emph{beings} itu semua dikumpulkan
lalu didapatkan \emph{Be}? Masih bukan! \emph{Be} yang dimaksud
Heidegger bukanlah kumpulan atau karakter umum dari \emph{beings}.
\emph{Be}-nya Heidegger dalam arti tertentu adalah yang mendasari, suatu
dasar yang menopang, yang menjadi \emph{ground}, yang
meng-\emph{ada}-kan semua \emph{beings}. Disebut `dalam arti tertentu'
karena \emph{Be} yang mendasari adanya \emph{beings} itu juga bukanlah
sebuah ``landasan.'' \emph{Be} tidak meng-ada-kan \emph{beings} melalui
hubungan kausal-kausal yang rasional (i.e.~karena ada \emph{Be}, maka
\emph{beings} dapat ada), melainkan lewat aktivitas ``perwahyuan'' atau
``ketersingkapan.''

Pembedaan ontologis ini menurut Heidegger berawal dari Platon dalam
penjelasannya tentang \emph{khora}. Konsep \emph{khora} merujuk pada
suatu \emph{gap} atau ruang di antara dunia \emph{Real} dengan dunia
\emph{Forma}. Disebut juga suatu rahim, atau matriks. Dalam
\emph{Timaeos} diceritakan tentang penciptaan dunia, yaitu bahwa dunia
itu diciptakan menurut apa yang Demiurgos bayangkan mengenai Paradeigma.
Khora adalah sebuah wadah atau tempat di mana yang sensibel (air, api,
udara, tanah) dibuat. Karena yang sensibel merupakan hasil bayangan
Demiurgos, maka ia tidak se-ideal \emph{paradeigma}. Khora ini juga di
sebut sebagai ``bentuk ke tiga'' (\emph{the 3rd kind}), di samping yang
sensibel dan inteligibel. Dalam pemikiran Heidegger, \emph{khora}
kemudian menjadi suatu ruang di antara \emph{Being} dan \emph{beings}.
Khora adalah \emph{ontological differance} itu sendiri, pembeda
ontologiss antara \emph{Being} dan \emph{beings}. Ia berasal dari yang
`lain,' \emph{the 3rd kind}, yang tidak bisa dipikirkan dan tidak bisa
dibicarakan.

Sebagai tambahan, menurut skema pembedaan ontologis Heidegger, apabila
\emph{Be} itu dikatakan Sang Pencipta, \emph{Ultima Ratio}, atau
\emph{Causa Sui}, maka \emph{Be} dengan demikian direpresentasikan,
diturunkan derajatnya menjadi sekadar beings pada umumnya. Sedangkan,
\emph{Be} dalam dirinya sendiri, yang lain sama sekali dari beings,
dilupakan, tidak dibahas! Itulah mengapa menurut Heidegger sejarah
filsafat Barat (dengan metafisikanya) adalah suatu sejarah pelupaan akan
\emph{Be} karena jatuh ke dalam struktur ontotheologis. Di mata
Heidegger, \emph{Be} yang selalu dilupakan oleh sejarah filsafat ini
adalah semacam ``kedalaman yang lebih dalam, yang tak terbahasakan, dan
tak terpikirkan oleh \emph{logos}.'' Menurut Heidegger, kita bisa
mengakses \emph{Be} ini bukan lagi lewat rasio-analitis yang memakai
logika, melainkan kita mesti memasuki ``\emph{meditative thinking}'' di
mana kita "hanya menunggu \emph{self-revelation} dari sang Ada itu.

\hypertarget{melampaui-metafisika}{%
\section{Melampaui Metafisika}\label{melampaui-metafisika}}

\begin{quote}
Lewat kritiknya atas Metafisika, Heidegger menawarkan sebuah arti baru
untuk aktivitas \emph{berpikir} yang meditatif. Bila Metafisika memang
mati, cara berpikir yang senantiasa mencari sesuatu yang \emph{beyond}
tidak pernah berakhir sebagaimana tampak dalam beberapa pemikir
kontemporer yang mengritik Heidegger, seperti Emmanuel Levinas dan Alain
Badiou.
\end{quote}

Dalam aktivitas berpikir yang meditatif, orang membuka diri dan
mendengarkan. Ia membuka diri pada realitas di hadaparnya dan
mendengarkan apa yang akan disingkapkan kepadanya. Tawaran Heidegger ini
beranjak dari pandangannya mengenai kebenaran, yang dihubungkan dengan
phusis Herakleitos (semacam alanı semesta, segala sesuatu yang ada, yang
lewat dan ada di sekitar manusia). Menurut Heidegger, pemahaman tentang
phusis ini dikacaukan oleh Platon dengan konsepnya tentang idea, oleh
Aristoteles dengan konsepnya tentang substansi, oleh Hegel dengan roh
absolut-nya, oleh Nietzsche dengan will to power-nya. Sejatinya, phusis
ini memilikd dua bagian, yang tampak/tak tersembunyi/a-letheia (legein)
dan yang tak tampak/tersembunyi/lethe (kruptein). Dari situ, kebenaran
bagi Heidegger bukan sesuatu yang terang-benderang, bukan homoiosis
(kesesuain), bukan korespondensi. Kebenaran adalah ketercerabutan dari
ketersembunyian, maka ia sekadar ``ketaktersembunyian'' (terjemahan
Heidegger tertiadap ale theia) oleh karena logos. Ada bagian dari phusis
yang tidak terungkapkan dan tetap tinggal tersembunyi. Dengan demikian,
di hadapan phusis, tidak mungkin manusia mencerap secara keseluruhan.
Maka, jalan terbaik adalah aktivitas berpikir yang meditatif. Di situ,
manusia membuka diri dan mendengarkan.

\hypertarget{kritik-levinas}{%
\subsection{Kritik Levinas}\label{kritik-levinas}}

Bagi Levinas, yang pertama bukanlah pencarian being qua being, melainkan
relasi etis kita dengan Yang Lain. Levinas mengkritik doktrin
Heideggerian tentang Ontologi Fundamental. Di matanya, ontologi sama
sekali tidak fundamental karena masih merupakan filsafat dari Yang Sama
(ontologie du Même), yaitu filsafat yang mereduksi kekayaan sesuatu yang
konkret melalui ``aktivitas mengetahui (menangkap, mencengkeram,
mencerabut)'' manusia, Inilah cara bekerja filsafat Barat, yaitu ketika
yang konkret direduksi sebagai pengetahuan. Levinas mengajukan sebuah
cara berfilsafat yang lain yang mendasarkan diri pada ``melihat Yang
Lain'' dan mengutamakan ``Yang Lain'' untuk mengembalikan
``irreduktibilitas Yang Lain''.

Reduksi transendental Husserl menunda (bracketting) semua penilaian
terhadap dunia dalam rangka menghindari representasionalisme. Namun
demikian, menurut Levinas reduksi ini tidak menunda kesadaran diri yang
selalu sadar akan sesuatu. Di situ, subjek tetap bertahan dalam fakta
bahwa ia mendasarkan diri pada sebuah keakuan absolut. Inilah kesadaran
diri konstitutif. Menurut Levinas, ada kesadaran diri yang mendahului
kesadaran diri konstitutif, yaitu kesadaran ciri non intensional, sebuah
kesadaran diri yang tanpa arah dan tujuan apa pun yang juga dapat
menjadi pengetahuan namun bukan pengetahuan dalam arti objektif yang
mengobjekkan apa yang diketahui dalam bentuk representasi). Kesadaran
non-reflekstif ini seperti ``orang asing,'' tanpa ``rumah,'' dan tanpa
``negara.'' Heidegger memang sudah menghilangkan ``ego'' menjadi
\emph{Dasein}, tetapi tetap saja ontologi fundamental berkutat pada
filsafat Yang Sama.

Levinas melihat bahwa proyek Heidegger sebetulnya terpusat pada soal
pemahaman, yaitu Dasein yang mau memahami segala sesuatunya dalam
cakrawala \emph{Be} (\emph{Sein}). Levinas menyebut hal ini sebagai
totalitas. Heidegger memberikan prioritas kepada Be daripada beings,
artinya dari kacamata Levinas, beings (yang salah satunya adalah
manusia) hanyalah sarana untuk pemahaman akan Be. Filsafat dari Platon
(Idea) sampai Hegel (Roh Absolut) menjadikan kita jatuh dalam sikap
``teoretis,'' berelasi dengan dunia dan manusia secara Impersonal, tidak
lagi berelasi secara konkret. Filsafat Neutrum ini berbahaya, karena
mengorbankan begitu saja eksistensi konkret demi Idea atau Roh Absolut
Filsafat demikian mengabaikan relasi etis (sebuah relasi langsung, mula
ketemu muka). Relasi dengan orang lain tidak dapat direduksi dalam
sekadar pemahaman, apalagi pemahamanku yang selalu egologis.

Kata ontologi bagi Levinas bermakna buruk la hendak merestorasi kata
``metafisika.'' Dalam pandangan Levinas, Metafisika merujuk pada sesuatu
yang benar-benar ``beyond, meta'' (yang mengajak keluar dan tanpa kembai
ke kita), merujuk pada sesuatu yang benar-benar transenden, bukan
pura-pura transenden tetapi lalu balik ke diri kita lagi. Metafisika
adalah relasi ets, sebuah hasrat keluar, yang terarah pada sesuatu yang
lain absolutely other), yang tidak dapat ditundukkan pada pemahaman
egologis. Ini adalah sebuah relasi yang bersifat ``mendasar,'' yang
tidak dapat distrukturkan secara sadar. Inilah relasi etis, relasi tanpa
relasi, pra-reflektif, yang adalah ``tanggung jawab kepada orang lain''.

\hypertarget{bahan-bahan-rujukan-3}{%
\subsection{Bahan-bahan Rujukan}\label{bahan-bahan-rujukan-3}}

\begin{enumerate}
\def\labelenumi{\arabic{enumi}.}
\tightlist
\item
  Anton Bakker. \emph{Ontologi: Metafisika Umum. Filsafat Pengada dan
  Dasar-dasar Kenyataan}. Yogyakarta: Kanisius, 1992.
\item
  A. Setyo Wibowo. ``Filsafat Pertama Bukan Ontologi Melainkan Etika.''
  Bahan Kuliah Metafisika: ``Emmanuel Levinas''.
\item
  A. Setyo Wibowio. ``Kuliah Metafisika: Heidegger''. Teks 2008, revisi
  akhir Februari 2011. STF Driyarkara, Jakarta.
\item
  A. Setyo Wibowo. ``Positivisme Auguste Comte.'' Disampaikan pada
  kuliah Sejarah Pemikiran Modern, 22 Februari 2010 di STF Driyarkara,
  Jakarta.
\end{enumerate}
\end{document}
